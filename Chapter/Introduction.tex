\usection{Введение}
Множество решений системы алгебраических уравнений и его обобщения~— проективные и~квазипроективные многообразия,
схемы и~абстрактные алгебраические многообразия~— являются центральными объектами алгебраической геометрии.
Особенный интерес представляют случаи, когда на~алгебраическом многообразии имеется дополнительная алгебраическая структура,
например, группы. Такие объекты называются алгебраическими группами.

\textit{Алгебраической подгруппой} группы $G$ называется подгруппа, также являющаяся
алгебраическим многообразием, то~есть задаваемая системой полиномиальных уравнений.

Наиболее известным примером являются кубики~— плоские алгебраические кривые, задаваемые уравнением третьего порядка.
Однако групповой закон на~эллиптических кривых устроен крайне нетривиально. Кроме того, часто важно сохранить структуру
из~объемлющего пространства, к~примеру, из~алгебраического тора $(K^\times)^n$ для~поля $K$.

В~свете этого, другим естественным примером алгебраической группы служит множество решений системы биномиальных уравнений вида
\[
  z_1^{\alpha_1} z_2^{\alpha_2} \ldots z_n^{\alpha_n} = z_1^{\beta_1} z_2^{\beta_2} \ldots z_n^{\beta_n},
\]
исследование параметрических заданий (параметризаций) которых и~составляет \textit{цель} настоящей работы.
