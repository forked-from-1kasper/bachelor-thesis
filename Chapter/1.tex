\section{Задание алгебраических подгрупп тора}
Как показывает следующая теорема \cite{Schm94}, такие группы, на~самом деле, исчерпывают все алгебраические многообразия,
глобально наследующие групповую структуру тора $(K^\times)$.

Пусть $G$~— некоторая группа. Целочисленная степень элемента группы $g \in G$ определяется стандартным образом:
\[
  g^k =
  \begin{cases}
    g \cdot g^{k - 1}, & k > 0; \\
    1, & k = 0; \\
    g^{-1} \cdot g^{k + 1}, & k < 0.
  \end{cases}
\]

Для~векторов $\alpha = (\alpha_1, \ldots, \alpha_n) \in \Z^n$ и~$z = (z_1, \ldots, z_n) \in G^n$ (где $G$~— группа)
обозначим $z^\alpha = z_1^{\alpha_1} \cdot z_2^{\alpha_2} \cdot \ldots \cdot z_n^{\alpha_n}$.

\begin{theorem*}[\notewrap{Шмидт}]
  Пусть $K$~— поле. Всякая алгебраическая подгруппа $H$ группы $(K^{\times})^n$ задаётся
  системой некоторого числа $N$ биномиальных уравнений, то~есть существуют $N$ таких показателей $\alpha_i, \beta_i \in \Z^n$, что
  \[
    H = \{ z \in (K^{\times})^n\ |\ \forall 1 \leq i \leq N.\ z^{\alpha_i} = z^{\beta_i} \}.
  \]
\end{theorem*}

Далее приведём замкнутое (не~зависящее от~других крупных теорем) доказательство этой теоремы.
Для~этого нам понадобится вспомогательное утверждение \cite{Art48}.

\subsection{Теорема Артина}
Обозначим как~$\Hom(G, H)$ множество гомоморфизмов между группами $G$ и~$H$.

Пусть $G$~— группа, $K$~— поле, а~$K^{\times}$~— его мультипликативная группа.
В~таком случае произвольный гомоморфизм $f \in \Hom(G, K^{\times})$ называют \textit{характером.}
Говорят, что характеры $f_1, \ldots, f_n$ \textit{линейно независимы,}
если
\[
  \forall \alpha_1, \ldots, \alpha_n \in K.\ \alpha_1 f_1 + \ldots + \alpha_n f_n = 0 \Rightarrow \alpha_1 = \ldots = \alpha_n = 0.
\]

\begin{theorem*}[\notewrap{Артин}]
  Любые $n$ попарно различных характеров линейно независимы.
\end{theorem*}

\begin{proof}

Докажем индукцией по~числу характеров $n$.

Возьмём произвольный характер $f$. Поскольку он является гомоморфизмом,
то~$f(1_G) = 1$, где $1_G$~— единица в~группе $G$. Но~тогда если $\alpha f = 0$,
то~и~$\alpha = \alpha \cdot 1 = \alpha \cdot f(1_G) = 0$, что доказывает базу индукции.

Пусть теперь утверждение теоремы верно для~любых $n$ различных характеров. Докажем его для~$(n + 1)$ характера.
Пусть $\alpha_1 f_1 + \alpha_2 f_2 + \ldots + \alpha_n f_n + \alpha_{n + 1} f_{n + 1} = 0$.
Зафиксируем произвольный $y \in G$. Тогда для~любого $x \in G$ имеем:
\begin{equation}\label{Artin:first}
  \alpha_1 f_1(yx) + \alpha_2 f_2(yx) + \ldots + \alpha_n f_n(yx) + \alpha_{n + 1} f_{n + 1}(yx) = 0.
  %\sum_{i = 1}^{n + 1} \alpha_i f_i(yx) = 0.
\end{equation}
Поскольку все $f_i$~— гомоморфизмы, $f_i(yx) = f_i(y) f_i(x)$, а~потому:
\[
  \alpha_1 f_1(y) f_1(x) + \ldots + \alpha_n f_n(y) f_n(x) + \alpha_{n + 1} f_{n + 1}(y) f_{n + 1}(x) = 0.
\]
С~другой стороны, для~$x \in G$ также верно, что:
\[
  \alpha_1 f_1(x) + \alpha_2 f_2(x) + \ldots + \alpha_n f_n(x) + \alpha_{n + 1} f_{n + 1}(x) = 0.
\]
Умножим это равенство на~$f_{n + 1}(y)$:
\begin{equation}\label{Artin:second}
  \alpha_1 f_{n + 1}(y) f_1(x) + \ldots + \alpha_n f_{n + 1}(y) f_n(x) + \alpha_{n + 1} f_{n + 1}(y) f_{n + 1}(x) = 0.
\end{equation}
Вычитая \eqref{Artin:first} из~\eqref{Artin:second}, получаем:
\[
  \alpha_1 (f_{n + 1}(y) - f_1(y)) f_1(x) + \ldots + \alpha_n (f_{n + 1}(y) - f_n(y)) f_n(x) = 0.
\]
Тогда, в~силу произвольности $x$, получаем линейную комбинацию $n$ характеров:
\[
  \alpha_1 (f_{n + 1}(y) - f_1(y)) f_1 + \ldots + \alpha_n (f_{n + 1}(y) - f_n(y)) f_n = 0.
\]

Но~в~таком случае, по~индуктивной гипотезе, $\alpha_i (f_{n + 1}(y) - f_i(y)) = 0$.
Теперь, выбрав для~каждого $i = 1, \ldots, n$ такое $y$, что $f_{n + 1}(y) \neq f_i(y)$
(это возможно, потому~что, по~условию теоремы, все $f_i$ попарно различны), получим,
что $\alpha_1 = \alpha_2 = \ldots = \alpha_n = 0$.

Таким образом, с~учётом вышесказанного, $\alpha_{n + 1} f_{n + 1} = \alpha_1 f_1 + \alpha_2 f_2 + \ldots + \alpha_n f_n + \alpha_{n + 1} f_{n + 1} = 0$.
Согласно базе индукции, $\alpha_{n + 1} = 0$.
\end{proof}

\subsection{Доказательство теоремы Шмидта}
\textit{Многочленом Лорана} над~кольцом $K$ называется функция $(K^\times)^k \rightarrow K$ вида
\[
  z \mapsto \sum_{i \in I} a_i z^{\alpha_i},
\]
где $a_i \in K$ и~$I \subseteq \Z^k$~— конечный набор индексов.

\begin{proof}
  Пусть $I \subseteq \Z^n$~— множество индексов,
  а~$P_j(z) = \sum_{i \in I} a_{ji} z^i$~— многочлены ($k$ штук), задающие подгруппу:
  \[
    H = \{ z \in (K^{\times})^n\ |\ \forall 1 \leq j \leq k.\ P_j(z) = 0 \}.
  \]

  Поскольку $z_1^{i} z_2^{i} = (z_1 z_2)^i$, отображение $z \mapsto z^i$ определяет
  характер $\chi_i \in \Hom(H, K^{\times})$.

  Рассмотрим на~множестве $I$ отношение эквивалентности $i \sim j \Leftrightarrow \chi_i = \chi_j$.
  Оно разбивает $I$ на~$m$ классов $I_1$, $I_2$, …, $I_m$. В~таком случае, для~всех $i_1, i_2 \in I_j$ верно, что $\chi_{i_1} = \chi_{i_2}$.
  Обозначим этот характер, соответствующий каждому $I_k$, как $\chi_k = \chi_{i_1} = \chi_{i_2}$.

  Собрав подобные слагаемые при~каждом $\chi_k$ в~многочленах $P_j$, получим:
  \[
    P_j = \sum_{i \in I} a_{ji} \chi_i = \sum_{k = 1}^{m} \left( \sum_{i \in I_k} a_{ji} \right) \chi_k.
  \]

  Но~$P_j$ равны нулю на~всём $H$, поэтому:
  \[
    \sum_{k = 1}^{m} \left( \sum_{i \in I_k} a_{ji} \right) \chi_k = 0.
  \]

  Все $\chi_k$ попарно различны по~их заданию, поэтому к~ним применима теорема Артина. Из~неё заключаем, что:
  \[
    \sum_{i \in I_k} a_{ji} = 0.
  \]

  Наконец, пусть $N_k$~— мощность $I_k$, $N = \sum_{k = 1}^m (N_k - 1)$ и~$I_k = \{i_{k, 1},\allowbreak i_{k, 2},\allowbreak \ldots,\allowbreak i_{k, N_k}\}$.
  $I_k$ были взяты так, что характеры $\chi_{i_{k, i}}$ и~$\chi_{i_{k, j}}$ совпадают на~$H$ для~фиксированного $k$ и~любых $i$ и~$j$.
  Другими словами, это означает, что всякий элемент $z \in H$ удовлетворяет для~всех $1 \leq i \leq N_k$ и~$1 \leq j \leq N_k$
  уравнениям $z^{i_{k, i}} = z^{i_{k, j}}$.

  Поскольку равенство рефлексивно, симметрично и~транзитивно, система всех уравнений $z^{i_{k, i}} = z^{i_{k, j}}$ равносильна
  системе $N$ уравнений, составленной из~подряд идущих индексов:
  \[
    z^{i_{1, 1}} = z^{i_{1, 2}}, z^{i_{1, 2}} = z^{i_{1, 3}}, \ldots, z^{i_{1, N_1 - 1}} = z^{i_{1, N_1}}, z^{i_{2, 1}} = z^{i_{2, 2}}, \ldots, z^{i_{m, N_m - 1}} = z^{i_{m, N_m}}.
  \]
  Множество решений этой системы обозначим как $A$.

  Покажем, что $A$ совпадает с~$H$. Действительно, пусть $z \in H$. Тогда, как уже отмечалось выше,
  по построению $I_k$ выполнено $z^{i_{k, j}} = \chi_k(z) = z^{i_{k, j + 1}}$; то~есть $z \in A$, и~$H \subseteq A$.

  Наоборот, пусть $z \in A$. Тогда, по~заданию $A$, $z^{i_{k, j_1}} = z^{i_{k, j_1 + 1}} = \ldots = z^{i_{k, j_2 - 1}} = z^{i_{k, j_2}}$
  для~любых $1 \leq j_1 \leq j_2 \leq N_k$. Выберем в~каждом $I_k$ по~представителю $i_k \in I_k$ и~соберём подобные слагаемые:
  \[
    P_j(z) = \sum_{k = 1}^{m} \left( \sum_{i \in I_k} a_{ji} \right) z^{i_k} = \sum_{k = 1}^{m} 0 \cdot z^{i_k} = 0,
  \]
  что означает $z \in H$, то~есть $A \subseteq H$.

  Таким образом, $H = A$, поэтому в~качестве $\alpha_i$ и~$\beta_i$ достаточно взять подряд идущие индексы во~всех $I_k$.
\end{proof}

