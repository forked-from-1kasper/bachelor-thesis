\section{Дальнейшие перспективы}
Важнейшим инструментом в~комплексном анализе является понятие \textit{вычета} голоморфной функции.
Напомним основные сведения из~одномерной теории вычетов.

Множество функций, голоморфных в~открытом множестве $D \subseteq \complex$ (или, более общо, в~$D \subseteq \complex^n$), обозначим как $\Hol(D)$.

\begin{theorem*}
  Пусть $D \subseteq \complex$~— область. Если $f \in \Hol(D)$, а~кусочно-гладкие кривые $\gamma_1$
  и~$\gamma_2 \subseteq D$ гомотопны в~$D$, то
  \[
    \int_{\gamma_1} f(z)dz = \int_{\gamma_2} f(z)dz.
  \]
\end{theorem*}

Рассмотрим область $D$ и~её подмножество $D^\prime = D \setminus \{ z_1, \ldots, z_k \} \subseteq D$, где $z_i \in D$.

Из~теоремы в~частности следует, что для~произвольной $f \in \Hol(D^\prime)$ и~её изолированной
особой точки $z_0 \in D \setminus D^\prime$ интегралы $\oint_{\Gamma_\epsilon} f(z)dz$ по~окружностям
$\Gamma_\epsilon = \{ z \in D \ | \ |z - z_0| = \epsilon \}$ не~зависят от~$\epsilon \in \R_{+}$
при~достаточно малом $\epsilon$ (конкретнее, таком, что в~круг $\{ z \in D \ | \ |z - z_0| < \epsilon \}$
не~попадают никакие другие особые точки $f$, кроме выбранной $z_0$).

В~таком случае \textit{вычетом} $f$ в~точке $z = z_0$ называют интеграл
\[
  \operatorname*{res}\limits_{z_0} f = \operatorname*{res}\limits_{z = z_0} f(z) = \lim\limits_{\epsilon \to 0} \frac{1}{2 \pi i} \oint_{\Gamma_\epsilon} f(z)dz = \frac{1}{2 \pi i} \oint_{\Gamma_{\epsilon_0}} f(z)dz,
\]
где $\epsilon_0$ достаточно мало в~описанном выше смысле.

Следующая теорема позволяет сводить вычисление интеграла от~голоморфной функции к~вычислению вычетов в~её особых точках,
что вместе со~сравнительной простотой вычисления обосновывает их практическое применение.
\begin{theorem*}[\notewrap{основная о вычетах}]
  Рассмотрим односвязную область $D \subseteq \complex$, замкнутую кусочно-гладкую кривую $\gamma \subseteq D'$,
  конечный набор точек $z_i \in D$ и~функцию $f \in \Hol(D^\prime)$, где $D^\prime = D \setminus \{ z_1, \ldots, z_k \}$.
  Тогда:
  \[
    \oint_{\gamma} f(z)dz = 2 \pi i \sum_{k = 1}^{n} \mathrm{Ind}_{\gamma}(z_k) \operatorname*{res}\limits_{z_k} f,
  \]
  где $\mathrm{Ind}_{\gamma}(z)$~— индекс точки $z$ относительно кривой $\gamma$.
\end{theorem*}

Для~построения описанных конструкций существенным фактом было то, что одномерные голоморфные функции допускают
существование \textit{изолированных} особых точек. Однако, как показывает следующая теорема \cite{ShaII}, в~многомерном случае
это допущение неверно.

\begin{theorem*}[\notewrap{о стирании компактных особенностей}]
  Пусть $D \subseteq \complex^n$~— область, причём $n > 1$, $\overline{K} \subseteq D$,
  $D \setminus K$ связно. Тогда всякая $f \in \Hol(D \setminus K)$
  голоморфно продолжается в~$D$.
\end{theorem*}

Таким образом, в~случае $n > 1$ невозможно вокруг особой точки голоморфной функции
выбрать малую сферу так, чтобы функция была голоморфна в~окрестности этой сферы (в~противном случае,
функцию можно было~бы продолжить внутрь сферы). Следовательно, любая такая сфера содержит
особые точки, а~потому интеграл по~ней, в~общем случае, не~определён.

В~отличие от~многомерных голоморфных функций, голоморфные отображения, то~есть функции $\complex^n \rightarrow \complex^n$,
\textit{могут} иметь изолированные нули или особые точки. Этот факт приводит к~построению так называемого \textit{вычета Гротендика.}

\subsection{Вычет Гротендика}
Зафиксируем набор многочленов Лорана $g_1, \ldots, g_n$ над~$\complex$.
Пусть $g(z) = g_1(z) \ldots g_n(z)$ и~$\Gamma = \{ z \in (\complex^\times)^n \ | \ g(z) = 0 \}$.
Также зафиксируем некоторое изолированное решение $z \in \Gamma$ системы $g_1(z) = \ldots = g_n(z) = 0$ и~его окрестность $U \subseteq (\complex^\times)^n$.
Для~всякого $\epsilon \in \R_{+}^n$ определим
\[
  \gamma_{z, \epsilon} = \{ z \in (\complex^\times)^n\ | \ \forall i.\ |g_i(z)| = \epsilon_i \}.
\]

Почти для~всех $\epsilon$ множество $\gamma_{z, \epsilon}$ представляет собой гладкое вещественное подмногообразие многообразия $U$.
Для~достаточно малых ненулевых $\epsilon$ многообразие $\gamma_{z, \epsilon}$ является компактным подмногообразием $U \setminus \Gamma$.
Определим на~нём ориентацию с~помощью дифференциальной формы $d(\mathrm{Arg}\ g_1) \wedge \ldots \wedge d(\mathrm{Arg}\ g_n)$.

\textit{Циклом Гротендика} $\gamma_z$ \cite{GelKho02} изолированного корня $z$ системы $g_1(z) = \ldots = g_n(z) = 0$ назовём многообразие $\gamma_{z, \epsilon}$ для~достаточно малого $\epsilon$.
Класс цикла в~$n$-мерной группе гомологий $U \setminus \Gamma$ не~зависит от~выбора $\epsilon$, но,~очевидно, зависит от~порядка уравнений $g_1(z) = 0, \ldots, g_n(z) = 0$.

Зафиксируем также ещё один многочлен Лорана $f$. \textit{Вычетом Гротендика} назовём следующий интеграл:
\[
  \operatorname*{res}\limits_{z} \frac{f}{g_1 \ldots g_n} = \frac{1}{(2 \pi i)^n} \int_{\gamma_z} \frac{f}{g_1 \ldots g_n}.
\]

Существенную трудность \cite{Tsikh92} при~его вычислении представляет устройство цикла $\gamma_z$.
В~случае $g_i(z) = z$ цикл $\gamma_z$ является просто $n$-мерным тором, и~вычисление вычета
сводится к~вычислению коэффициента в~ряде~Лорана, как и~одномерном случае; однако в~общем случае
нетривиальна даже топологическая структура $\gamma_z$. Выдвигаем гипотезу, что для~изучения
структуры этого цикла (например, с~целью замены на~гомологичную ему линейную комбинацию торов)
можно использовать аппарат кобордизмов и~теорию Морса.

\subsection{Кобордизмы и теория Морса}
Обозначим $\R^n_+ = \{ (x_1, \ldots, x_n) \in \R^n\ |\ x_n \geq 0 \}$.
\textit{$n$-мерным} \textit{$k$-гладким} \textit{многообразием} называем топологическое пространство $(X, \tau)$
вместе с~\textit{гладкой структурой} $A$, состоящей из~пар $(U, h)$, где $U \in \tau$
(называемое координатной окрестностью) и~$h$~— гомеоморфизм между $U$ и~открытым
подмножеством пространства $\R^n_+$, удовлетворяющей следующим условиям:
\begin{enumerate}
  \item Координатные окрестности покрывают $X$.
  \item Если $(U_1, h_1), (U_2, h_2) \in \tau$, то~$h_1 \circ h_2^{-1} : h_2(U_1 \cup U_2) \rightarrow \R^n_+$~— $k$-гладкое отображение
        (называемое отображением перехода).
  \item Набор $A$ максимален по~отношению к~предыдущему условию; то~есть если $(U, h) \notin A$,
  то~это условие не~выполняется для~набора $A \cup \{(U, h)\}$.
\end{enumerate}

Обычно под~гладким (без~указания класса гладкости) многообразием понимают $\infty$-гладкое многообразие.

\textit{Границей многообразия $W$,} которую обозначают $\partial W$, называют множество точек,
не~имеющих окрестности, гомеоморфной $\R^n$. \textit{Триадой} гладких многообразий \cite{Mil65} называют
тройку $(W; V_0, V_1)$, где~$W$~— компактное гладкое многообразие, причём $V_0 \cap V_1 = \varnothing$
и~$\partial W = V_0 \cup V_1$.

\textit{Кобордизмом} между гладкими многообразиями $M_0$ и~$M_1$ называем пятёрку
$(W;\allowbreak V_0,\allowbreak V_1;\allowbreak h_0,\allowbreak h_1)$ такую,
что $(W; V_0, V_1)$~— триада гладких многообразий, а~$h_i : V_i \rightarrow M_i$~— диффеоморфизмы.

Так называемые \textit{функции Морса} предоставляют один из~наиболее важных инструментов в~теории
кобордизмов. Пусть зафиксированы гладкое многообразие $W$ и~гладкая функция $f \in C^{\infty}(W)$.
Точка $p \in W$ называется \textit{критической точкой} $f$, если $(\nabla f)(p) = 0$ в~некоторой
координатной окрестности. Такая точка называется \textit{невырожденной,} если
\[
  \det\left(\left.\frac{\partial^2 f}{\partial x_i \partial x_j}\right|_{p}\right)_{i,j} \neq 0.
\]

\textit{Функцией Морса} для~триады $(W; V_0, V_1)$ называется гладкая функция $f : W \rightarrow [a; b]$
такая, что:
\begin{enumerate}
  \item $f^{-1}(a) = V_0$ и~$f^{-1}(b) = V_1$.
  \item Все критические точки $f$ невырожденные и~лежат в~$W \setminus \partial W$.
\end{enumerate}
Функция Морса позволяет, в~определённом смысле, разбить $W$ на~составные части. Так, например,
если $0 < t_1 < \ldots < t_n < 1$~— критические точки функции Морса $f$,
то~для~любых $t, t' \in (t_i;\ t_{i + 1})$ множества $f^{-1}([0; t))$ и~$f^{-1}([0; t'))$
являются диффеоморфными многообразиями; то~есть топология $f^{-1}([0; t))$ при~увеличении
параметра $t$ меняется только при~переходе через критические точки, что позволяет исследовать
структуру $W$ через критические точки его функции Морса.

Локальное поведение критических точек регулируется леммой Морса.
\begin{lemma*}[\notewrap{Морса}]
  Если $p$~— невырожденная критическая точка $f$, то~существует система координат
  в~окрестности $p$ такая, что $f(x_1, \ldots, x_n) = \mathrm{const} - x_1^2 - \ldots - x_\lambda^2 +
  x_{\lambda + 1}^2 + \ldots + x_n^2$ для~некоторого натурального $\lambda$ между $0$ и~$n$.
\end{lemma*}

Число $\lambda$ при~этом называется \textit{индексом} критической точки. В~каноническом случае $n = 2$
и~$\lambda = 1$ получаем представление $f(x_1, x_2) = C - x_1^2 + x_2^2$ для~константы $C \in \R$.
Тогда прообразы $f^{-1}(\varepsilon)$ задаются уравнением $x_1^2 - x_2^2 = C - \varepsilon = \varepsilon'$,
то~есть представляют собой семейство гипербол.

Поскольку это уравнение триномиальное, по~теореме Шмидта, подгруппу тора оно не~определяет.
Однако если сделать замену переменных $x_1 \mapsto (u + v) \sqrt{\varepsilon'} / 2$
и~$x_2 \mapsto (u - v) \sqrt{\varepsilon'} / 2$, мы получим:
\[
\begin{aligned}
  x_1^2 - x_2^2 &= \frac{(u + v)^2 \varepsilon'}{4} - \frac{(u - v)^2 \varepsilon'}{4} \\
                &= \frac{(u^2 + 2uv + v^2 - u^2 + 2uv - v^2)\varepsilon'}{4} \\
                &= uv \varepsilon';
\end{aligned}
\]
то~есть в~новой системе координат $f^{-1}(\varepsilon)$ задаётся \textit{биномиальным} уравнением $uv = 1$,
а~потому является алгебраической подгруппой тора, что позволяет нам использовать разработанную теорию,
например, перейти в~этой окрестности к~мономиальной параметризации.

Кроме того, если кобордизм \textit{элементарный} ($f$ имеет ровно одну критическую точку), то, как доказывается
в~главе~3 в~\cite{Mil65}, гипербола для~$\varepsilon = -1$ представляет порождающий элемент группы относительных
гомологий $H_\lambda(W, V_0)$. Исследование групп относительных гомологий возвращает нас к~интегралам, введённым
в~предыдущем подразделе. Выдвигаем гипотезу, что подобная процедура может быть проделана для~случаев произвольных
размерности $n$ и~индекса $\lambda$.

