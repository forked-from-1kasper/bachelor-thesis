\section{Мономиальные параметризации и их свойства}
Для~абелевой группы $G$ и~векторов $\alpha_1, \ldots, \alpha_n \in \Z^k$ рассмотрим отображение $\phi_\alpha(t) = (t^{\alpha_1}, t^{\alpha_2}, \ldots, t^{\alpha_n})$
из~$G^k$ в~$G^n$, где $\alpha$~— матрица со~строками $\alpha_i$. Поскольку $(t_1 \cdot t_2)^{\alpha_i} = t_1^{\alpha_i} \cdot t_2^{\alpha_i}$,
$\phi_\alpha$~— гомоморфизм.

Далее увидим, что если $K$~— поле, то~отображение $\phi_\alpha$ для~группы $G = K^\times$ задаёт \textit{параметризацию} некоторой алгебраической группы,
которую мы будем называть \textit{мономиальной.}

\begin{statement*}
  $\phi_{E} = 1_{G^n}$, где $E$~— единичная матрица $n \times n$, $1_X$~— тождественная функция на~множестве $X$.
\end{statement*}

\begin{statement*}
  $
    \forall \alpha \in \Z^{k \times m}, \beta \in \Z^{m \times n}.\ \phi_\alpha \circ \phi_\beta = \phi_{\alpha \beta}
  $.
\end{statement*}

\begin{proof}
  Действительно, рассмотрим $t \in G^n$. Тогда $i$-я компонента вектора $\phi_\alpha(\phi_\beta(t))$
  равна
  $ \phi_\beta^{\alpha_i}(t) = (t^{\beta_1})^{\alpha_i^1} \cdot \ldots \cdot (t^{\beta_m})^{\alpha_i^m}
                             = t_1^{\beta_1^1 \alpha_i^1 + \ldots + \beta_m^1 \alpha_i^m} \cdot
                               \ldots \cdot
                               t_n^{\beta_1^n \alpha_i^1 + \ldots + \beta_m^n \alpha_i^m}
                             = t^{(\alpha \beta)_i}.
  $
\end{proof}

Вместе эти два утверждения составляют условие на~функториальность. Более точно, рассмотрим категорию
$\mathrm{Matr}(R)$ матриц над~кольцом $R$, объектами которой являются натуральные числа, а~стрелками
между числами $m$ и~$n$~— матрицы $R^{n \times m}$, и~категорию $\mathrm{Grp}$ малых групп,
объектами которой являются малые группы, а~стрелками~— гомоморфизмы между ними.
Тогда определён функтор $\phi : \mathrm{Matr}(\Z) \rightarrow \mathrm{Grp}$:
\[
  \phi = \begin{cases}
    k \mapsto G^k; \\
    \alpha \mapsto \phi_\alpha.
  \end{cases}
\]

Как показывает лемма \ref{lemma:faithful}, в~полях нулевой характеристики этот функтор~— строгий
(подробнее общее понятие функтора изложено в~приложении).

\begin{lemma}
\label{lemma:infiniteMulOrder}
  Пусть $K$~— поле, причём $\mathrm{char}(K) = 0$.
  Тогда $K^\times$ содержит элемент бесконечного порядка.
\end{lemma}

\begin{proof}
  Действительно, рассмотрим $2 = 1 + 1 \in K^\times$. Тогда для~всякого $n > 0$:
  \[
      2^n - 1 = (1 + 1)^n - 1 = \sum_{k = 0}^n C^n_k 1^k 1^{n - k} - 1
                              = \sum_{k = 0}^n C^n_k \cdot 1 - 1
                              = \left(\sum_{k = 1}^n C^n_k \right) \cdot 1.
  \]

  Справа имеем сумму $\sum_{k = 1}^n C^n_k > 0$ единиц. Поскольку $\mathrm{char}(K) = 0$,
  она не равна нулю, то~есть $2^n - 1 \neq 0$.
\end{proof}

Как $e_i$ обозначим \textit{естественный базис} над~$\Z^k$, то~есть такой вектор, у~которого на~$i$-м месте единица, а~на~остальных~— нули.
$j$-ю компоненту вектора $\alpha_i$ обозначим как $\alpha_i^j$, а~вектор, состоящий из~$j$-х компонент,
как $\alpha^j = (\alpha_1^j, \alpha_2^j, \ldots, \alpha_n^j)$.

\begin{lemma}[\notewrap{строгость}]
\label{lemma:faithful}
  $
    \mathrm{char}(K) = 0 \Rightarrow \forall \alpha, \beta \in \Z^{n \times k}.\ \phi_\alpha = \phi_\beta \Leftrightarrow \alpha = \beta
  $.
\end{lemma}

\begin{proof}
  Импликация справа налево очевидна.

  Наоборот, пусть $\phi_\alpha = \phi_\beta$. По~лемме \ref{lemma:infiniteMulOrder} зафиксируем
  элемент $z \in K^\times$ бесконечного мультипликативного порядка.
  Тогда для~всякого $1 \leq i \leq n$ верно, что
  \[
    (z^{\alpha_1^i}, \ldots, z^{\alpha_k^i}) = \phi_\alpha(z \cdot e_i) = \phi_\beta(z \cdot e_i) = (z^{\beta_1^i}, \ldots, z^{\beta_k^i}).
  \]

  Таким образом, $z^{\alpha_j^i} = z^{\beta_j^i}$, то~есть $z^{\alpha_j^i - \beta_j^i} = 1$.
  $z$ имеет бесконечный мультипликативный порядок, поэтому $\alpha_j^i - \beta_j^i = 0$.
\end{proof}

\subsection{Инъективность мономиальных параметризаций}
Нетрудно получить необходимое условие инъективности отображения $\phi_\alpha$.
Для~этого заметим, что очевидно следующее утверждение.

\begin{statement*}
  $\Span_\Z(\alpha_1, \ldots, \alpha_n) = \Z^k \Leftrightarrow \forall i.\ e_i \in \Span_\Z(\alpha_1, \ldots, \alpha_n)$.
\end{statement*}

\begin{lemma}
  Если $\alpha_i \in \Z^k$ порождают всю решётку, то~$\phi_\alpha$ инъективно.
\end{lemma}

\begin{proof}
  Достаточно показать, что $\ker(\phi_\alpha) = \{1\}$, поскольку $\phi_\alpha$~— гомоморфизм.
  Возьмём $t \in G^k$ такой, что $\phi_\alpha(t) = 1$, и~докажем, что $t = 1$.

  Действительно, поскольку $\alpha_i$ порождают всю решётку, каждый $e_j$ выражается как линейная комбинация векторов $\alpha_i$:
  \[
    e_j = b^j_1 \alpha_1 + \ldots + b^j_n \alpha_n.
  \]

  $\phi_\alpha(t) = 1$ означает, что $t^{\alpha_i} = 1$ для~всех $1 \leq i \leq n$.
  Зафиксируем $1 \leq j \leq k$ и~возведём это равенство в~степень $b^j_i$: $t^{b^j_i \alpha_i} = (t^{\alpha_i})^{b^j_i} = 1^{b^j_i} = 1$.
  Наконец, перемножим полученные равенства:
  \[
    t_j = t^{e_j} = t^{b^j_1 \alpha_1 + \ldots + b^j_n \alpha_n} = t^{b^j_1 \alpha_1} \ldots t^{b^j_n \alpha_n} = 1 \cdot \ldots \cdot 1 = 1,
  \]
  что и~требовалось.
\end{proof}

\begin{example*}
  $(t_1, t_2) \mapsto (t_1^2 t_2^3, t_1 t_2^2)$ инъективно над~$\complex$,
  поскольку $(1, 0) = 2 \cdot (2, 3) + (-3) \cdot (1, 2)$ и~$(0, 1) = -1 \cdot (2, 3) + 2 \cdot (1, 2)$.
\end{example*}

Теперь покажем, что полученное в~лемме необходимое условие является также и~достаточным для~группы $G = \torus$.

Будем пользоваться существованием для~целочисленных матриц \textit{нормальной формы Смита} \cite{Smth60}.
Пусть $\diag^{m \times n}_r(x_1, \ldots, x_r)$~— диагональная матрица $\diag(x_1, \ldots, x_r)$,
дополненная (или обрезанная) справа снизу нулями до~матрицы размера $m \times n$.

\begin{theorem*}[\notewrap{о существовании нормальной формы Смита}]
  Для~всякой матрицы $\alpha \in \Z^{m \times n}$ существуют
  пара матриц $\beta_1 \in \GL^m(\Z)$ и~$\beta_2 \in \GL^n(\Z)$,
  а~также набор чисел $\varepsilon_1, \ldots, \varepsilon_r \in \Z \setminus \{0\}$,
  такие, что:
  \[
    \beta_1 \alpha \beta_2 = \diag^{m \times n}_r(\varepsilon_1, \ldots, \varepsilon_r),
  \]
  причём $\varepsilon_1 \divides \varepsilon_2 \divides \ldots \divides \varepsilon_r$ (так называемые инвариантные факторы) и~$r = \rank(\alpha)$.
\end{theorem*}

Поскольку числа $\varepsilon_1$, $\ldots$, $\varepsilon_r$ выбираются единственным образом с~точностью
до~обратимого элемента, то~есть в~случае $\Z$ с~точностью до~$\pm 1$, можем всегда выбрать их положительными.
Для~них матрицу $\diag^{m \times n}_r(\varepsilon_1, \ldots, \varepsilon_r)$ обозначим как $\SNF(\alpha)$.

Помимо этого, с~учётом функториальности $\phi$, ясно, что $\phi_\beta$ биективно, если $\beta \in \GL^n(\Z)$.
Действительно, $\phi_\beta \circ \phi_{\beta^{-1}} = \phi_{\beta \beta^{-1}} = \phi_E = 1_{(K^\times)^n}$
и~$\phi_{\beta^{-1}} \circ \phi_\beta = \phi_{\beta^{-1} \beta} = \phi_E = 1_{(K^\times)^n}$.
Поэтому имеет место следующая лемма.

\begin{lemma}
\label{lemma:injectivityOutOfSNF}
  $\phi_\alpha : (\complex^\times)^k \rightarrow (\complex^\times)^n$ инъективна тогда и~только тогда, когда
  \[
    \SNF(\alpha) = \diag^{n \times k}_k(1, 1, \ldots, 1).
  \]
\end{lemma}

\begin{proof}
  Пользуясь теоремой, возьмём матрицы $\beta_1 \in \GL^n(\Z)$ и~$\beta_2 \in \GL^k(\Z)$
  такие, что $\beta_1 \alpha \beta_2 = \SNF(\alpha)$.
  В~таком случае, $\alpha = \beta_1^{-1} \SNF(\alpha) \beta_2^{-1}$, и~$\phi_\alpha = \phi_{\beta_1^{-1} \SNF(\alpha) \beta_2^{-1}}
                                                                                    = \phi_{\beta_1^{-1}} \circ \phi_{\SNF(\alpha)} \circ \phi_{\beta_2^{-1}}$.

  Поскольку, по~замечанию выше, $\phi_{\beta_1^{-1}}$ и~$\phi_{\beta_2^{-1}}$ биективны, то~$\phi_\alpha$
  инъективно тогда и~только тогда, когда инъективно $\phi_{\mathrm{SNF(\alpha)}}$.

  $\phi_{\mathrm{SNF(\alpha)}}(t_1, \ldots, t_k) = (t_1^{\varepsilon_1}, \ldots, t_r^{\varepsilon_r}, 1, \ldots, 1)$,
  но~$t \mapsto t^\varepsilon$ инъективно над~$\complex^\times$ только в~случае $\varepsilon = \pm 1$
  (иначе $t^\varepsilon = t^\varepsilon \cdot 1 = t^\varepsilon \cdot u^\varepsilon = (tu)^\varepsilon$,
  где~$u \neq 1$~— нетривиальный корень из~единицы степени $\varepsilon$).
  В~силу выбора знаков, $\phi_{\mathrm{SNF(\alpha)}}$ инъективна лишь когда $\varepsilon_1 = \ldots = \varepsilon_r = 1$
  и~$r = k \leq n$, что и~требовалось.
\end{proof}

Сформулируем ещё одну вспомогательную теорему \cite{TsikhSad14}.
\textit{$\Z$-линейную оболочку} векторов $\alpha_1, \ldots, \alpha_n$ обозначим как
\[
  \Span_\Z(\alpha_1, \ldots, \alpha_n) = \{ k_1 \alpha_1 + \ldots + k_n \alpha_n \ | \ k_1, \ldots, k_n \in \Z \}.
\]

Если $\Span_\Z(\alpha_1, \ldots, \alpha_n) = \Z^k$, то~говорим, что $\alpha_1, \ldots, \alpha_n$ \textit{порождают всю решётку} $\Z^k$.

\begin{theorem}
\label{theorem:TsikhSadykov}
  Пусть $\alpha \in \Z^{n \times k}$. Следующие утверждения равносильны:
  \begin{enumerate}
    \item Строки матрицы $\alpha$ порождают всю решётку $\Z^k$.
    \item Миноры максимальной размерности матрицы $\alpha$ взаимно просты.
    \item $\SNF(\alpha) = \diag^{n \times k}_k(1, \ldots, 1)$.
  \end{enumerate}
\end{theorem}

Таким образом, из~леммы \ref{lemma:injectivityOutOfSNF} и~теоремы \ref{theorem:TsikhSadykov}
получаем необходимое и~достаточное условие инъективности $\phi_\alpha$.

\begin{theorem}
\label{theorem:InjectivityCondition}
  $\phi_\alpha$ инъективно над~полем $K = \complex$ тогда и~только тогда, когда~$\alpha_i$ порождают всю решётку $\Z^k$.
\end{theorem}

\subsection{Алгебраичность и существование}
Как и~в~случае теории кривых и~поверхностей, говорим, что подгруппа алгебраического тора \textit{параметризуется,}
если она представима как образ некоторого отображения. Аналогично, подгруппа \textit{параметризуется мономиально,}
если она представима как образ отображения $\phi_\alpha$ для~некоторой матрицы $\alpha$.

Поскольку $\phi_\alpha$~— гомоморфизм, его полный образ $\image(\phi_\alpha)$ является подгруппой в~$(K^\times)^n$.
Изучим два вопроса: является~ли $\image(\phi_\alpha)$ алгебраической подгруппой (алгебраичность) и~всякая~ли алгебраическая подгруппа задаётся некоторым $\phi_\alpha$ (существование).

Для~этого, прежде всего, заметим, что система биномиальных уравнений $z^{\beta'_i} = z^{\beta''^i}$ эквивалентна
системе $z^{\beta'_i - \beta''_i} = 1$, то~есть в~точности ядру оператора $\phi_{\beta'_i - \beta''_i}$;
поэтому первый вопрос эквивалентен тому, можно~ли данный гомоморфизм $\phi_\alpha$ достроить
до~точной последовательности $(K^\times)^k \xrightarrow[]{\phi_\alpha} (K^\times)^n \xrightarrow[]{\phi_\beta} (K^\times)^m$.

\begin{lemma}
\label{lemma:leftInvExactSequence}
  Пусть $G$~— группа, $H$~— абелева группа, $f \in \Hom(G, H)$, $g \in \Hom(H, G)$ и~$g \circ f = 1_G$.
  Тогда последовательность $G \xrightarrow[]{f} H \xrightarrow[]{f \circ g - 1_H} H$ точна.
\end{lemma}

\begin{proof}
  Действительно, $(f \circ g - 1_H) \circ f = f \circ g \circ f - f = f - f = 0$, то~есть $\image(f) \subseteq \ker(f \circ g - 1_H)$. 
  Обратно, пусть $h \in \ker(f \circ g - 1_H)$. Тогда $f(g(h)) - h = 0 \Leftrightarrow h = f(g(h))$, то~есть $h \in \image(f)$,
  и~$\ker(f \circ g - 1_H) \subseteq \image(f)$.
\end{proof}

\begin{lemma}
  $\phi_\alpha$, где $\alpha \in \Z^{n \times k}$, представимо как некоторый образ $\image(\phi_\beta)$,
  если отображения $g \mapsto g^{\varepsilon_i}$ сюръективны в~$G$ для~всех $\varepsilon_i$ из~$\SNF(\alpha)$.
\end{lemma}

\begin{proof}
  Снова представим $\alpha$ в~виде $\alpha = \beta_1^{-1} \varepsilon \beta_2^{-1}$, где
  \[
    \varepsilon = \SNF(\alpha) = \diag^{n \times k}_r(\varepsilon_1, \ldots, \varepsilon_r).
  \]

  Рассмотрим $\delta = \diag^{n \times r}_r(1, \ldots, 1)$ и~$\alpha' = \beta^{-1} \delta$.
  $\phi_{\alpha'}$ инъективна как композиция инъективных функций. С~другой стороны, очевидно, что:
  \[
    \image(\phi_\alpha) = \phi_{\beta_1^{-1}} (\phi_\varepsilon (\phi_{\beta_2^{-1}} (G^k)))
                        = \phi_{\beta_1^{-1}} (\phi_\varepsilon (G^k))
                        = \phi_{\beta_1^{-1}} (\phi_{\delta} (G^r))
                        = \image(\phi_{\alpha'}).
  \]

  Нетрудно видеть, что $\alpha'$ имеет левой обратной матрицу $\beta' = \delta^{\top} \beta$,
  а~потому гомоморфизм $\phi_{\alpha'}$ имеет левым обратным гомоморфизм $\phi_{\beta'}$.
  Поскольку $(\phi_{\alpha'} \circ \phi_{\beta'}) \phi_{-E} = \phi_{\alpha' \beta' - E}$, остаётся лишь применить лемму \ref{lemma:leftInvExactSequence}.
\end{proof}

\begin{theorem}
\label{theorem:AlgebraicityCondition}
  Всякая параметризация $\phi_\alpha$, где $\alpha \in \Z^{n \times k}$, задаёт алгебраическую подгруппу $(K^\times)^n$,
  если поле $K$ алгебраически замкнуто.
\end{theorem}

\begin{example*}
  Над~алгебраически незамкнутым полем теорема не~выполняется: образ $t \mapsto t^2$ над~$\R$
  представляет собой положительный луч $\R_{>0}$. Как известно, множество корней вещественного многочлена либо конечно,
  либо совпадает со~всем $\R$, поэтому указанная параметризация не~определяет алгебраическое множество.
\end{example*}

Ответ на~второй вопрос несколько сложнее. Например, уравнение $z^n = 1$ задаёт алгебраическую
группу для~любого $n \in \Z$, состоящую из~$n$ точек на~$\complex^\times$; но~если $n \neq 0, \pm 1$,
то~легко видеть, что она не~может быть параметризована никакой $\phi_\alpha$.

Действительно, $\phi_\alpha$~— голоморфное отображение, $(\complex^\times)^k$~— открытое множество,
поэтому всякая проекция полного образа $\image(\phi_\alpha) = \phi_\alpha((\complex^\times)^k)$ либо, в~силу
открытости непостоянных голоморфных отображений, открыта, либо состоит из~одной точки; но~$z^n = 1$
открыто, только если $n = 0$, и~состоит из~одной точки, только если $n = \pm 1$.

Более того, из~этих рассуждений видно, что~по~всякой системе биномиальных уравнений можно построить новую систему,
которая гарантированно не~параметризуется мономиально, возведя, например, произвольное уравнение в~степень $n > 1$.

Чтобы понять, при~каких условиях существует мономиальная параметризация, сначала заметим, что группа $\image(\phi_\alpha)$
изоморфна группе $(\torus)^r$. Действительно, в~доказательстве теоремы мы видели, что для~всякого $\phi_\alpha$
можно построить инъективный $\phi_{\alpha'} : (\torus)^r \rightarrow (\torus)^n$ такой, что $\image(\phi_\alpha) = \image(\phi_{\alpha'})$;
но~тогда $\phi_{\alpha'}$ и~задаёт искомый изоморфизм. Таким образом, верно следующее утверждение.

\begin{statement*}
  Для~любой матрицы $\alpha \in \Z^{n \times k}$ группа $\image(\phi_\alpha)$ изоморфна алгебраическому тору размерности не~более чем $k$.
\end{statement*}

Кроме того, поскольку отображение $\phi_{\alpha'}$ полиномиально, оно, в~частности, непрерывно в~естественной топологии на~$K = \complex$.
Множество $(\complex^\times)^r$ связно, а, как известно, непрерывный образ связного множества связен, в~силу чего верно ещё одно утверждение.

\begin{statement*}
  $\image(\phi_\alpha) \subseteq (\complex^\times)^n$ связно.
\end{statement*}

Теперь отметим, что группа $z^n = 1$ связна в~точности тогда, когда $n = 0, \pm 1$.
Это соображение указывает на~то, что критерием существования мономиальной параметризации
для~алгебраической подгруппы тора является связность.

Группу корней степени $n$ из~единицы, задаваемую уравнением $z^n = 1$ в~поле $K$, обозначим как $\omega_K(n) \subseteq K^\times$.
Для~вектора $x \in \Z^k$ также обозначим $\omega_K(x) = \omega_K(x_1) \times \ldots \times \omega_K(x_k) \subseteq (K^\times)^k$.
Для~краткости будем писать $\omega(x) = \omega_\complex(x)$. Известно, что группа $\omega(x)$ изоморфна
группе $\Z / x_1 \Z \times \ldots \times \Z / x_k \Z$.

Чтобы доказать описанный выше критерий, рассмотрим число $\Pi(\alpha) = |\omega_K(\varepsilon)|$, где $|G|$~— порядок группы $G$.
$|G \times H| = |G| |H|$, поэтому $\Pi(\alpha) = |\omega_K(\varepsilon_1)| \cdot \ldots \cdot |\omega_K(\varepsilon_r)|$.
Так как $\omega(n)$ изоморфно $\Z / n \Z$, в~случае поля $K = \complex$ выполняется
равенство $\Pi(\alpha) = \varepsilon_1 \cdot \ldots \cdot \varepsilon_r$.

\begin{lemma}
\label{lemma:exactOutOfPi}
  Если $\Pi(\beta) = 1$, то~существует такое $\alpha$, что $\ker(\phi_\beta) = \image(\phi_\alpha)$.
\end{lemma}

\begin{proof}
  Пусть $H = \ker(\phi_\beta)$. Без~потери общности считаем, что строки матрицы $\beta$ линейно независимы над~$\Z$
  (ясно, что строки, выражающиеся как линейная комбинация остальных строк, можно убрать из~матрицы $\beta$ без~изменения ядра).

  Запишем для~$\beta$ нормальную форму Смита: $\beta = \beta_1^{-1} \varepsilon \beta_2^{-1}$.
  Поскольку, по~замечанию выше, $\beta$ имеет полный ранг, матрица $\varepsilon$ не~имеет нулевых строк.
  Кроме того, $\Pi(\beta) = 1$, поэтому $|\omega_K(\varepsilon_i)| = 1$, то~есть уравнения $z^{\varepsilon_i} = 1$
  имеют решением только $z = 1$.

  Ядро $\varepsilon$ задаётся векторами вида $(0, \ldots, 0, t_{k + 1}, \ldots, t_{n})$.
  Рассмотрим матрицу $\delta \in \Z^{n \times (n - k)}$, соответствующую линейному оператору
  \[
    (t_1, \ldots, t_{n - k}) \mapsto (0, \ldots, 0, t_1, \ldots, t_{n - k}).
  \]

  Пусть также $\alpha = \beta_2 \delta$. Докажем, что $\image(\phi_\alpha) = \ker(\phi_\beta)$.

  Действительно, по~заданию $\delta$, $\varepsilon \delta = 0$, поэтому 
  $\phi_{\beta} \circ \phi_{\alpha} = \phi_{\beta_1^{-1} \varepsilon \beta_2^{-1} \beta_2 \delta} = \phi_{\beta_1^{-1} \varepsilon \delta} = \phi_{0} = 1$,
  то~есть $\image(\phi_{\alpha}) \subseteq \ker(\phi_{\beta})$.

  Обратно, пусть $z \in \ker(\phi_\beta)$. Тогда $\phi_{\beta_1^{-1}} (\phi_{\varepsilon \beta_2^{-1}}(z)) = \phi_\beta(z) = 1$,
  откуда $\phi_{\varepsilon}(\phi_{\beta_2^{-1}}(z)) = \phi_{\varepsilon \beta_2^{-1}}(z) = \phi_{\beta_1}(1) = 1$.
  Из~вида матрицы $\varepsilon$ получаем, что $\phi_{\beta_2^{-1}}(z) = (1, \ldots, 1, t_{k + 1}, \ldots, t_n) = \phi_\delta(t_{k + 1}, \ldots, t_n)$;
  то~есть для~некоторых $t_j$ выполнено $z = \phi_{\beta_2}(\phi_\delta(t_{k + 1}, \ldots, t_n)) = \phi_\alpha(t_{k + 1}, \ldots, t_n)$.
  Таким образом, $z \in \image(\phi_\alpha)$, и~$\ker(\phi_\beta) \subseteq \image(\phi_\alpha)$.
\end{proof}

\begin{theorem}
\label{theorem:connectedComponentsNumbers}
  Число связных компонент $\ker(\phi_\beta) \subseteq (\torus)^n$ равно $\Pi(\beta)$.
\end{theorem}

\begin{proof}
  Снова рассмотрим нормальную форму для~$\beta$: $\beta$ = $\beta_1^{-1} \varepsilon \beta_2^{-1}$.
  $\phi_{\beta_1^{-1}}$~— изоморфизм, поэтому $\ker(\phi_{\beta}) = \ker(\phi_{\varepsilon \beta_2^{-1}})$.

  Строки матрицы $\beta_2^{-1}$ обозначим как $b_i$.
  Для~вектора $u \in \omega(\varepsilon)$ рассмотрим множество $H_u = \{z \in (\torus)^n \ | \ \forall 1 \leq i \leq r.\ z^{b_i} = u_i\}$.
  Условие $\phi_\varepsilon(\phi_{\beta_2^{-1}}(z)) = 1$, очевидно, эквивалентно условию $\exists u \in \omega(\varepsilon).\ z \in H_u$.
  Ясно, что множества $H_u$ дизъюнктны, поэтому $\ker(\phi_{\beta})$ распадается в~дизъюнктное объединение:
  \[
    \ker(\phi_{\beta}) = \bigsqcup_{u \in \omega(\varepsilon)} H_u.
  \]

  Векторов $u \in \omega(\varepsilon)$ ровно $\Pi(\beta)$ штук, поэтому достаточно показать, что каждая компонента $H_u$ связна.

  Поскольку, по~определению, $H_1 = \ker(\phi_{\delta \beta_2^{-1}})$, где $\delta = \diag^{k \times n}_r(1, \ldots, 1)$,
  и~$\Pi(\delta \beta_2^{-1}) = 1$, компонента $H_1$, по~замечанию, связна.

  Зафиксировав $u$, рассмотрим матрицу $\tau = \diag(1 / u_1, \ldots, 1 / u_r, 1, \ldots, 1)$ и~отображение $\psi = \phi_{\beta_2} \circ \phi_\tau \circ \phi_{\beta_2^{-1}}$.
  $\psi$~— непрерывная биекция, причём $\psi^{-1} = \phi_{\beta_2} \circ \phi_{\tau^{-1}} \circ \phi_{\beta_2^{-1}}$.
  По~определению $\psi$, $\phi_{\beta_2^{-1}}(\psi(z)) = \phi_{\tau \beta_2^{-1}}(z)$.

  Если теперь $z \in H_u$, то~ $\psi(z)^{b_i} = z^{b_i} / u_i = u_i / u_i = 1$. Обратно, если $z \in H_1$,
  то~$\psi^{-1}(z)^{b_i} = u_i z^{b_i} = u_i$. Таким образом, $\psi(H_u) = H_1$,
  то~есть $H_u$ гомеоморфно $H_1$, но~$H_1$ связно.
\end{proof}

Как мы видели ранее, всякая алгебраическая подгруппа $H$ тора $(K^\times)^n$ представляется как $\ker(\phi_\beta)$
для~некоторого $\beta$. Заметим, что $\Pi(\beta)$ не~зависит от~выбора $\beta$ для~группы $H$.

\begin{theorem}
  Если $\mathrm{char}(K) = 0$ и~$\ker(\phi_{\beta}) = \ker(\phi_{\beta'})$, то~$\Pi(\beta) = \Pi(\beta')$.
\end{theorem}

\begin{proof}
  Запишем нормальные формы Смита: $\beta = \beta_1^{-1} \varepsilon \beta_2^{-1}$
  и~$\beta' = {\beta'}_1^{-1} \varepsilon' {\beta'}_2^{-1}$.
  Тогда $\ker(\phi_{\varepsilon \beta_2^{-1}}) = \ker(\phi_{\beta}) = \ker(\phi_{\beta'}) = \ker(\phi_{\varepsilon' {\beta'}_2^{-1}})$.

  Пусть $\varepsilon = \diag^{k \times n}_r(\varepsilon_1, \ldots, \varepsilon_r)$
  и~$\varepsilon' = \diag^{k' \times n}_{r'}(\varepsilon'_1, \ldots, \varepsilon'_r)$.
  Обозначим $\delta = \diag^{k \times n}_r(1, \ldots, 1)$ и~$\delta' = \diag^{k' \times n}_{r'}(1, \ldots, 1)$.

  Поскольку $\Pi(\delta \beta_2^{-1}) = \Pi(\delta' {\beta'}_2^{-1}) = 1$, обе подгруппы согласно лемме \ref{lemma:exactOutOfPi}
  параметризуются некоторыми $\phi_\alpha$ и~$\phi_{\alpha'}$ соответственно.

  $\image(\phi_\alpha) = \ker(\phi_{\delta \beta_2^{-1}}) \subseteq \ker(\phi_{\varepsilon \beta_2^{-1}}) = \ker(\phi_{\varepsilon' {\beta'}_2^{-1}})$,
  поэтому $\phi_{\varepsilon' {\beta'}_2^{-1}} \circ \phi_\alpha = 1$.
  В~силу леммы \ref{lemma:faithful}, $\varepsilon' {\beta'}_2^{-1} \alpha = 0$.
  Умножив обе части равенства на~матрицу $\diag(1 / \varepsilon'_1, \ldots, 1 / \varepsilon'_r, 1, \ldots, 1)$, получим, что $\delta' {\beta'}_2^{-1} \alpha = 0$;
  но~это значит, что
  \[
    \ker(\phi_{\delta \beta_2^{-1}}) = \image(\phi_\alpha) \subseteq \ker(\phi_{\delta' {\beta'}_2^{-1}}).
  \]
  Совершенно аналогично получим обратное включение. В~силу экстенсиональности, $\ker(\phi_{\delta \beta_2^{-1}}) = \ker(\phi_{\delta' {\beta'}_2^{-1}})$.

  Рассмотрим фактор-группу $\ker(\phi_\beta) / \ker(\phi_{\delta \beta_2^{-1}}) = \ker(\phi_{\beta'}) / \ker(\phi_{\delta' {\beta'}_2^{-1}})$,
  называемую также \textit{группой компонент.}
  Отображение \[ \phi_{\delta \beta_2^{-1}} : \ker(\phi_\beta) \rightarrow \omega_K(\varepsilon) \times \{ 1 \} \times \ldots \times \{ 1 \} \]
  индуцирует инъективный гомоморфизм из~фактора. Помимо этого, в~теореме \ref{theorem:connectedComponentsNumbers} была построена биекция между компонентами $H_u$ (которая обобщается
  без~изменений на~случай произвольного поля $K$), а~потому все они непусты.
  Это означает, что $\phi_{\delta \beta_2^{-1}}$ сюръективно, а~потому индуцированное отображение тоже сюръективно.
  Аналогично проводятся рассуждения для~группы $\omega_K(\varepsilon')$.
  Таким образом, получаем цепочку изоморфизмов:
  \[
    \omega(\varepsilon) \cong \ker(\phi_\beta) / \ker(\phi_{\delta \beta_2^{-1}}) = \ker(\phi_{\beta'}) / \ker(\phi_{\delta' {\beta'}_2^{-1}}) \cong \omega(\varepsilon').
  \]

  Окончательно, $\Pi(\beta) = |\omega_K(\varepsilon)| = |\omega_K(\varepsilon')| = \Pi(\beta')$,
  так как изоморфные группы имеют одинаковые порядки.
\end{proof}

Таким образом, для~всякой алгебраической подгруппы $H$ тора можно определить число $\Pi(H)$
равное $\Pi(\beta)$ для~всякого $\beta$ такого, что $H = \ker(\phi_\beta)$.

Теперь для~всякой алгебраической подгруппы $H \subseteq (\complex^\times)^n$ определим \textit{компоненту единицы} $H^\circ$
как компоненту связности, содержащую нейтральный элемент группы. Из~теоремы \ref{theorem:connectedComponentsNumbers} следует,
что $\Pi(H^\circ) = 1$. Таким образом, доказаны следующие утверждения.

\begin{theorem}
\label{theorem:ExistenceCondition}
  Алгебраическая подгруппа $H$ тора $(K^\times)^n$ параметризуема тогда и~только тогда, когда $\Pi(H) = 1$.
\end{theorem}

\begin{consequence}
\label{consequence:ExistenceCondition}
  Алгебраическая подгруппа $H$ тора $(\complex^\times)^n$ параметризуема тогда и~только тогда, когда является связной.
\end{consequence}

\begin{consequence}
  Всякая алгебраическая подгруппа $H$ тора $(\complex^\times)^n$ содержит параметризуемую подгруппу $H^\circ$ той~же размерности.
\end{consequence}

\begin{proof}
  Все связные компоненты $H$ гомеоморфны, что немедленно доказывает теорему.
\end{proof}

