\documentclass{article}

\usepackage[utf8]{inputenc}
\usepackage[T2A]{fontenc}
\usepackage[russian]{babel}
\usepackage[tmargin=1in,bmargin=1in,lmargin=1.25in,rmargin=1.25in]{geometry}

\usepackage[hidelinks,unicode]{hyperref}
\usepackage[threshold=0]{csquotes}
\usepackage{indentfirst}
\usepackage{enumerate}
\usepackage{totcount}
\usepackage{titlesec}
\usepackage{etoolbox}
\usepackage{amsmath}
\usepackage{amssymb}
\usepackage{amsthm}
\usepackage{hhline}
\usepackage{array}

\newtheorem*{consequence*}{Следствие}
\newtheorem*{theorem*}{Теорема}
\newtheorem*{lemma*}{Лемма}

\def\paddedtext#1#2{\leavevmode\hbox to#1{\hss#2\hss}\ignorespaces}
\patchcmd{\thebibliography}{\section*{\refname}}{}{}{}

\begin{document}

\titlelabel{\thetitle.\:}

\begin{titlepage}
  \centering

  {\LargeТема:\vspace{0.2cm}}\par
  {\largeТеория Морса и кобордизмов, применение к изучению полиэдров Вейля}
\end{titlepage}

\tableofcontents

\section{Введение}

\section{Вычет Гротендика}

Зафиксируем набор многочленов Лорана $g_1, \ldots, g_n$.
Обозначим $g(z)~= g_1(z) \ldots g_n(z)$ и~$\Gamma~= \{ z \in \mathbb{C}^n \ | \ g(z) = 0 \}$.

Также зафиксируем некоторое изолирование решение $z \in \Gamma$ системы $g_1(z) = \ldots = g_n(z) = 0$ и~его окрестность $U \subseteq \mathbb{C}^n$.
Почти для~всех точек $\epsilon \in \mathbb{R}_{+}^n$ множество $\gamma_{z, \epsilon} = \{ z \in \mathbb{C}^n \ | \ \forall i{:}\ |g_i(z)| = \epsilon_i \}$~—
гладкое вещественное подмногообразие $U$. Для~достаточно малых ненулевых $\epsilon$ многообразие $\gamma_{z, \epsilon}$ является компактным
подмногообразием $U \setminus \Gamma$. Определим на~нём ориентацию с~помощью дифференциальной формы $d(\mathrm{Arg}\ g_1) \wedge \ldots \wedge d(\mathrm{Arg}\ g_n)$.

\textit{Циклом Гротендика} $\gamma_z$ изолированного корня $z$ системы $g_1(z) = \ldots = g_n(z) = 0$ назовём многообразие $\gamma_{z, \epsilon}$ для~достаточно малого $\epsilon$.
Класс цикла в~$n$-мерной группе гомологий $U \setminus \Gamma$ не~зависит от~выбора $\epsilon$, но,~очевидно, зависит от~порядка уравнений $g_1(z) = 0, \ldots, g_n(z) = 0$.

Зафиксируем также ещё один многочлен Лорана $f$. \textit{Вычетом Гротендика} назовём следующий интеграл:
$$
  \mathrm{res}_{z} \frac{f}{g_1 \ldots g_n} = \frac{1}{(2 \pi i)^n} \int_{\gamma_z} \frac{f}{g_1 \ldots g_n} dz_1 \ldots dz_n
$$

\section{Кобордизмы и теория Морса}

\section{Алгебраические подгруппы тора}

\section{Заключение}

\section{Приложение. Категории с кобордизмами}

\end{document}