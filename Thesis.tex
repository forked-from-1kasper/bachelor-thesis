\documentclass[a4paper,oneside]{article}

\usepackage[utf8]{inputenc}
\usepackage[T2A]{fontenc}
\usepackage[russian]{babel}
\usepackage[tmargin=20mm,bmargin=20mm,lmargin=30mm,rmargin=10mm]{geometry}

\usepackage[fontsize=14pt]{fontsize}

\usepackage[hidelinks,unicode]{hyperref}
\usepackage[threshold=0]{csquotes}
\usepackage{indentfirst}
\usepackage{enumerate}
\usepackage{totcount}
\usepackage{titlesec}
\usepackage{etoolbox}
\usepackage{amsmath}
\usepackage{amssymb}
\usepackage{tikz-cd}
\usepackage{hhline}
\usepackage{array}
\usepackage{url}

\usepackage{tocloft}

\usepackage{amsthm}

\newtheorem{statement}{Утверждение}
\newtheorem{consequence}{Следствие}
\newtheorem{theorem}{Теорема}
\newtheorem{lemma}{Лемма}

\newtheorem*{statement*}{Утверждение}
\newtheorem*{consequence*}{Следствие}
\newtheorem*{theorem*}{Теорема}
\newtheorem*{lemma*}{Лемма}

\newcommand{\divides}{\mid}

\newcommand{\N}{\mathbb{N}}
\newcommand{\Z}{\mathbb{Z}}
\newcommand{\Q}{\mathbb{Q}}
\newcommand{\R}{\mathbb{R}}

\newcommand{\primes}{\mathbb{P}}
\newcommand{\complex}{\mathbb{C}}
\newcommand{\quaternion}{\mathbb{H}}

\newcommand{\torus}{\complex^\times}

\newcommand{\image}{\mathrm{Im}}

\newcommand{\Hom}{\mathrm{Hom}}
\newcommand{\Hol}{\mathrm{Hol}}

\newcommand{\diag}{\mathrm{diag}}

\newcommand{\rank}{\mathrm{rank}}
\newcommand{\Span}{\mathrm{Span}}

\newcommand{\GL}{\mathrm{GL}}
\newcommand{\SNF}{\mathrm{SNF}}

\def\paddedtext#1#2{\leavevmode\hbox to#1{\hss#2\hss}\ignorespaces}
\patchcmd{\thebibliography}{\section*{\refname}}{}{}{}


\def\title{Алгебраические подгруппы комплексного тора}
\hypersetup{pdfinfo={Title=\title}}

\newtotcounter{citenum}
\def\oldbibitem{} \let\oldbibitem=\bibitem
\def\bibitem{\stepcounter{citenum}\oldbibitem}

\urlstyle{same}
\allowdisplaybreaks

\setlength{\parindent}{12.5mm}
\linespread{1.3}

\begin{document}

\titlelabel{\thetitle.\:}

\begin{titlepage}
  \centering
  Министерство науки и высшего образования РФ\par
  Федеральное государственное автономное\par
  образовательное учреждение высшего образования\par
  «СИБИРСКИЙ ФЕДЕРАЛЬНЫЙ УНИВЕРСИТЕТ»\par

  \vspace{0.3cm}
  Институт математики и фундаментальной информатики\par
  Кафедра теории функций\par

  \vspace{0.5cm}
  \begin{flushright}
    \begin{tabular}{l@{}}
      \textbf{\MakeUppercase{Утверждаю}}\\
      Заведующий кафедрой\\
      \hspace{1.5cm} / А.~К.~Цих\\
      «\underline{\hspace{0.9cm}}» \underline{\hspace{1.5cm}} 2024 г.
    \end{tabular}
  \end{flushright}

  \vspace{1cm}
  \textbf{\MakeUppercase{Бакалаврская работа}}

  \vspace{0.1cm}
  \textbf{Направление} 01.03.01 Математика

  \vspace{0.6cm}
  \textbf{\MakeUppercase{\title}}

  \vspace{1.5cm}
  \begin{tabular}{m{0.22\linewidth}m{0.13\linewidth}m{0.36\linewidth}m{0.18\linewidth}}
    Руководитель   & \rule{0pt}{40pt} & профессор, доктор физико-{\allowbreak}математических наук & А.~К.~Цих    \\ %\hline
    Выпускник      & \rule{0pt}{40pt} &                                                           & Н.~А.~Мишко  \\ %\hline
    Нормоконтролер & \rule{0pt}{40pt} &                                                           & Т.~Н.~Шипина
  \end{tabular}

  \vfill
  Красноярск, 2024
\end{titlepage}

\titleformat*{\section}{\bfseries}
\titleformat*{\subsection}{\bfseries}

\section*{\centering\MakeUppercase{Реферат}}
\thispagestyle{empty}

Выпускная квалификационная работа по~теме «\title»
содержит \pageref{TheEnd}~страниц текста, 1~приложение, \total{citenum}~использованных источников.

???
\pagebreak

\renewcommand{\contentsname}{\hfill\bfseries\normalsize\MakeUppercase{Содержание}\hfill}
\renewcommand{\cftaftertoctitle}{\hfill}
\renewcommand{\cftsecleader}{\cftdotfill{\cftdotsep}}
\tableofcontents
\thispagestyle{empty}

\pagebreak

\usection{Введение}

<история вопроса?>

Множество решений системы алгебраических уравнений и его обобщения~— проективные и~квазипроективные многообразия,
схемы и~абстрактные алгебраические многообразия~— являются центральными объектами алгебраической геометрии.
Особенный интерес представляют случаи, когда на~алгебраическом многообразии имеется дополнительная алгебраическая структура,
например, группы. Такие объекты называются алгебраическими группами.

\textbf{Алгебраической подгруппой} группы $G$ называется подгруппа, также являющася
алгебраическим многообразием, то~есть задаваемая системой полиномиальных уравнений.

Наиболее известным примером являются кубики~— плоские алгебраические кривые, задаваемые уравнением третьего порядка.
Однако групповой закон на~эллиптических кривых устроен крайне нетривиально. Кроме того, часто важно сохранить структуру из~объемлющего пространства,
к~примеру, из~алгебраического тора $(K^\times)^n$ для~поля $K$.

Естественным примером такой алгебраической группы служит множество решений системы биномиальных уравнений вида
$$
    z_1^{\alpha_1} z_2^{\alpha_2} \ldots z_n^{\alpha_n} = z_1^{\beta_1} z_2^{\beta_2} \ldots z_n^{\beta_n}.
$$

Как показывает следующая теорема \cite{Schm94}, такие группы, на~самом деле, исчерпывают все алгебраические многообразия,
глобально наследующие групповую структуру тора $(K^\times)$.

Пусть $G$~— некоторая группа. Целочисленная степень элемента группы $g \in G$ определяется стандартным образом:
$$
  g^k =
  \begin{cases}
    g \cdot g^{k - 1}, & k > 0; \\
    1, & k = 0; \\
    g^{-1} \cdot g^{k + 1}, & k < 0.
  \end{cases}
$$

Для~векторов $\alpha = (\alpha_1, \ldots, \alpha_n) \in \mathbb{Z}^n$ и~$z = (z_1, \ldots, z_n) \in G^n$ (где $G$~— группа)
обозначим $z^\alpha = z_1^{\alpha_1} \cdot z_2^{\alpha_2} \cdot \ldots \cdot z_n^{\alpha_n}$.

\begin{theorem*}[Шмидт]
    Пусть $K$~— поле. Всякая алгебраическая подгруппа $H$ группы $(K^{\times})^n$ задаётся
    системой некоторого числа $N$ биномиальных уравнений, то~есть существуют $N$ таких показателей $\alpha_i, \beta_i \in \mathbb{Z}^n$, что
    $$
        H = \{ z \in (K^{\times})^n\ |\ \forall 1 \leq i \leq N{:}\ z^{\alpha_i} = z^{\beta_i} \}.
    $$
\end{theorem*}

Для~её доказательства нам понадобится вспомогательное утверждение \cite{Art48}.

\section{Теорема Артина}

Обозначим как~$\Hom(G, H)$ множество гомоморфизмов между группами $G$ и~$H$.

Пусть $G$~— группа, $K$~— поле, а~$K^{\times}$~— его мультипликативная группа.
В~таком случае произвольный гомоморфизм $f \in \Hom(G, K^{\times})$ называют \textbf{характером.}
Говорят, что характеры $f_1, f_2, \ldots, f_n$ \textbf{линейно независимы,}
если
$$
    \forall \alpha_1, \alpha_2, \ldots, \alpha_n \in K{:}\ \alpha_1 f_1 + \alpha_2 f_2 + \ldots + \alpha_n f_n = 0 \Rightarrow \alpha_1 = \alpha_2 = \ldots = \alpha_n = 0.
$$

\begin{theorem*}[Артин]
    Любые $n$ попарно различных характеров линейно независимы.
\end{theorem*}

\begin{proof}

Докажем индукцией по~числу характеров $n$.

Возьмём произвольный характер $f$. Поскольку он является гомоморфизмом,
то~$f(1_G) = 1$, где $1_G$~— единица в~группе $G$. Но~тогда если $\alpha f = 0$,
то~и~$\alpha = \alpha \cdot 1 = \alpha \cdot f(1_G) = 0$, что доказывает базу индукции.

Пусть теперь утверждение теоремы верно для~любых $n$ различных характеров. Докажем его для~$(n + 1)$ характера.
Пусть $\alpha_1 f_1 + \alpha_2 f_2 + \ldots + \alpha_n f_n + \alpha_{n + 1} f_{n + 1} = 0$.
Зафиксируем произвольный $y \in G$. Тогда для~любого $x \in G$ имеем:
\begin{equation}\label{Artin:first}
    \alpha_1 f_1(yx) + \alpha_2 f_2(yx) + \ldots + \alpha_n f_n(yx) + \alpha_{n + 1} f_{n + 1}(yx) = 0.
    %\sum_{i = 1}^{n + 1} \alpha_i f_i(yx) = 0.
\end{equation}
Поскольку все $f_i$~— гомоморфизмы, $f_i(yx) = f_i(y) f_i(x)$, а~потому:
$$
    \alpha_1 f_1(y) f_1(x) + \ldots + \alpha_n f_n(y) f_n(x) + \alpha_{n + 1} f_{n + 1}(y) f_{n + 1}(x) = 0.
$$
С~другой стороны, для~$x \in G$ также верно, что:
$$
    \alpha_1 f_1(x) + \alpha_2 f_2(x) + \ldots + \alpha_n f_n(x) + \alpha_{n + 1} f_{n + 1}(x) = 0.
$$
Умножим это равенство на~$f_{n + 1}(y)$:
\begin{equation}\label{Artin:second}
    \alpha_1 f_{n + 1}(y) f_1(x) + \ldots + \alpha_n f_{n + 1}(y) f_n(x) + \alpha_{n + 1} f_{n + 1}(y) f_{n + 1}(x) = 0.
\end{equation}
Вычитая \eqref{Artin:first} из~\eqref{Artin:second}, получаем:
$$
    \alpha_1 (f_{n + 1}(y) - f_1(y)) f_1(x) + \ldots + \alpha_n (f_{n + 1}(y) - f_n(y)) f_n(x) = 0.
$$
Тогда, в~силу произвольности $x$, получаем линейную комбинацию $n$ характеров:
$$
    \alpha_1 (f_{n + 1}(y) - f_1(y)) f_1 + \ldots + \alpha_n (f_{n + 1}(y) - f_n(y)) f_n = 0.
$$

Но~в~таком случае, по~индуктивной гипотезе, $\alpha_i (f_{n + 1}(y) - f_i(y)) = 0$.
Теперь, выбрав для~каждого $i = 1, \ldots, n$ такое $y$, что $f_{n + 1}(y) \neq f_i(y)$
(это возможно, потому~что, по~условию теоремы, все $f_i$ попарно различны), получим,
что $\alpha_1 = \alpha_2 = \ldots = \alpha_n = 0$.

Таким образом, с~учётом вышесказанного, $\alpha_{n + 1} f_{n + 1} = \alpha_1 f_1 + \alpha_2 f_2 + \ldots + \alpha_n f_n + \alpha_{n + 1} f_{n + 1} = 0$.
Согласно базе индукции, $\alpha_{n + 1} = 0$.
\end{proof}

\section{Доказательство теоремы Шмидта}

\textit{Многочленом Лорана} над~кольцом $K$ называется функция $(K^\times)^k \rightarrow K$ вида
$$
    z \mapsto \sum_{i \in I} a_i z^{\alpha_i},
$$
где $a_i \in K$ и~$I \subseteq \mathbb{Z}^k$~— конечный набор индексов.

\begin{proof}
  Пусть $I \subseteq \mathbb{Z}^n$~— множество индексов,
  а~$P_j(z) = \sum_{i \in I} a_{ji} z^i$~— многочлены ($k$ штук), задающие подгруппу:
  $$
      H = \{ z \in (K^{\times})^n\ |\ \forall 1 \leq j \leq k{:}\ P_j(z) = 0 \}.
  $$

  Поскольку $z_1^{i} z_2^{i} = (z_1 z_2)^i$, отображение $z \mapsto z^i$ определяет
  характер $\chi_i \in \Hom(H, K^{\times})$.

  Рассмотрим на~множестве $I$ отношение эквивалентности $i \sim j \Leftrightarrow \chi_i = \chi_j$.
  Оно разбивает $I$ на~$m$ классов $I_1$, $I_2$, …, $I_m$. В~таком случае, для~всех $i_1, i_2 \in I_j$ верно, что $\chi_{i_1} = \chi_{i_2}$.
  Обозначим этот характер, соответствующий каждому $I_k$, как $\chi_k = \chi_{i_1} = \chi_{i_2}$.

  Собрав подобные слагаемые при~каждом $\chi_k$ в~многочленах $P_j$, получим:
  $$
      P_j = \sum_{i \in I} a_{ji} \chi_i = \sum_{k = 1}^{m} \left( \sum_{i \in I_k} a_{ji} \right) \chi_k.
  $$

  Но~$P_j$ равны нулю на~всём $H$, поэтому:
  $$
  \sum_{k = 1}^{m} \left( \sum_{i \in I_k} a_{ji} \right) \chi_k = 0.
  $$

  Все $\chi_k$ попарно различны по~их заданию, поэтому к~ним применима теорема Артина. Из~неё заключаем, что:
  $$
  \sum_{i \in I_k} a_{ji} = 0.
  $$

  Наконец, пусть $N_k$~— мощность $I_k$, $N = \sum_{k = 1}^m (N_k - 1)$ и~$I_k = \{i_{k, 1},\allowbreak i_{k, 2},\allowbreak \ldots,\allowbreak i_{k, N_k}\}$.
  $I_k$ были взяты так, что характеры $\chi_{i_{k, i}}$ и~$\chi_{i_{k, j}}$ совпадают на~$H$ для~фиксированного $k$ и~любых $i$ и~$j$.
  Другими словами, это означает, что всякий элемент $z \in H$ удовлетворяет для~всех $1 \leq i \leq N_k$ и~$1 \leq j \leq N_k$
  уравнениям $z^{i_{k, i}} = z^{i_{k, j}}$.

  Поскольку равенство рефлексивно, симметрично и~транзитивно, система всех уравнений $z^{i_{k, i}} = z^{i_{k, j}}$ равносильна
  системе $N$ уравнений, составленной из~подряд идущих индексов:
  $$
      z^{i_{1, 1}} = z^{i_{1, 2}}, z^{i_{1, 2}} = z^{i_{1, 3}}, \ldots, z^{i_{1, N_1 - 1}} = z^{i_{1, N_1}}, z^{i_{2, 1}} = z^{i_{2, 2}}, \ldots, z^{i_{m, N_m - 1}} = z^{i_{m, N_m}}.
  $$
  Множество решений этой системы обозначим как $A$.

  Покажем, что $A$ совпадает с~$H$. Действительно, пусть $z \in H$. Тогда, как уже отмечалось выше,
  по построению $I_k$ выполнено $z^{i_{k, j}} = \chi_k(z) = z^{i_{k, j + 1}}$; то~есть $z \in A$, и~$H \subseteq A$.

  Наоборот, пусть $z \in A$. Тогда, по~заданию $A$, $z^{i_{k, j_1}} = z^{i_{k, j_1 + 1}} = \ldots = z^{i_{k, j_2 - 1}} = z^{i_{k, j_2}}$
  для~любых $1 \leq j_1 \leq j_2 \leq N_k$. Выберем в~каждом $I_k$ по~представителю $i_k \in I_k$ и~соберём подобные слагаемые:
  $$
      P_j(z) = \sum_{k = 1}^{m} \left( \sum_{i \in I_k} a_{ji} \right) z^{i_k} = \sum_{k = 1}^{m} 0 \cdot z^{i_k} = 0,
  $$
  что означает $z \in H$, то~есть $A \subseteq H$.

  Таким образом, $H = A$, поэтому в~качестве $\alpha_i$ и~$\beta_i$ достаточно взять подряд идущие индексы во~всех $I_k$.
\end{proof}

\section{Инъективность мономиальных параметризаций}

Для~абелевой группы $G$ и~векторов $\alpha_1, \ldots, \alpha_n \in \Z^k$ рассмотрим отображение $\phi_\alpha(t) = (t^{\alpha_1}, t^{\alpha_2}, \ldots, t^{\alpha_n})$
из~$G^k$ в~$G^n$, где $\alpha$~— матрица со~строками $\alpha_i$. Поскольку $(t_1 \cdot t_2)^{\alpha_i} = t_1^{\alpha_i} \cdot t_2^{\alpha_i}$,
$\phi_\alpha$~— гомоморфизм.

Далее увидим, что если $K$~— поле, то~отображение $\phi_\alpha$ для~группы $G = K^\times$ задаёт \textit{параметризацию} некоторой алгебраической группы.

\begin{statement*}
    $\phi_{E} = 1_{G^n}$, где $E$~— единичная матрица $n \times n$, $1_X$~— тождественная функция на~множестве $X$.
\end{statement*}

\begin{statement*}
    $
        \forall \alpha \in \Z^{k \times m}, \beta \in \Z^{m \times n}{:}\ \phi_\alpha \circ \phi_\beta = \phi_{\alpha \beta}
    $.
\end{statement*}

\begin{proof}
    Действительно, рассмотрим $t \in G^n$. Тогда $i$-я компонента вектора $\phi_\alpha(\phi_\beta(t))$
    равна
    $$ \phi_\beta^{\alpha_i}(t) = (t^{\beta_1})^{\alpha_i^1} \cdot \ldots \cdot (t^{\beta_m})^{\alpha_i^m}
                                = t_1^{\beta_1^1 \alpha_i^1 + \ldots + \beta_m^1 \alpha_i^m} \cdot
                                  \ldots \cdot
                                  t_n^{\beta_1^n \alpha_i^1 + \ldots + \beta_m^n \alpha_i^m}
                                = t^{(\alpha \beta)_i}.
    $$
\end{proof}

Вместе эти два утверждения составляют условие на~функториальность. Более точно, рассмотрим категорию
$\mathrm{Matr}(R)$ матриц над~кольцом $R$, объектами которой являются натуральные числа, а~стрелками
между числами $m$ и~$n$~— матрицы $R^{n \times m}$, и~категорию $\mathrm{Grp}$ малых групп,
объектами которой являются малые группы, а~стрелками~— гомоморфизмы между ними.
Тогда определён функтор $\phi : \mathrm{Matr}(\Z) \rightarrow \mathrm{Grp}$:
$$
    \phi = \begin{cases}
        k \mapsto G^k; \\
        \alpha \mapsto \phi_\alpha.
    \end{cases}
$$

Как показывает лемма \ref{lemma:faithful}, в~полях нулевой характеристики этот функтор~— строгий.

\begin{lemma}
\label{lemma:infiniteMulOrder}
    Пусть $K$~— поле, причём $\mathrm{char}(K) = 0$.
    Тогда $K^\times$ содержит элемент бесконечного порядка.
\end{lemma}

\begin{proof}
    Действительно, рассмотрим $2 = 1 + 1 \in K^\times$. Тогда для~всякого $n > 0$:
    $$
        2^n - 1 = (1 + 1)^n - 1 = \sum_{k = 0}^n C^n_k 1^k 1^{n - k} - 1
                                = \sum_{k = 0}^n C^n_k \cdot 1 - 1
                                = \left(\sum_{k = 1}^n C^n_k \right) \cdot 1.
    $$

    Справа имеем сумму $\sum_{k = 1}^n C^n_k > 0$ единиц. Поскольку $\mathrm{char}(K) = 0$,
    она не равна нулю, то~есть $2^n - 1 \neq 0$.
\end{proof}

Как $e_i$ обозначим \textit{естественный базис} над~$\Z^k$, то~есть такой вектор, у~которого на~$i$-м месте единица, а~на~остальных~— нули.
$j$-ю компоненту вектора $\alpha_i$ обозначим как $\alpha_i^j$, а~вектор, состоящий из~$j$-х компонент,
как $\alpha^j = (\alpha_1^j, \alpha_2^j, \ldots, \alpha_n^j)$.

\begin{lemma}
\label{lemma:faithful}
    $
        \mathrm{char}(K) = 0 \Rightarrow \forall \alpha, \beta \in \Z^{n \times k}{:} \ \phi_\alpha = \phi_\beta \Leftrightarrow \alpha = \beta
    $.
\end{lemma}

\begin{proof}
    Импликация справа налево очевидна.

    Наоборот, пусть $\phi_\alpha = \phi_\beta$. По~лемме \ref{lemma:infiniteMulOrder} зафиксируем
    элемент $z \in K^\times$ бесконечного мультипликативного порядка.
    Тогда для~всякого $1 \leq i \leq n$ верно, что
    $$
        (z^{\alpha_1^i}, \ldots, z^{\alpha_k^i}) = \phi_\alpha(z \cdot e_i) = \phi_\beta(z \cdot e_i) = (z^{\beta_1^i}, \ldots, z^{\beta_k^i}).
    $$

    Таким образом, $z^{\alpha_j^i} = z^{\beta_j^i}$, то~есть $z^{\alpha_j^i - \beta_j^i} = 1$.
    $z$ имеет бесконечный мультипликативный порядок, поэтому $\alpha_j^i - \beta_j^i = 0$.
\end{proof}

Нетрудно получить необходимое условие инъективности отображения $\phi_\alpha$.
Для~этого заметим, что очевидно следующее утверждение.

\begin{statement*}
    $\Span_\Z(\alpha_1, \ldots, \alpha_n) = \Z^k \Leftrightarrow \forall i{:}\ e_i \in \Span_\Z(\alpha_1, \ldots, \alpha_n)$.
\end{statement*}

\begin{lemma}
    Если $\alpha_i \in \Z^k$ порождают всю решётку, то~$\phi_\alpha$ инъективно.
\end{lemma}

\begin{proof}
    Достаточно показать, что $\ker(\phi_\alpha) = \{1\}$, поскольку $\phi_\alpha$~— гомоморфизм.
    Возьмём $t \in G^k$ такой, что $\phi_\alpha(t) = 1$, и~докажем, что $t = 1$.

    Действительно, поскольку $\alpha_i$ порождают всю решётку, каждый $e_j$ выражается как линейная комбинация векторов $\alpha_i$:
    $$
        e_j = b^j_1 \alpha_1 + \ldots + b^j_n \alpha_n.
    $$

    $\phi_\alpha(t) = 1$ означает, что $t^{\alpha_i} = 1$ для~всех $1 \leq i \leq n$.
    Зафиксируем $1 \leq j \leq k$ и~возведём это равенство в~степень $b^j_i$: $t^{b^j_i \alpha_i} = (t^{\alpha_i})^{b^j_i} = 1^{b^j_i} = 1$.
    Наконец, перемножим полученные равенства:
    $$
        t_j = t^{e_j} = t^{b^j_1 \alpha_1 + \ldots + b^j_n \alpha_n} = t^{b^j_1 \alpha_1} \ldots t^{b^j_n \alpha_n} = 1 \cdot \ldots \cdot 1 = 1,
    $$
    что и~требовалось.
\end{proof}

\noindent\textbf{Пример.} $(t_1, t_2) \mapsto (t_1^2 t_2^3, t_1 t_2^2)$ инъективно над~$\complex$,
поскольку $(1, 0) = 2 \cdot (2, 3) + (-3) \cdot (1, 2)$ и~$(0, 1) = -1 \cdot (2, 3) + 2 \cdot (1, 2)$.
\medskip

Теперь покажем, что полученное в~лемме небходимое условие является также и~достаточным для~группы $G = \torus$.

Будем пользоваться существованием для~целочисленных матриц \textit{нормальной формы Смита} \cite{Smth60}.
Пусть $\diag^{m \times n}_r(x_1, \ldots, x_r)$~— диагональная матрица $\diag(x_1, \ldots, x_r)$,
дополненная (или обрезанная) справа снизу нулями до~матрицы размера $m \times n$.

\begin{theorem*}[о существовании нормальной формы Смита]
    Для~всякой матрицы $\alpha \in \Z^{m \times n}$ существуют
    пара матриц $\beta_1 \in \GL^m(\Z)$ и~$\beta_2 \in \GL^n(\Z)$,
    а~также набор чисел $\varepsilon_1, \ldots, \varepsilon_r \in \Z \setminus \{0\}$,
    такие, что:
    $$
        \beta_1 \alpha \beta_2 = \diag^{m \times n}_r(\varepsilon_1, \ldots, \varepsilon_r),
    $$
    причём $\varepsilon_1 \divides \varepsilon_2 \divides \ldots \divides \varepsilon_r$ (так называемые инвариантные факторы) и~$r = \rank(\alpha)$.
\end{theorem*}

Поскольку числа $\varepsilon_1$, $\ldots$, $\varepsilon_r$ выбираются единственным образом с~точностью
до~обратимого элемента, то~есть в~случае $\Z$ с~точностью до~$\pm 1$, можем всегда выбрать их положительными.
Для~них матрицу $\diag^{m \times n}_r(\varepsilon_1, \ldots, \varepsilon_r)$ обозначим как $\SNF(\alpha)$.

Помимо этого, с~учётом функториальности $\phi$, ясно, что $\phi_\beta$ биективно, если $\beta \in \GL^n(\Z)$.
Действительно, $\phi_\beta \circ \phi_{\beta^{-1}} = \phi_{\beta \beta^{-1}} = \phi_E = 1_{(K^\times)^n}$
и~$\phi_{\beta^{-1}} \circ \phi_\beta = \phi_{\beta^{-1} \beta} = \phi_E = 1_{(K^\times)^n}$.
Поэтому имеет место следующая лемма.

\begin{lemma}
\label{lemma:injectivityOutOfSNF}
    $\phi_\alpha : (\complex^\times)^k \rightarrow (\complex^\times)^n$ инъективна тогда и~только тогда, когда
    $$
        \SNF(\alpha) = \diag^{n \times k}_k(1, 1, \ldots, 1).
    $$
\end{lemma}

\begin{proof}
    Пользуясь теоремой, возьмём матрицы $\beta_1 \in \GL^n(\Z)$ и~$\beta_2 \in \GL^k(\Z)$
    такие, что $\beta_1 \alpha \beta_2 = \SNF(\alpha)$.
    В~таком случае, $\alpha = \beta_1^{-1} \SNF(\alpha) \beta_2^{-1}$, и~$\phi_\alpha = \phi_{\beta_1^{-1} \SNF(\alpha) \beta_2^{-1}}
                                                                                      = \phi_{\beta_1^{-1}} \circ \phi_{\SNF(\alpha)} \circ \phi_{\beta_2^{-1}}$.

    Поскольку, по~замечанию выше, $\phi_{\beta_1^{-1}}$ и~$\phi_{\beta_2^{-1}}$ биективны, то~$\phi_\alpha$
    инъективно тогда и~только тогда, когда инъективно $\phi_{\mathrm{SNF(\alpha)}}$.

    $\phi_{\mathrm{SNF(\alpha)}}(t_1, \ldots, t_k) = (t_1^{\varepsilon_1}, \ldots, t_r^{\varepsilon_r}, 1, \ldots, 1)$,
    но~$t \mapsto t^\varepsilon$ инъективно над~$\complex^\times$ только в~случае $\varepsilon = \pm 1$
    (иначе $t^\varepsilon = t^\varepsilon \cdot 1 = t^\varepsilon \cdot u^\varepsilon = (tu)^\varepsilon$,
     где~$u \neq 1$~— нетривиальный корень из~единицы степени $\varepsilon$).
    В~силу выбора знаков, $\phi_{\mathrm{SNF(\alpha)}}$ инъективна лишь когда $\varepsilon_1 = \ldots = \varepsilon_r = 1$
    и~$r = k \leq n$, что и~требовалось.
\end{proof}

Сформулируем ещё одну вспомогательную теорему \cite{TsikhSad14}.
\textit{$\Z$-линейную оболочку} векторов $\alpha_1, \ldots, \alpha_n$ обозначим как
$$
    \Span_\Z(\alpha_1, \ldots, \alpha_n) = \{ k_1 \alpha_1 + \ldots + k_n \alpha_n \ | \ k_1, \ldots, k_n \in \Z \}.
$$

Если $\Span_\Z(\alpha_1, \ldots, \alpha_n) = \Z^k$, то~говорим, что $\alpha_1, \ldots, \alpha_n$ \textit{порождают всю решётку} $\Z^k$.

\begin{theorem}
\label{theorem:TsikhSadykov}
    Пусть $\alpha \in \Z^{n \times k}$. Следующие утверждения равносильны:
    \begin{enumerate}
        \item Строки матрицы $\alpha$ порождают всю решётку $\Z^k$.
        \item Миноры максимальной размерности матрицы $\alpha$ взаимно просты.
        \item $\SNF(\alpha) = \diag^{n \times k}_k(1, \ldots, 1)$.
    \end{enumerate}
\end{theorem}

Таким образом, из~леммы \ref{lemma:injectivityOutOfSNF} и~теоремы \ref{theorem:TsikhSadykov}
получаем необходимое и~достаточное условие инъективности $\phi_\alpha$.

\begin{theorem*}
    $\phi_\alpha$ инъективно над~полем $K = \complex$ тогда и~только тогда, когда~$\alpha_i$ порождают всю решётку $\Z^k$.
\end{theorem*}

\section{Мономиальная параметризуемость алгебраических групп}

Как и~в~случае теории кривых и~поверхностей, говорим, что подгруппа алгебраического тора \textit{параметризуется,}
если она представима как образ некоторого отображения. Аналогично, подгруппа \textit{параметризуется мономиально,}
если она представима как образ отображения $\phi_\alpha$ для~некоторой матрицы $\alpha$.

Поскольку $\phi_\alpha$~— гомоморфизм, его полный образ $\image(\phi_\alpha)$ является подгруппой в~$(K^\times)^n$.
Изучим два вопроса: является~ли $\image(\phi_\alpha)$ алгебраической подгруппой и~всякая~ли алгебраическая подгруппа задаётся некоторым $\phi_\alpha$.

Для~этого, прежде всего, заметим, что система биномиальных уравнений $z^{\beta'_i} = z^{\beta''^i}$ эквивалентна
системе $z^{\beta'_i - \beta''_i} = 1$, то~есть в~точности ядру оператора $\phi_{\beta'_i - \beta''_i}$;
поэтому первый вопрос эквивалентен тому, можно~ли данный гомоморфизм $\phi_\alpha$ достроить
до~точной последовательности $(K^\times)^k \xrightarrow[]{\phi_\alpha} (K^\times)^n \xrightarrow[]{\phi_\beta} (K^\times)^m$.

\begin{lemma}
\label{lemma:leftInvExactSequence}
    Пусть $G$~— группа, $H$~— абелева группа, $f \in \Hom(G, H)$, $g \in \Hom(H, G)$ и~$g \circ f = 1_G$.
    Тогда последовательность $G \xrightarrow[]{f} H \xrightarrow[]{f \circ g - 1_H} H$ точна.
\end{lemma}

\begin{proof}
    Действительно, $(f \circ g - 1_H) \circ f = f \circ g \circ f - f = f - f = 0$, то~есть $\image(f) \subseteq \ker(f \circ g - 1_H)$. 
    Обратно, пусть $h \in \ker(f \circ g - 1_H)$. Тогда $f(g(h)) - h = 0 \Leftrightarrow h = f(g(h))$, то~есть $h \in \image(f)$,
    и~$\ker(f \circ g - 1_H) \subseteq \image(f)$.
\end{proof}

\begin{lemma}
    $\phi_\alpha$, где $\alpha \in \Z^{n \times k}$, представимо как некоторый образ $\image(\phi_\beta)$,
    если отображения $g \mapsto g^{\varepsilon_i}$ сюръективны в~$G$ для~всех $\varepsilon_i$ из~$\SNF(\alpha)$.
\end{lemma}

\begin{proof}
    Снова представим $\alpha$ в~виде $\alpha = \beta_1^{-1} \varepsilon \beta_2^{-1}$, где
    $$
        \varepsilon = \SNF(\alpha) = \diag^{n \times k}_r(\varepsilon_1, \ldots, \varepsilon_r).
    $$

    Рассмотрим $\delta = \diag^{n \times r}_r(1, \ldots, 1)$ и~$\alpha' = \beta^{-1} \delta$.
    $\phi_{\alpha'}$ инъективна как композиция инъективных функций. С~другой стороны, очевидно, что:
    $$
        \image(\phi_\alpha) = \phi_{\beta_1^{-1}} (\phi_\varepsilon (\phi_{\beta_2^{-1}} (G^k)))
                            = \phi_{\beta_1^{-1}} (\phi_\varepsilon (G^k))
                            = \phi_{\beta_1^{-1}} (\phi_{\delta} (G^r))
                            = \image(\phi_{\alpha'}).
    $$

    Нетрудно видеть, что $\alpha'$ имеет левой обратной матрицу $\beta' = \delta^{\top} \beta$,
    а~потому гомоморфизм $\phi_{\alpha'}$ имеет левым обратным гомоморфизм $\phi_{\beta'}$.
    Поскольку $(\phi_{\alpha'} \circ \phi_{\beta'}) \phi_{-E} = \phi_{\alpha' \beta' - E}$, остаётся лишь применить лемму \ref{lemma:leftInvExactSequence}.
\end{proof}

\begin{theorem}
    Всякая параметризация $\phi_\alpha$, где $\alpha \in \Z^{n \times k}$, задаёт алгебраическую подгруппу $(K^\times)^n$,
    если поле $K$ алгебраически замкнуто.
\end{theorem}

\noindent\textbf{Пример.} Над~алгебраически незамкнутым полем теорема не~выполняется: образ $t \mapsto t^2$ над~$\R$
представляет собой положительный луч $\R_{>0}$. Как известно, множество корней вещественного мночлена либо конечно,
либо совпадает со~всем $\R$, поэтому указанная параметризация не~определяет алгебраическое множество.
\medskip

Ответ на~второй вопрос несколько сложнее. Например, уравнение $z^n = 1$ задаёт алгебраическую
группу для~любого $n \in \Z$, состоящую из~$n$ точек на~$\complex^\times$; но~если $n \neq 0, \pm 1$,
то~легко видеть, что она не~может быть параметризована никакой $\phi_\alpha$.

Действительно, $\phi_\alpha$~— голоморфное отображение, $(\complex^\times)^k$~— открытое множество,
поэтому всякая проекция полного образа $\image(\phi_\alpha) = \phi_\alpha((\complex^\times)^k)$ либо, в~силу
открытости непостоянных голоморфных отображений, открыта, либо состоит из~одной точки; но~$z^n = 1$
открыто, только если $n = 0$, и~состоит из~одной точки, только если $n = \pm 1$.

Более того, из~этих рассуждений видно, что~по~всякой системе биномиальных уравнений можно построить новую систему,
которая гарантированно не~параметризуется мономиально, возведя, например, произвольное уравнение в~степень $n > 1$.

Чтобы понять, при~каких условиях существует мономиальная параметризация, сначала заметим, что группа $\image(\phi_\alpha)$
изоморфна группе $(\torus)^r$. Действительно, в~доказательстве теоремы мы видели, что для~всякого $\phi_\alpha$
можно построить инъективный $\phi_{\alpha'} : (\torus)^r \rightarrow (\torus)^n$ такой, что $\image(\phi_\alpha) = \image(\phi_{\alpha'})$;
но~тогда $\phi_{\alpha'}$ и~задаёт искомый изоморфизм. Таким образом, верно следующее утверждение.

\begin{statement*}
    Для~любой матрицы $\alpha \in \Z^{n \times k}$ группа $\image(\phi_\alpha)$ изоморфна алгебраическому тору размерности не~более чем $k$.
\end{statement*}

Кроме того, поскольку отображение $\phi_{\alpha'}$ полиномиально, оно, в~частности, непрерывно в~естественной топологии на~$K = \complex$.
Множество $(\complex^\times)^r$ связно, а, как известно, непрерывный образ связного множества связен, в~силу чего верно ещё одно утверждение.

\begin{statement*}
    $\image(\phi_\alpha) \subseteq (\complex^\times)^n$ связно.
\end{statement*}

Теперь отметим, что группа $z^n = 1$ связна в~точности тогда, когда $n = 0, \pm 1$.
Это соображение указывает на~то, что критерием существования мономиальной параметризации
для~алгебраической подгруппы тора является связность.

Группу корней степени $n$ из~единицы, задаваемую уравнением $z^n = 1$ в~поле $K$, обозначим как $\omega_K(n) \subseteq K^\times$.
Для~вектора $x \in \Z^k$ также обозначим $\omega_K(x) = \omega_K(x_1) \times \ldots \times \omega_K(x_k) \subseteq (K^\times)^k$.
Для~краткости будем писать $\omega(x) = \omega_\complex(x)$. Известно, что группа $\omega(x)$ изоморфна
группе $\Z / x_1 \Z \times \ldots \times \Z / x_k \Z$.

Чтобы доказать описанный выше критерий, рассмотрим число $\Pi(\alpha) = |\omega_K(\varepsilon)|$, где $|G|$~— порядок группы $G$.
$|G \times H| = |G| |H|$, поэтому $\Pi(\alpha) = |\omega_K(\varepsilon_1)| \cdot \ldots \cdot |\omega_K(\varepsilon_r)|$.
Так как $\omega(n)$ изоморфно $\Z / n \Z$, в~случае поля $K = \complex$ выполняется
равенство $\Pi(\alpha) = \varepsilon_1 \cdot \ldots \cdot \varepsilon_r$.

\begin{lemma}
\label{lemma:exactOutOfPi}
    Если $\Pi(\beta) = 1$, то~существует такое $\alpha$, что $\ker(\phi_\beta) = \image(\phi_\alpha)$.
\end{lemma}

\begin{proof}
    Пусть $H = \ker(\phi_\beta)$. Без~потери общности считаем, что строки матрицы $\beta$ линейно независимы над~$\Z$
    (ясно, что строки, выражающиеся как линейная комбинация остальных строк, можно убрать из~матрицы $\beta$ без~изменения ядра).

    Запишем для~$\beta$ нормальную форму Смита: $\beta = \beta_1^{-1} \varepsilon \beta_2^{-1}$.
    Поскольку, по~замечанию выше, $\beta$ имеет полный ранг, матрица $\varepsilon$ не~имеет нулевых строк.
    Кроме того, $\Pi(\beta) = 1$, поэтому $|\omega_K(\varepsilon_i)| = 1$, то~есть уравнения $z^{\varepsilon_i} = 1$
    имеют решением только $z = 1$.

    Ядро $\varepsilon$ задаётся векторами вида $(0, \ldots, 0, t_{k + 1}, \ldots, t_{n})$.
    Рассмотрим матрицу $\delta \in \Z^{n \times (n - k)}$, соответствующую линейному оператору
    $$
        (t_1, \ldots, t_{n - k}) \mapsto (0, \ldots, 0, t_1, \ldots, t_{n - k}).
    $$

    Пусть также $\alpha = \beta_2 \delta$. Докажем, что $\image(\phi_\alpha) = \ker(\phi_\beta)$.

    Действительно, по~заданию $\delta$, $\varepsilon \delta = 0$, поэтому 
    $\phi_{\beta} \circ \phi_{\alpha} = \phi_{\beta_1^{-1} \varepsilon \beta_2^{-1} \beta_2 \delta} = \phi_{\beta_1^{-1} \varepsilon \delta} = \phi_{0} = 1$,
    то~есть $\image(\phi_{\alpha}) \subseteq \ker(\phi_{\beta})$.

    Обратно, пусть $z \in \ker(\phi_\beta)$. Тогда $\phi_{\beta_1^{-1}} (\phi_{\varepsilon \beta_2^{-1}}(z)) = \phi_\beta(z) = 1$,
    откуда $\phi_{\varepsilon}(\phi_{\beta_2^{-1}}(z)) = \phi_{\varepsilon \beta_2^{-1}}(z) = \phi_{\beta_1}(1) = 1$.
    Из~вида матрицы $\varepsilon$ получаем, что $\phi_{\beta_2^{-1}}(z) = (1, \ldots, 1, t_{k + 1}, \ldots, t_n) = \phi_\delta(t_{k + 1}, \ldots, t_n)$;
    то~есть для~некоторых $t_j$ выполнено $z = \phi_{\beta_2}(\phi_\delta(t_{k + 1}, \ldots, t_n)) = \phi_\alpha(t_{k + 1}, \ldots, t_n)$.
    Таким образом, $z \in \image(\phi_\alpha)$, и~$\ker(\phi_\beta) \subseteq \image(\phi_\alpha)$.
\end{proof}

\begin{theorem}
\label{theorem:connectedComponentsNumbers}
    Число связных компонент $\ker(\phi_\beta) \subseteq (\torus)^n$ равно $\Pi(\beta)$.
\end{theorem}

\begin{proof}
    Снова рассмотрим нормальную форму для~$\beta$: $\beta$ = $\beta_1^{-1} \varepsilon \beta_2^{-1}$.
    $\phi_{\beta_1^{-1}}$~— изоморфизм, поэтому $\ker(\phi_{\beta}) = \ker(\phi_{\varepsilon \beta_2^{-1}})$.

    Строки матрицы $\beta_2^{-1}$ обозначим как $b_i$.
    Для~вектора $u \in \omega(\varepsilon)$ рассмотрим множество $H_u = \{z \in (\torus)^n \ | \ \forall 1 \leq i \leq r{:}\ z^{b_i} = u_i\}$.
    Условие $\phi_\varepsilon(\phi_{\beta_2^{-1}}(z)) = 1$, очевидно, эквивалентно условию $\exists u \in \omega(\varepsilon){:}\ z \in H_u$.
    Ясно, что множества $H_u$ дизъюнктны, поэтому $\ker(\phi_{\beta})$ распадается в~дизъюнктное объединение:
    $$
        \ker(\phi_{\beta}) = \bigsqcup_{u \in \omega(\varepsilon)} H_u.
    $$

    Векторов $u \in \omega(\varepsilon)$ ровно $\Pi(\beta)$ штук, поэтому достаточно показать, что каждая компонента $H_u$ связна.

    Поскольку, по~определению, $H_1 = \ker(\phi_{\delta \beta_2^{-1}})$, где $\delta = \diag^{k \times n}_r(1, \ldots, 1)$,
    и~$\Pi(\delta \beta_2^{-1}) = 1$, компонента $H_1$, по~замечанию, связна.

    Зафиксировав $u$, рассмотрим матрицу $\tau = \diag(1 / u_1, \ldots, 1 / u_r, 1, \ldots, 1)$ и~отображение $\psi = \phi_{\beta_2} \circ \phi_\tau \circ \phi_{\beta_2^{-1}}$.
    $\psi$~— непрерывная биекция, причём $\psi^{-1} = \phi_{\beta_2} \circ \phi_{\tau^{-1}} \circ \phi_{\beta_2^{-1}}$.
    По~определению $\psi$, $\phi_{\beta_2^{-1}}(\psi(z)) = \phi_{\tau \beta_2^{-1}}(z)$.

    Если теперь $z \in H_u$, то~ $\psi(z)^{b_i} = z^{b_i} / u_i = u_i / u_i = 1$. Обратно, если $z \in H_1$,
    то~$\psi^{-1}(z)^{b_i} = u_i z^{b_i} = u_i$. Таким образом, $\psi(H_u) = H_1$,
    то~есть $H_u$ гомеоморфно $H_1$, но~$H_1$ связно.
\end{proof}

Как мы видели ранее, всякая алгебраическая подгруппа $H$ тора $(K^\times)^n$ представляется как $\ker(\phi_\beta)$
для~некоторого $\beta$. Заметим, что $\Pi(\beta)$ не~зависит от~выбора $\beta$ для~группы $H$.

\begin{theorem}
    Если $\mathrm{char}(K) = 0$ и~$\ker(\phi_{\beta}) = \ker(\phi_{\beta'})$, то~$\Pi(\beta) = \Pi(\beta')$.
\end{theorem}

\begin{proof}
    Запишем нормальные формы Смита: $\beta = \beta_1^{-1} \varepsilon \beta_2^{-1}$
    и~$\beta' = {\beta'}_1^{-1} \varepsilon' {\beta'}_2^{-1}$.
    Тогда $\ker(\phi_{\varepsilon \beta_2^{-1}}) = \ker(\phi_{\beta}) = \ker(\phi_{\beta'}) = \ker(\phi_{\varepsilon' {\beta'}_2^{-1}})$.

    Пусть $\varepsilon = \diag^{k \times n}_r(\varepsilon_1, \ldots, \varepsilon_r)$
    и~$\varepsilon' = \diag^{k' \times n}_{r'}(\varepsilon'_1, \ldots, \varepsilon'_r)$.
    Обозначим $\delta = \diag^{k \times n}_r(1, \ldots, 1)$ и~$\delta' = \diag^{k' \times n}_{r'}(1, \ldots, 1)$.

    Поскольку $\Pi(\delta \beta_2^{-1}) = \Pi(\delta' {\beta'}_2^{-1}) = 1$, обе подгруппы согласно лемме \ref{lemma:exactOutOfPi}
    параметризуются некоторыми $\phi_\alpha$ и~$\phi_{\alpha'}$ соответственно.

    $\image(\phi_\alpha) = \ker(\phi_{\delta \beta_2^{-1}}) \subseteq \ker(\phi_{\varepsilon \beta_2^{-1}}) = \ker(\phi_{\varepsilon' {\beta'}_2^{-1}})$,
    поэтому $\phi_{\varepsilon' {\beta'}_2^{-1}} \circ \phi_\alpha = 1$.
    В~силу леммы \ref{lemma:faithful}, $\varepsilon' {\beta'}_2^{-1} \alpha = 0$.
    Умножив обе части равенства на~матрицу $\diag(1 / \varepsilon'_1, \ldots, 1 / \varepsilon'_r, 1, \ldots, 1)$, получим, что $\delta' {\beta'}_2^{-1} \alpha = 0$;
    но~это значит, что $$
        \ker(\phi_{\delta \beta_2^{-1}}) = \image(\phi_\alpha) \subseteq \ker(\phi_{\delta' {\beta'}_2^{-1}}).
    $$
    Совершенно аналогично получим обратное включение. В~силу экстенсиональности, $\ker(\phi_{\delta \beta_2^{-1}}) = \ker(\phi_{\delta' {\beta'}_2^{-1}})$.

    Рассмотрим фактор-группу $\ker(\phi_\beta) / \ker(\phi_{\delta \beta_2^{-1}}) = \ker(\phi_{\beta'}) / \ker(\phi_{\delta' {\beta'}_2^{-1}})$,
    называемую также \textit{группой компонент.}
    Отображение $$\phi_{\delta \beta_2^{-1}} : \ker(\phi_\beta) \rightarrow \omega_K(\varepsilon) \times \{ 1 \} \times \ldots \times \{ 1 \}$$
    индуцирует инъективный гомоморфизм из~фактора. Помимо этого, в~теореме \ref{theorem:connectedComponentsNumbers} была построена биекция между компонентами $H_u$ (которая обобщается
    без~изменений на~случай произвольного поля $K$), а~потому все они непусты.
    Это означает, что $\phi_{\delta \beta_2^{-1}}$ сюръективно, а~потому индуцированное отображение тоже сюръективно.
    Аналогично проводятся рассуждения для~группы $\omega_K(\varepsilon')$.
    Таким образом, получаем цепочку изоморфизмов:
    $$
        \omega(\varepsilon) \cong \ker(\phi_\beta) / \ker(\phi_{\delta \beta_2^{-1}}) = \ker(\phi_{\beta'}) / \ker(\phi_{\delta' {\beta'}_2^{-1}}) \cong \omega(\varepsilon').
    $$

    Окончательно, $\Pi(\beta) = |\omega_K(\varepsilon)| = |\omega_K(\varepsilon')| = \Pi(\beta')$,
    так как изоморфные группы имеют одинаковые порядки.
\end{proof}

Таким образом, для~всякой алгебраической подгруппы $H$ тора можно определить число $\Pi(H)$
равное $\Pi(\beta)$ для~всякого $\beta$ такого, что $H = \ker(\phi_\beta)$.

Теперь для~всякой алгебраической подгруппы $H \subseteq (\complex^\times)^n$ определим \textit{компоненту единицы} $H^\circ$
как компоненту связности, содержащую нейтральный элемент группы. Из~теоремы \ref{theorem:connectedComponentsNumbers} следует,
что $\Pi(H^\circ) = 1$. Таким образом, доказаны следующие утверждения.

\begin{theorem}
    Алгебраическая подгруппа $H$ тора $(K^\times)^n$ параметризуема тогда и~только тогда, когда $\Pi(H) = 1$.
\end{theorem}

\begin{consequence*}
    Алгебраическая подгруппа $H$ тора $(\complex^\times)^n$ параметризуема тогда и~только тогда, когда является связной.
\end{consequence*}

\begin{consequence*}
    Всякая алгебраическая подгруппа $H$ тора $(\complex^\times)^n$ содержит параметризуемую подгруппу $H^\circ$ той~же размерности.
\end{consequence*}

\begin{proof}
    Все связные компоненты $H$ гомеоморфны, что немедленно доказывает теорему.
\end{proof}

\section{Линейная независимость над абелевой группой}

Во~всякой абелевой группе $G$ естественным образом определено умножение на~целые числа.
Для~вектора $\alpha = (\alpha_1, \ldots, \alpha_k) \in \Z^k$ и~элемента $g \in G$
под~$g \alpha$ понимаем вектор $(\alpha_1 g, \ldots, \alpha_k g) \in G^k$.
Говорим, что набор векторов $\alpha_1, \ldots, \alpha_n \in \Z^k$ \textit{линейно независим над~абелевой группой} $G$, если
$$
    \forall g_1, \ldots, g_n \in G{:}\ g_1 \alpha_1 + \ldots + g_n \alpha_n = 0 \Rightarrow g_1 = \ldots = g_n = 0.
$$

Видно, что линейная независимость векторов $\Z^k$ над~аддитивной группой $\R$ эквивалентна линейной независимости
в~векторном пространстве $\R^k$. Однако, например, $\R / 2 \pi \Z$, как известно, не~обладает структурой кольца,
поэтому и~не~имеет смысла говорить о~модуле $(\R / 2 \pi \Z)^k$ над~$\R / 2 \pi \Z$, как и~о~линейной независимости
в~нём.

Сразу отметим, каким образом данное определение связано с~изучаемыми параметризациями $\phi_\alpha$.

\begin{theorem}
    $\phi_\alpha : G^k \rightarrow G^n$ инъективно тогда и~только тогда, когда $\alpha^j$ линейно независимы над~группой $G$.
\end{theorem}

\begin{proof}
    Имеет место цепочка эквивалентностей:
    \begin{align*}
                       \ & \phi_\alpha\ \text{инъективно}
        \Leftrightarrow \ker(\phi_\alpha) = \{ 1 \}
        \Leftrightarrow \ker(\phi_\alpha) \subseteq \{ 1 \} \\
        \Leftrightarrow\ & (\forall t \in G^k{:}\ t^{\alpha_1} = 1 \wedge \ldots \wedge t^{\alpha_n} = 1 \Rightarrow t = 1) \\
        \Leftrightarrow\ & (\forall t_1, \ldots, t_k \in G{:}\ t_1^{\alpha^1} \ldots t_k^{\alpha^k} = 1 \Rightarrow t_1 = \ldots = t_k = 1).
    \end{align*}

    Но~последнее условие~— это в~точности определение линейной независимости над~$G$.
\end{proof}

Далее нам понадобятся две общие леммы.

\begin{lemma}
    Пусть $G$ и~$H$~— абелевы группы. $\mu_1, \ldots, \mu_n \in \Z^k$ линейно независимы над~$G \times H$
    тогда и~только тогда, когда они~же линейно независимы над~$G$ и~над~$H$.
\end{lemma}

\begin{proof}
    Элементы группа $G \times H$ представляются как пары $(g, h)$, где $g \in G$ и~$h \in H$, поэтому
    линейная независимость над~$G \times H$ запишется как:
    \begin{gather*}
        \forall (g_1, h_1) \ldots, (g_n, h_n) \in G \times H{:}\\ (g_1, h_1) \mu_1 + \ldots + (g_n, h_n) \mu_n = 0 \Rightarrow (g_1, h_1) = \ldots = (g_n, h_n) = 0.
    \end{gather*}
    Кроме того, ясно, что
    \begin{gather*}
        (g_1, h_1) \mu_1 + \ldots + (g_n, h_n) \mu_n = 0 \Leftrightarrow\\ g_1 \mu_1 + \ldots + g_n \mu_n = 0 \wedge h_1 \mu_1 + \ldots + h_n \mu_n = 0,
    \end{gather*}
    а~также
    $$
        (g_1, h_1) = \ldots = (g_n, h_n) = 0 \Leftrightarrow g_1 = \ldots = g_n = 0 \wedge h_1 = \ldots = h_n = 0.
    $$

    Зафиксировав поочерёдно $g_1 = \ldots = g_n = 0$ и~$h_1 = \ldots = h_n = 0$, получим импликацию слева направо.
    Импликация справа налево очевидна в~свете выше указанных эквивалентностей.
\end{proof}

\begin{lemma}
    Пусть $G$ и~$H$~— абелевы группы, $f : G \rightarrow H$~— изоморфизм. Тогда $\mu_1, \ldots, \mu_n \in \Z^k$
    линейно независимы над~$G$ тогда и~только тогда, когда они линейно независимы над~$H$.
\end{lemma}

\begin{proof}
    Поскольку $f$~— изоморфизм, получаем цепочку эквивалентностей:
    \begin{align*}
                       &\ \forall g_1, \ldots, g_n \in G{:}\ g_1 \alpha_1 + \ldots + g_n \alpha_n = 0 \Rightarrow g_1 = \ldots = g_n = 0 \\
        \Leftrightarrow&\ \forall g_1, \ldots, g_n \in G{:}\ f(g_1 \alpha_1 + \ldots + g_n \alpha_n) = 0 \Rightarrow g_1 = \ldots = g_n = 0 \\
        \Leftrightarrow&\ \forall g_1, \ldots, g_n \in G{:}\ f(g_1) \alpha_1 + \ldots + f(g_n) \alpha_n = 0 \Rightarrow f(g_1) = \ldots = f(g_n) = 0 \\
        \Leftrightarrow&\ \forall h_1, \ldots, h_n \in G{:}\ h_1 \alpha_1 + \ldots + h_n \alpha_n = 0 \Rightarrow h_1 = \ldots = h_n = 0.
    \end{align*}
\end{proof}

Применив леммы, получим критерий инъективности для~случая $K = \complex$.

\begin{consequence*}
    $\phi_\alpha$ инъективно тогда и~только тогда, когда $\alpha^j$ линейно независимы над~группами $\R$ и~$\R / 2 \pi \Z$.
\end{consequence*}

\begin{proof}
    Пусть $S^1 = \{ z \in \complex\ | \ |z| = 1 \}$~— группа окружности, подгруппа в~$\torus$.
    Посредством тригонометрического представления $z = re^{i \theta}$ группа $\torus$
    изоморфна произведению $\R_{> 0}^\times \times S^1$. Отображение $t \mapsto e^t$ определяет
    изоморфизм групп $\R$ и~$\R_{> 0}^\times$, а~отображение $\theta \mapsto e^{i\theta}$~— групп $\R / 2 \pi \Z$ и~$S^1$;
    что~доказывает следствие в~силу двух предыдущих лемм.
\end{proof}

\begin{consequence*}
    Если $\phi_\alpha$ инъективно над~$K = \complex$, то~$\rank(\alpha) = k$.
\end{consequence*}

\begin{proof}
    Как отмечалось выше, линейная независимость над~$\R$ эквивалентна стандартной линейно независимости
    в~векторном пространстве $\R^k$ над~$\R$, а~она, в~свою очередь, эквивалентна полноте ранга.
\end{proof}

В~частности, это означает, что параметризация с~числом переменных $k > n$ заведомо неинъективна, чего и~следовало ожидать.

Заметив, что $\R_{>0}^\times \cong \R_{>0}^\times \times \Z / 1\Z$, $\R^\times \cong \R_{>0}^\times \times \Z / 2 \Z$
и~$\quaternion^\times \cong \R_{>0}^\times \times S^3$ (кватернионы), легко получить подобные условия для~инъективности
$\phi_\alpha$ над~группами $\R_{>0}^\times$, $\R^\times$ и~$\quaternion^\times$.

Естественным кажется, что, поскольку рассматриваемые вектора $\alpha^j$ целочисленные, их линейная независимость
над~$\R$ должна сводиться к~линейной независимости над~$\Z$. Докажем это.

\begin{lemma*}
    $\mu_1, \ldots, \mu_n \in \Z^k$ линейно независимы над~$\R$ тогда и~только тогда, когда линейно независимы над~$\Q$.
\end{lemma*}

\begin{proof}
    Импликация слева направа очевидна. Обратно, пусть $\mu_1,\allowbreak \ldots,\allowbreak \mu_n$ линейно независимы над~$\Q$.
    Рассмотрим их линейную комбинацию: $r_1 \mu_1 + \ldots + r_n \mu_n = 0$.

    Как известно, $\R$ является (бесконечномерным) векторным пространством над~$\Q$, а~всякое векторное пространство
    при~условии аксиомы выбора имеет базис (Гамеля) \cite{Brbk70}. Пользуясь этим, зафиксируем базис Гамеля $B$ для~$\R$ над~$\Q$.
    Разложим $r_i$ по~этому базису:
    $$
        r_i = q_i^1 b_1 + \ldots + q_i^s b_s,
    $$
    где $q_i^j \in \Q$ и~$b_j \in B$. Векторов $r_i$ конечное число, поэтому наборы базисных векторов $b_j$ для~них можно выбрать одинаковыми.

    Подставим разложение в~линейную комбинацию:
    \begin{align*}
        r_1 \mu_1 + \ldots + r_n \mu_n & = (q_1^1 b_1 + \ldots + q_1^s b_s) \mu_1 + \ldots + (q_n^1 b_1 + \ldots + q_n^s b_s) \mu_n \\
                                       & = (q_1^1 \mu_1 + \ldots + q_n^1 \mu_n) b_1 + \ldots + (q_1^s \mu_1 + \ldots + q_n^s \mu_n) b_s \\
                                       & = 0.
    \end{align*}

    В~каждой проекции получаем нулевую рациональную линейную комбинацию чисел $b_j$. В~силу линейной независимости,
    все эти проекции равны нулю, поэтому равны нулю и~составленные из~них векторы: $q_1^j \mu_1 + \ldots + q_n^j \mu_n = 0$.

    Однако все $q_i^j$ рациональны, а~$\mu_i$ линейно независимы над~$\Q$, поэтому $q_i^j = 0$.
    Таким образом, $r_i = q_i^1 b_1 + \ldots + q_i^s b_s = 0 \cdot b_1 + \ldots + 0 \cdot b_s = 0$.
\end{proof}

\begin{lemma*}
    $\mu_1, \ldots, \mu_n \in \Z^k$ линейно независимы над~$\Q$ тогда и~только тогда, когда линейно независимы над~$\Z$.
\end{lemma*}

\begin{proof}
    Импликация слева направа снова очевидна. Обратно, пусть $q_1 \mu_1 + \ldots + q_n \mu_n = 0$, где $q_i \in \Q$.
    Выберем для~дробей $q_i$ общий знаменатель $q \in \N \setminus \{ 0 \}$. Числители обозначим как $p_i \in \Z$, то~есть $q_i = p_i / q$.

    В~таком случае $(p_1 \mu_1 + \ldots + p_n \mu_n) / q = 0$, откуда и~$p_1 \mu_1 + \ldots + p_n \mu_n = 0$.
    Все $p_i$~— целые числа, поэтому по~линейной независимости над~$\Z$ получаем, что $p_1 = \ldots = p_n = 0$.
    Таким образом, $q_i = p_i / q = 0 / q = 0$.
\end{proof}

\begin{consequence*}
    $\mu_1, \ldots, \mu_n \in \Z^k$ линейно независимы над~$\R$ тогда и~только тогда, когда линейно независимы над~$\Z$.
\end{consequence*}

Изучим подробнее следствия из~линейной независимости над~$\R / 2 \pi \Z$.

\begin{statement*}
    $\mu_1, \ldots, \mu_n$ линейно независимы над~$\R / 2 \pi \Z$
    тогда и~только тогда, когда они~же линейно независимы над~$\R / \Z$.
\end{statement*}

\begin{proof}
    Следует из~того, что группы $\R / 2 \pi \Z$ и~$\R / \Z$ изоморфны.
\end{proof}

Далее для~числа $d \in \Z$ и~вектора $x \in \Z^k$ под~$d \divides x$ понимаем, что $d \divides x_i$ для~всех $1 \leq i \leq k$
или, что эквивалентно, $d \divides \gcd(x_1, \ldots, x_n)$.

\begin{lemma*}
    Если $\mu_1, \ldots, \mu_n$ линейно независимы над~$\R / \Z$, то:
    $$
        \forall d \in \Z{:}\ \forall x \in \Z^n{:}\ d \divides x_1 \mu_1 + \ldots + x_n \mu_n \Rightarrow d \divides x.
    $$
\end{lemma*}

\begin{proof}
    Действительно, рассмотрим дроби $q_i = x_i / d \in \R / \Z$. Делимость $x_1 \mu_1 + \ldots + x_n \mu_n$ на~$d$ означает,
    что в~$\R / \Z$ выполнено $(x_1 \mu_1 + \ldots + x_n \mu_n) / d = 0$, то~есть $q_1 \mu_1 + \ldots + q_n \mu_n = 0$.
    Однако в~силу линейной независимости верно, что $q_1 = \ldots = q_n = 0$, а~это означает $d \divides x_i$ для~всех $1 \leq i \leq n$.
\end{proof}

Из~доказанного немедленно следует простой критерий для~проверки неинъективности $\phi_\alpha$.

\begin{lemma*}
    Если $\phi_\alpha$ инъективно, то~$\forall j{:}\ \gcd(\alpha^j) = 1$.
\end{lemma*}

\begin{consequence*}
    Если $\exists j{:}\ \gcd(\alpha^j) \neq 1$, то~$\phi_\alpha$ не~является инъективным отображением.
\end{consequence*}

\begin{proof}
    Выбрав $x_j = 1$ и~$x_1 = \ldots = x_{j - 1} = x_{j + 1} = \ldots = x_k = 0$, получим, что:
    \begin{align*}
                         & \forall d \in \Z{:}\ \forall x \in \Z^k{:}\ d \divides x_1 \alpha^1 + \ldots + x_k \alpha^k \Rightarrow d \divides x \\
            \Rightarrow\ & \forall j{:}\ \forall d \in \Z{:}\ d \divides \alpha^j \Rightarrow d \divides 1 \\
        \Leftrightarrow\ & \forall j{:}\ \gcd(\alpha^j) = 1.
    \end{align*}

    Нетрудно видеть, что использование поля $\complex$ в~этой лемме \textit{существенно;} поскольку, например,
    отображение $t \mapsto (t^3, t^2)$ инъективно над~$\R$, но~не~над~$\complex$.
\end{proof}

Множество простых чисел обозначим как $\primes \subseteq \Z$.

\begin{lemma*}
    Для~произвольных целочисленных векторов $\mu_1, \ldots, \mu_n$ верно, что:
    \begin{align*}
                         & \forall d \in \Z{:}\ \forall x \in \Z^n{:}\ d \divides x_1 \mu_1 + \ldots + x_n \mu_n \Rightarrow d \divides x \\
        \Leftrightarrow\ & \forall p \in \primes{:}\ \forall \beta \in \N{:}\ \forall x \in \Z^n{:}\ p^\beta \divides x_1 \mu_1 + \ldots + x_n \mu_n \Rightarrow p^\beta \divides x \\
        \Leftrightarrow\ & \forall p \in \primes{:}\ \forall x \in \Z^n{:}\ p \divides x_1 \mu_1 + \ldots + x_n \mu_n \Rightarrow p \divides x.
    \end{align*}
\end{lemma*}

\begin{proof}
    Импликации слева направо очевидны. Докажем импликации справа налево.

    Зафиксируем произвольное $d \in \Z$ и~некоторый вектор $x \in \Z^k$.
    Разложим $d$ на~простые множители: $d = p_1^{\beta_1} \ldots p_m^{\beta_m}$.

    Поскольку $x_1 \mu_1 + \ldots + x_n \mu_n$ делится на~$d$, то~оно делится и~на~каждое $p_i^{\beta_i}$.
    Пользуясь предпосылкой, получаем, что $p_i^{\beta_i} \divides x$ для~всех $1 \leq i \leq m$;
    но~тогда и~$d = p_1^{\beta_1} \ldots p_m^{\beta_m} \divides x$, что и~требовалось.

    Далее, зафиксируем степень $\beta \in \N$ и~простое число $p$.
    Поскольку $x_1 \mu_1 + \ldots + x_n \mu_n$ делится на~$p^\beta$,
    то~оно делится и~на~$p$, поэтому из~предпосылки следует, что $p \divides x$.

    Но~это означает, что $x_i / p$~— целые числа, а~$(x_1 / p) \mu_1 + \ldots + (x_n / p) \mu_n$ делится на~$p^{\beta - 1}$.
    Снова применив предпосылку, получим, что $p \divides x / p$.
    Повторив эту процедуру $\beta$ раз, окончательно заключим, что $p \divides x / p^{\beta - 1}$;
    но~это эквивалентно $p^\beta \divides x$.
\end{proof}

Последнее условие можно переписать как линейную независимость векторов $\mu_i$ над~полями $\Z / p\Z$ для~всех простых $p$,
что эквивалентно условию на~максимальность ранга матрицы $\mu$.

\begin{consequence*}
    Для~произвольных целочисленных векторов $\mu_1, \ldots, \mu_n$ верно, что:
    \begin{align*}
                         & \forall d \in \Z{:}\ \forall x \in \Z^n{:}\ d \divides x_1 \mu_1 + \ldots + x_n \mu_n \Rightarrow d \divides x \\
        \Leftrightarrow\ & \forall p \in \primes{:}\ \forall x \in (\Z / p\Z)^n{:}\ x_1 \mu_1 + \ldots + x_n \mu_n = 0 \Rightarrow x = 0 \\
        \Leftrightarrow\ & \forall p \in \primes{:}\ \mu_1, \ldots, \mu_n\ \text{линейно независимы над~полем}\ \Z / p\Z \\
        \Leftrightarrow\ & \forall p \in \primes{:}\ \rank_{\Z / p \Z}(\mu) = \max(n, k).
    \end{align*}
\end{consequence*}

Наконец, мы можем передоказать уже упомянутое достаточное условие.

\begin{lemma*}
    Если $\phi_\alpha$ инъективно, то~$\alpha_i$ порождают всю решётку.
\end{lemma*}

\begin{proof}
    От~противного. Пусть $\alpha_i$ не~порождают решётку. Тогда миноры максимальной размерности
    матрицы $\alpha$ имеют общий делитель $d > 1$.

    Возьмём некоторый простой делитель $p$ числа $d$. Поскольку максимальные миноры
    делятся на~$d$, то~они делятся и~на~$p$; поэтому в~поле $\Z / p \Z$
    все максимальные миноры $\alpha$ равны нулю, что означает $\rank_{\Z / p \Z}(\alpha) < \max(n, k)$ \cite{Brbk70},
    но~это противоречит заключению из~следствия.
\end{proof}

\begin{consequence*}
    $\phi_\alpha$ инъективно тогда и~только тогда, когда~$\alpha_i$ порождают всю решётку.
\end{consequence*}

\section{Голоморфные характеры}

Пусть $K$~— поле. Для~каждого многочлена Лорана $f(z) = \sum_{i \in I} a_i z^i$ и~индекса $k \in \Z^n$ рассмотрим отображение
$$
    \mathrm{coeff}_k(f) =
    \begin{cases}
        a_k, & \text{если $k \in I$}; \\
        0,   & \text{иначе.}
    \end{cases}
$$

Нетрудно видеть, что справедливы следующие утверждения.
\begin{lemma*}
    Для~любых многочленов Лорана $f$ и~$g$ и~вектора $w \in K^n$ верно, что:
    \begin{enumerate}
        \item $\mathrm{coeff}_k(0) = 0$.
        \item $\mathrm{coeff}_k(f + g) = \mathrm{coeff}_k(f) + \mathrm{coeff}_k(g)$.
        \item $\mathrm{coeff}_k(w f) = w \cdot \mathrm{coeff}_k(f)$.
        \item $\mathrm{coeff}_k(f \circ (z \mapsto wz)) = w^k \cdot \mathrm{coeff}_k(f)$.
    \end{enumerate}
\end{lemma*}

\begin{theorem*}
    Всякий многочлен Лорана $f \in \Hom((K^\times)^n, K^\times)$ является мономом.
\end{theorem*}

\begin{proof}
    Действительно, для~произвольного $w \in K^n$ в~силу того, что $f$~— гомоморфизм,
    выполняется равенство $f \circ (z \mapsto wz) = f(w) \cdot f$. Откуда:
    $$
        w^k \cdot \mathrm{coeff}_k(f) = \mathrm{coeff}_k(f \circ (z \mapsto wz)) = \mathrm{coeff}_k(f(w) \cdot f) = f(w) \cdot \mathrm{coeff}_k(f),
    $$
    а~значит $(f(w) - w^k) \cdot \mathrm{coeff}_k(f) = 0$.
    Поскольку $f$ нигде не~равен нулю, существует $k_0$ такой, что $\mathrm{coeff}_{k_0}(f) \neq 0$;
    но~тогда $f(w) - w^{k_0} = 0$ для~всех $w$.
\end{proof}

Отдельно рассмотрим случай $K = \complex$. Как известно, в~этом случае коэффициенты можно представить как интегралы:
$$
    \mathrm{coeff}_k(f) = \frac{1}{(2 \pi i)^n} \oint_{T_\varepsilon} \frac{f(z)}{z^k} \frac{dz_1}{z_1} \wedge \ldots \wedge \frac{dz_n}{z_n},
$$
где $T_\varepsilon = \{ z \in (\torus)^n\ |\ |z_1| = \varepsilon_1 \wedge \ldots \wedge |z_n| = \varepsilon_n \}$~— вещественный тор,
а~$\varepsilon = (\varepsilon_1, \ldots, \varepsilon_n)$~— вектор положительных чисел.

Пусть теперь $f$~— голоморфная на~$(\torus)^n$ функция. Определим для~неё числа $\mathrm{coeff}_k(f)$ интегралами выше.
Тогда сформулированная лемма остаётся справедливой. Действительно, пункты~1—3 верны в~силу линейности интеграла.
Пункт~4 получим, сделав замену переменных $z \mapsto z \cdot w = \xi$ и~пользуясь тем, что интегралы
по~гомотопным циклам от~голоморфной функции равны.
\begin{align*}
    \mathrm{coeff}_k(f \circ (z \mapsto wz)) &= \frac{1}{(2 \pi i)^n} \oint_{T_\varepsilon} \frac{(f \circ (z \mapsto wz))(z)}{z^k} \frac{dz_1}{z_1} \wedge \ldots \wedge \frac{dz_n}{z_n} \\
                                             &= \frac{1}{(2 \pi i)^n} \oint_{T_\varepsilon} \frac{f(wz)}{z^k} \frac{dz_1}{z_1} \wedge \ldots \wedge \frac{dz_n}{z_n} \\
                                             &= \frac{1}{(2 \pi i)^n} \oint_{T_{\varepsilon'}} \frac{f(\xi)}{(\xi / w)^k} \frac{d(\xi_1 / w_1)}{\xi_1 / w_1} \wedge \ldots \wedge \frac{d(\xi_n / w_n)}{\xi_n / w_n} \\
                                             &= \frac{w^k}{(2 \pi i)^n} \oint_{T_{\varepsilon'}} \frac{f(\xi)}{\xi^k} \frac{d\xi_1}{\xi_1} \wedge \ldots \wedge \frac{d\xi_n}{\xi_n} \\
                                             &= \frac{w^k}{(2 \pi i)^n} \oint_{T_{\varepsilon}} \frac{f(\xi)}{\xi^k} \frac{d\xi_1}{\xi_1} \wedge \ldots \wedge \frac{d\xi_n}{\xi_n} \\
                                             &= w^k \cdot \mathrm{coeff}_k(f),
\end{align*}
где $\varepsilon' = (\varepsilon_1 |w_1|, \ldots, \varepsilon_n |w_n|)$.
Теперь аналогичная теорема доказывается без~изменений.

\begin{theorem*}
    Всякий голоморфный $f \in \Hom((\torus)^n, \torus)$~— моном, то~есть:
    $$
        \exists \alpha \in \Z^n{:}\ \forall z{:}\ f(z) = z^\alpha.
    $$
\end{theorem*}

\usection{Заключение}
\stepcounter{section}

Важнейшим инструментом в~комплексном анализе является понятие \textit{вычета} голоморфной функции.
Напомним основные сведения из~одномерной теории вычетов.

Множество функций, голоморфных в~открытом множестве $D \subseteq \complex$ (или, более общо, в~$D \subseteq \complex^n$), обозначим как $\Hol(D)$.

\begin{theorem*}
  Пусть $D \subseteq \complex$~— область. Если $f \in \Hol(D)$, а~кусочно-гладкие кривые $\gamma_1$
  и~$\gamma_2 \subseteq D$ гомотопны в~$D$, то
  $$
    \int_{\gamma_1} f(z)dz = \int_{\gamma_2} f(z)dz.
  $$
\end{theorem*}

Рассмотрим область $D$ и~её подмножество $D^\prime = D \setminus \{ z_1, \ldots, z_k \} \subseteq D$, где $z_i \in D$.

Из~теоремы в~частности следует, что для~произвольной $f \in \Hol(D^\prime)$ и~её изолированной
особой точки $z_0 \in D \setminus D^\prime$ интегралы $\oint_{\Gamma_\epsilon} f(z)dz$ по~окружностям
$\Gamma_\epsilon = \{ z \in D \ | \ |z - z_0| = \epsilon \}$ не~зависят от~$\epsilon \in \mathbb{R}_{+}$
при~достаточно малом $\epsilon$ (конкретнее, таком, что в~круг $\{ z \in D \ | \ |z - z_0| < \epsilon \}$
не~попадают никакие другие особые точки $f$, кроме выбранной $z_0$).

В~таком случае \textit{вычетом} $f$ в~точке $z = z_0$ называют интеграл
$$
  \operatorname*{res}\limits_{z_0} f = \operatorname*{res}\limits_{z = z_0} f(z) = \lim\limits_{\epsilon \to 0} \frac{1}{2 \pi i} \oint_{\Gamma_\epsilon} f(z)dz = \frac{1}{2 \pi i} \oint_{\Gamma_{\epsilon_0}} f(z)dz,
$$
где $\epsilon_0$ достаточно мало в~описанном выше смысле.

Следующая теорема позволяет сводить вычисление интеграла от~голоморфной функции к~вычислению вычетов в~её особых точках,
что вместе со~сравнительной простотой вычисления обосновывает их практическое применение.
\begin{theorem*}[основная о вычетах]
  Пусть $D \subseteq \complex$~— односвязная область, $\gamma \subseteq D$~— замкнутая кусочно-гладкая кривая,
  $z_i \in D$, $f \in \Hol(D^\prime)$, где $D^\prime = D \setminus \{ z_1, \ldots, z_k \}$.
  Тогда:
  $$
    \oint_{\gamma} f(z)dz = 2 \pi i \sum_{k = 1}^{n} \mathrm{Ind}_{\gamma}(z_k) \operatorname*{res}\limits_{z_k} f,
  $$
  где $\mathrm{Ind}_{\gamma}(z)$~— индекс точки $z$ относительно кривой $\gamma$.
\end{theorem*}

Для~построения описанных конструкций существенным фактом было то, что одномерные голоморфные функции допускают
существование \textit{изолированных} особых точек. Однако, как показывает следующая теорема \cite{ShaII}, в~многомерном случае
это допущение неверно.

\begin{theorem*}[о стирании компактных особенностей]
  Пусть $D \subseteq \complex^n$~— область, причём $n > 1$, $\overline{K} \subseteq D$,
  $D \setminus K$ связно. Тогда всякая $f \in \Hol(D \setminus K)$
  голоморфно продолжается в~$D$.
\end{theorem*}

Таким образом, в~случае $n > 1$ невозможно вокруг особой точки голоморфной функции
выбрать малую сферу так, чтобы функция была голоморфна на~всей сфере (в~противном случае,
функцию можно было~бы продолжить внутрь сферы); а~потому интеграл по~такой сфере, в~общем случае, не~определён.

В~отличие от~многомерных голоморфных функций, голоморфные отображение, то~есть функции $\complex^n \rightarrow \complex^n$,
\textit{могут} иметь изолированные нули или особые точки. Этот факт приводит к~построению так называемого \textit{вычета Гротендика.}

\subsection{Вычет Гротендика}

Зафиксируем набор многочленов Лорана $g_1, \ldots, g_n$ над~$\complex$.
Пусть $g(z) = g_1(z) \ldots g_n(z)$ и~$\Gamma = \{ z \in (\complex^\times)^n \ | \ g(z) = 0 \}$.
Также зафиксируем некоторое изолирование решение $z \in \Gamma$ системы $g_1(z) = \ldots = g_n(z) = 0$ и~его окрестность $U \subseteq (\complex^\times)^n$.
Для~всякого $\epsilon \in \mathbb{R}_{+}^n$ определим
$$
    \gamma_{z, \epsilon} = \{ z \in (\complex^\times)^n\ | \ \forall i{:}\ |g_i(z)| = \epsilon_i \}.
$$

Почти для~всех $\epsilon$ множество $\gamma_{z, \epsilon}$ представляет собой гладкое вещественное подмногообразие многообразия $U$.
Для~достаточно малых ненулевых $\epsilon$ многообразие $\gamma_{z, \epsilon}$ является компактным подмногообразием $U \setminus \Gamma$.
Определим на~нём ориентацию с~помощью дифференциальной формы $d(\mathrm{Arg}\ g_1) \wedge \ldots \wedge d(\mathrm{Arg}\ g_n)$.

\textit{Циклом Гротендика} $\gamma_z$ \cite{GelKho02} изолированного корня $z$ системы $g_1(z) = \ldots = g_n(z) = 0$ назовём многообразие $\gamma_{z, \epsilon}$ для~достаточно малого $\epsilon$.
Класс цикла в~$n$-мерной группе гомологий $U \setminus \Gamma$ не~зависит от~выбора $\epsilon$, но,~очевидно, зависит от~порядка уравнений $g_1(z) = 0, \ldots, g_n(z) = 0$.

Зафиксируем также ещё один многочлен Лорана $f$. \textit{Вычетом Гротендика} назовём следующий интеграл:
$$
  \operatorname*{res}\limits_{z} \frac{f}{g_1 \ldots g_n} = \frac{1}{(2 \pi i)^n} \int_{\gamma_z} \frac{f}{g_1 \ldots g_n}.
$$

Существенную трудность при~его вычислении представляет устройство цикла $\gamma_z$.
В~случае $g_i(z) = z$ цикл $\gamma_z$ является просто $n$-мерным тором, и~вычисление вычета
сводится к~вычислению коэффициента в~ряде~Лорана, как и~одномерном случае; однако в~общем случае
нетривиальна даже топологическая структура $\gamma_z$.

\subsection{Кобордизмы и теория Морса}

stub

\pagebreak

\usection{Список использованных источников}

\begin{thebibliography}{99}

\bibitem{Schm94} Schmidt, W.~M. Heights of points on subvarieties of $\mathbb{G}^n_m$~/ W.~M.~Schmidt~//
London Mathematical Society Lecture Note Series. Issue~235: Number Theory.
Séminaire de~théorie des~nombres de~Paris 1993–94.~— 1996.~— P.~157—187.

\bibitem{Art48} Artin, Е. Galois Theory~/ E.~Artin; London : University of~Notre Dame, 1942, 1944.~— 96~p.~— ISBN: 978-0~48~662342~9.

\bibitem{Smth60} Smith, H.J.S. On~systems of~linear indeterminate equations and~congruences. Philos.~/ H.J.S.~Smith~//
Philosophical Transactions of~the~Royal Society.~— 1861.~— Vol.~151.~— P.~293—326.

\bibitem{TsikhSad14} Садыков, Т.~М. Гипергеометрические и~алгебраические функции многих переменных~/ Т.~М.~Садыков, А.~К.~Цих;
Москва : Наука, 2014.~— 408~с.~— ISBN: 978-5-02-039082-9.

\bibitem{Brbk70} Bourbaki, N. Algèbre: Chapitres 1 à~3. Éléments de~mathématique~/ N.~Bourbaki; Berlin : Springer, 2006.~— 652~p.~— ISBN: 978-3-54-033849-9.

\bibitem{ShaII} Shabat, B.~V. Introduction to~complex analysis. Part~II. Functions of~several variables / B.~V.~Shabat; Providence : American Mathematical Society, 1992.~— 371~p.~— ISBN: 978-0-82-181975-3.

\bibitem{GelKho02} Gelfond, O.~A. Toric geometry and~Grothendieck residues~/ O.~A.~Gelfond, A.~G.~Khovanskii~//
Moscow Mathematical Journal.~— 2002.~— Vol.~2.~— P.~99—112.

\bibitem{MacLane71} MacLane, S. Categories for the~Working Mathematician~/ S.~MacLane; New~York : Springer-Verlag, 1971.~— 262~p.~— ISBN: 978-0-38-790035-3.

\bibitem{Mil65} Milnor, J. Lectures on~the~H-Cobordism Theorem~/ J.~Milnor; Princeton : Princeton University Press, 1965.~— 124~p.~— ISBN: 978-0-69-165113-2.

\bibitem{Stong68} Stong, R.~E. Notes on~Cobordism Theory~/ R.~E.~Stong; Princeton : Princeton University Press, 1969.~— 422~p.~— ISBN: 978-0-69-162221-7.

\bibitem{Weston} Weston T., An~introduction to~cobordism theory~/ T.~Weston~//
{Stan\-ford} University. School of~Humanities and~Sciences.
The~Department of~{Math\-e\-mat\-ics} : [сайт].~— URL: \url{https://math.stanford.edu/~ralph/morsecourse/cobordismintro%20.pdf} (дата обращения: 23.08.2022).

\end{thebibliography}

\label{TheEnd}
\pagebreak

\renewcommand{\thesection}{\Alph{section}}
\setcounter{section}{1}
\setcounter{subsection}{0}

\usection{Приложение}

В~приложении формализуем общую концепцию «теории кобордизмов» в~терминах «категорий с~кобордизмами».
Перед этим изложим основные сведения из~теории категорий.

\subsection{Основные понятия теории категорий}

Под~\textit{ориентированным графом} (или направленным) \cite{MacLane71} понимаем упорядоченную четвёртку $(O,\allowbreak A,\allowbreak \dom,\allowbreak \cod)$,
где $A$ и~$O$~— множества, $\dom$ и~$\cod$~— функции $A \rightarrow O$.
Для~графа определим множество \textit{перемножаемых стрелок} как $A \times_O A = \{ (g, f) \in A \times A \ |\ \dom(g) = \cod(f) \}$.

\textit{Категорией} называется граф вместе с~парой функций $\id : O \rightarrow A$
и~$\circ : A \times_O A \rightarrow A$ таких, что для~всех $a, b \in O$ и~$f, g, h \in A$ верно:
\begin{enumerate}
    \item $\dom(\id(a)) = \cod(\id(a)) = a$.
    \item $\dom(f \circ g) = \dom(f)$, если $(f, g) \in A \times_O A$.
    \item $\cod(f \circ g) = \cod(g)$, если $(f, g) \in A \times_O A$.
    \item $f \circ (g \circ h) = (f \circ g) \circ h$, если $(f, g), (g, h) \in A \times_O A$ (ассоциативность).
    \item $f \circ \id(a) = f$, если $\dom(f) = a$ (правая унитальность).
    \item $\id(b) \circ f = f$, если $\cod(f) = b$ (левая унитальность).
\end{enumerate}

Для~категории $C = (O, A, \dom, \cod, \id, \circ)$ элементы множества $O$ называются \textit{объектами},
а~элементы множества $A$~— \textit{стрелками} (или морфизмами) этой категории. Обозначим $\Ob(C) = O$ и~$\Arr(C) = A$.
Стрелки $\id(x)$ называются \textit{единичными,} а~стрелка $f \circ g$~— \textit{композицией.} Для~категории $C$ также будем писать $\id_C$ и~$\circ_C$.
Для~объектов $a, b \in \Ob(C)$ определим \textit{множество стрелок}:
$$
    \hom_C(a, b) = \{ f \in \Arr(C)\ | \ \dom(f) = a \wedge \cod(f) = b \}.
$$

\textit{Функтором} между категориями $A$ и~$B$ называется упорядоченная пара $T = (T_1, T_2)$,
где $T_1$~— функция $\Ob(A) \rightarrow \Ob(B)$, а~$T_2$~— $\Arr(A) \rightarrow \Arr(B)$,
такая, что для~всяких объектов $a, b, c \in \Ob(A)$ и~стрелок $f \in \hom_C(a, b)$ и~$g \in \hom_C(b, c)$ выполняются
следующие условия:
\begin{enumerate}
    \item $T_2(f) \in \hom_C(T_1(a), T_1(b))$.
    \item $T_2(\id(a)) = \id(T_1(a))$.
    \item $T_2(g \circ f) = T_2(g) \circ T_2(f)$.
\end{enumerate}

В~этом случае пишем $T : A \rightarrow B$. $T_1$ называется \textit{объектной функцией,} а~$T_2$~— \textit{стрелочной.}
Всюду далее будем следовать общепринятому соглашению: объектную и~стрелочную функции одного функтора обозначаем одной и~той~же буквой.
Функтор категории в~себя также называется \textit{эндофунктором.}

Для~всякой категории определён единичный функтор $1_C : C \rightarrow C$, объектная и~стрелочная функции которого~— тождественные.
Для~зафиксированных категорий $A$, $B$ и~объекта $b \in \Ob(B)$ определён \textit{диагональный функтор} $\Delta_b : A \rightarrow B$
такой, что $\Delta_b(a) = b$ и~$\Delta_b(f) = \id_B(b)$ для~любых $a \in \Ob(A)$ и~$f \in \Arr(A)$.

Если дана пара функтор $F = (F_1, F_2) : B \rightarrow C$ и~$G = (G_1, G_2) : A \rightarrow B$,
то~определена их \textit{композиция} $F \circ G = (F_1 \circ G_1, F_2 \circ G_2) : A \rightarrow C$.

Стрелка $f \in \hom_C(a, b)$ называется \textit{изоморфизмом,} если существует стрелка $g \in \hom_C(b, a)$
такая, что $f \circ g = \id(b)$ и~$g \circ f = \id(a)$. Тогда также пишем $f : a \cong b$.
Нетрудно видеть, что функтор переводит изоморфизмы в~изоморфизмы.

\textit{Подграфом} ориентированного графа $D$ называется граф $D'$ такой, что $\Ob(D') \subseteq \Ob(D)$, $\Arr(D') \subseteq \Arr(D)$,
а~также $\dom_{D'} = \dom_{D} |_{\Arr(D')}$ и~$\cod_{D'} = \cod_{D} |_{\Arr(D')}$.
Говорим, что граф $D$ \textit{содержит} (под)граф $D'$.
Аналогично говорим, что $C'$~— \textit{подкатегория} в~$C$, если $C'$ в~ней подграф, $\id_{C'} = \id_{C} |_{\Ob(C')}$
и~$\circ_{C'} = \circ_{C} |_{\Arr(C') \times \Arr(C')}$.
Если задан произвольный набор подкатегорий $C_i$ для~$i \in I$, то~ясно, что определено их пересечение, которое тоже является подкатегорией:
$$
    \bigcap\limits_{i \in I} C_i = \left( \bigcap\limits_{i \in I} \Ob(C_i), \bigcap\limits_{i \in I} \Arr(C_i), \dom, \cod, \id, \circ \right).
$$

\textit{Диаграммой} в~категории называем любой (обычно конечный либо счётный) её подграф.
Как и~всякий ориентированный граф, диграмма графически изображается с~помощью вершин,
в~теории категорий обычно обозначаемых буквами без~дополнительной обводки, и~стрелок.
Например:
\[
    \begin{tikzcd}
        \arrow{d}{h_1} A \arrow{r}{f} & B \arrow{d}{h_2} \\
                       C \arrow{r}{g} & D.
    \end{tikzcd}
\]

Для~всякого подграфа $D$ категории $C$ пересечение всех содержащих его подкатегорий
(то~есть наименьшая подкатегория, содержащая данный подграф) назовём \textit{пополнением} $\overline{D}$.

Категория $C$ называется \textit{тонкой,} если между любой парой объектов имеется не~более одного морфизма,
то~есть $$\forall a, b \in \Ob(C){:}\ \forall f, g \in \hom_C(a, b){:}\ f = g.$$
Говорим, что \textit{диаграмма коммутативна,} если её пополнение~— тонкая категория.

Если даны функторы $F, G : A \rightarrow B$, то~\textit{естественным преобразованием} $\eta : F \Rightarrow G$ называется набор морфизмов
$\eta_x : \hom_B(F(x), G(x))$ для~каждого объекта $x \in \Ob(A)$ такой, что $G(f) \circ \eta_x = \eta_y \circ F(f)$
для~всякой стрелки $f \in \hom_A(x, y)$; то~есть следующая диаграмма коммутативна:
\[
  \begin{tikzcd}
    \arrow{d}{\eta_x} F(x) \arrow{r}{F(f)} & F(y) \arrow{d}{\eta_y} \\
                      G(x) \arrow{r}{G(f)} & G(y).
  \end{tikzcd}
\]

Стрелки $\eta_x$ называют \textit{компонентами} преобразования. Если все компонентны естественного
преобразования~— изоморфизмы, то~говорят, что определён \textit{естественный изоморфизм.} Пишем $\eta : F \cong G$.

Для~всякого функтора $F$ также обозначим как~$1_F$ тождественное естественное преобразование, все компонентны которого~— единичные стрелки.
Далее, если заданы естественные преобразования $\varepsilon : G \Rightarrow H$ и~$\eta : F \Rightarrow G$,
то~определена их \textit{вертикальная композиция} $\varepsilon \circ \eta : F \Rightarrow H$,
где $(\varepsilon \circ \eta)_x = \varepsilon_x \circ \eta_x$.

Как и~для~объектов в~категории, говорим, что категории $A$ и~$B$ \textit{изоморфны,} если существует
пара функторов $F : A \rightarrow B$ и~$G : B \rightarrow A$ такая, что $F \circ G = 1_B$ и~$G \circ F = 1_A$.
Изоморфизм категорий~— слишком сильное требование, поэтому в~теории категорий чаще используется более слабое
условие эквивалентности категорий, которое, тем не~менее, сохраняет все существенные категорные свойства.
По~аналогии с~переходом от~гомеоморфизма к~гомотопической эквивалентности в~топологии, для~которого нужно ослабить
равенство до~существования гомотопии, в~случае категорий равенство нужно ослабить до~естественного изоморфизма.
То~есть, категории $A$ и~$B$ эквивалентны, если существуют $F : A \rightarrow B$ и~$G : B \rightarrow A$
такие, что $F \circ G \cong 1_B$ и~$G \circ F \cong 1_A$.

\textit{Инициальным объектом} в~категории $C$ называется такой объект $0 \in \Ob(C)$,
что из~него во~всякой другой объект категории существует ровно одна стрелка:
$$
  \forall x \in \Ob(C){:}\ \exists! f \in \Arr(C){:}\ f \in \hom_C(0, x).
$$

Эту единственную стрелку обозначаем $!_x \in \hom_C(0, x)$.

\textit{Копроизведением} (или суммой) объектов $a$ и~$b \in \Ob(C)$ называется упорядоченная тройка $(c, \inl, \inr)$,
где $c \in \Ob(C)$, $\inl \in \hom_C(a, c)$ и~$\inr \in \hom_C(b, c)$, такая, что
выполнено следующее универсальное свойство: для~любого объекта $x \in \Ob(C)$ и~пары морфизмов $f \in \hom_C(a, x)$
и~$g \in \hom_C(b, x)$ существует единственный (универсальный) морфизм $h \in \hom_C(c, x)$ такой, что $h \circ \inl = f$
и~$h \circ \inr = g$; то~есть коммутативна следующая диаграмма:
\[
    \begin{tikzcd}
        & x & \\ a \arrow{r}{}[swap]{\inl} \arrow{ur}{f} & \arrow[dashed]{u}{h} c & \arrow{l}{\inr}[swap]{} \arrow{ul}{}[swap]{g} y.
    \end{tikzcd}
\]

Также пишем $c = a + b$, $\inl = \inl_{a, b}$, $\inr = \inr_{a, b}$
и~$h = \rec_{a, b}(f, g)$.

Наконец, функтор $T : A \rightarrow B$ \textit{полуаддитивен}\rlap{\textit{,}}\footnote{
    Нестандартный термин, не~следует путать с~полуаддитивной категорией.
} если для~всякого копроизведения $(c, i, j)$ объектов $a$ и~$b \in \Ob(A)$
тройка $(T(c), T(i), T(j))$~— копроизведение $T(a)$ и~$T(b)$. Функтор $T$ \textit{аддитивен,} если он полуаддитивен и~для~всякого инициального объекта
$o \in \Ob(A)$ объект $T(o)$~— инициальный в~$B$.

\begin{lemma*}[$\eta$-правило]
    Если тройка $(c, i, j)$ является копроизведением, $x \in \Ob(C)$
    и~$f \in \hom_C(c,\allowbreak x)$, то~$\rec(f \circ i, f \circ j) = f$.
\end{lemma*}

\begin{proof}
    Действительно, $\rec(f \circ i, f \circ j) \circ i = f \circ i$
    и~$\rec(f \circ i, f \circ j) \circ j = f \circ j$.
    В~силу единственности такой стрелки получаем лемму.
\end{proof}

\begin{lemma*}
    Пусть $(c, i, j)$~— копроизведение, $x \in \Ob(C)$, $f, g \in \hom_C(c, x)$.
    Если $f \circ i = g \circ i$ и~$f \circ j = g \circ j$, то~$f = g$.
\end{lemma*}

\begin{proof}
    По~доказанному ранее, $\rec(f \circ i, f \circ j) = f$ и~$\rec(g \circ i, g \circ j)$.
    Но~тогда имеем:
    $$
        f = \rec(f \circ i, f \circ j) = \rec(g \circ i, g \circ j) = g.
    $$
\end{proof}

\begin{lemma*}
    Пусть $(c_1, i_1, j_1)$ и~$(c_2, i_2, j_2)$~— копроизведения $a$ и~$b$.
    Тогда существует $\phi : c_1 \cong c_2$, причём $\phi \circ i_1 = i_2$
    и~$\phi \circ j_1 = j_2$.
\end{lemma*}

\begin{proof}
    Обозначим как $\rec_1$ универсальную стрелку для~первой тройки,
    а~как $\rec_2$~— для~второй. Пусть $\phi = \rec_1(i_2, j_2)$
    и~$\psi = \rec_2(i_1, j_1)$. Тогда:
    $$
        \phi \circ \psi \circ i_2 = \phi \circ \rec_2(i_1, j_1) \circ i_2
                                  = \phi \circ i_1
                                  = \rec_1(i_2, j_2) \circ i_1
                                  = i_2
                                  = \id_C(c_2) \circ i_2.
    $$

    Аналогично, $\phi \circ \psi \circ j_2 = \id_C(c_2) \circ j_2$.
    В~силу предыдущей леммы, $\phi \circ \psi = \id_C(c_2)$.
    Точно так~же получаем, что $\psi \circ \phi = \id_C(c_1)$.
    Наконец, $\phi \circ i_1 = \rec_1(i_2, j_2) \circ i_1 = i_2$
    и~$\phi \circ j_1 = \rec_1(i_2, j_2) \circ j_1 = j_2$.
\end{proof}

\begin{theorem*}[критерий полуаддитивности]
    Пусть категории $A$ и~$B$ содержат все копроизведения.
    Тогда функтор $T : A \rightarrow B$ полуаддитивен тогда и~только тогда, когда:
    \begin{enumerate}
        \item Для~всякой пары объектов $a, b \in \Ob(A)$ определён изоморфизм $\phi_{a, b} : T(a) + T(b) \cong T(a + b)$.
        \item $\phi_{a, b} \circ \inl_{T(a), T(b)} = T(\inl_{a, b})$.
        \item $\phi_{a, b} \circ \inr_{T(a), T(b)} = T(\inr_{a, b})$.
    \end{enumerate}
\end{theorem*}

\begin{proof}
    Слева направо: пусть $T$ полуаддитивен. Так как $(a + b,\allowbreak \inl_{a, b},\allowbreak \inr_{a, b})$~— копроизведение,
    по~полуаддитивности, $(T(a + b), T(\inl_{a, b}), T(\inr_{a, b}))$~— тоже.
    Копроизведение единственно, поэтому $\phi : T(a) + T(b) \cong T(a + b)$, причём $\phi \circ \inl_{T(a), T(b)} = T(\inl_{a, b})$
    и~$\phi \circ \inr_{T(a), T(b)} = T(\inr_{a, b})$.

    Обратно, пусть $(c, i, j)$~— некоторое копроизведение $a$ и~$b$. Докажем, что $(T(c), T(i), T(j))$~— копроизведение.
    Зафиксируем объект $x \in \Ob(B)$ и~пару стрелок $f \in \hom_B(T(a), x)$ и~$g \in \hom_B(T(b), x)$.

    Снова в~силу единственности имеем стрелку $\psi : c \cong a + b$.
    Тогда $h = \rec_{T(a), T(b)}(f, g) \circ \phi_{a, b}^{-1} \circ T(\psi) \in \hom_B(T(c), x)$, причём:
    \begin{align*}
        h \circ T(i) &= \rec_{T(a), T(b)}(f, g) \circ \phi_{a, b}^{-1} \circ T(\psi \circ i) \\
                     &= \rec_{T(a), T(b)}(f, g) \circ \phi_{a, b}^{-1} \circ T(\inl_{a, b}) \\
                     &= \rec_{T(a), T(b)}(f, g) \circ \inl_{T(a), T(b)} \\
                     &= f.
    \end{align*}
    Аналогично $h \circ T(j) = g$. Наконец, пусть $h' \in \hom_B(T(c), x)$,
    $h' \circ T(i) = f$ и~$h' \circ T(j) = g$.

    $u = T(\psi^{-1}) \circ \phi_{a, b}$~— изоморфизм как композиция изоморфизмов,
    поэтому $h = h' \Leftrightarrow h \circ u = h' \circ u$. Далее,
    \begin{align*}
        h \circ u \circ \inl_{T(a), T(b)} &= \rec_{T(a), T(b)}(f, g) \circ \phi_{a, b}^{-1} \circ T(\psi \circ \psi^{-1}) \circ \phi_{a, b} \circ \inl_{T(a), T(b)} \\
                                          &= \rec_{T(a), T(b)}(f, g) \circ \inl_{T(a), T(b)} \\
                                          &= f \\
                                          &= h' \circ T(i) \\
                                          &= h' \circ u \circ u^{-1} \circ T(i) \\
                                          &= h' \circ u \circ \phi_{a,b}^{-1} \circ T(\psi) \circ T(i) \\
                                          &= h' \circ u \circ \phi_{a,b}^{-1} \circ T(\psi \circ i) \\
                                          &= h' \circ u \circ \phi_{a,b}^{-1} \circ T(\inl_{a, b}) \\
                                          &= h' \circ u \circ \inl_{T(a), T(b)}.
    \end{align*}
    Аналогично, $h \circ u \circ \inr_{T(a), T(b)} = h' \circ u \circ \inr_{T(a), T(b)}$, поэтому $h \circ u = h' \circ u$.
\end{proof}

\begin{theorem*}[критерий аддитивности]
    Пусть категории $A$ и~$B$ содержат все копроизведения и~инициальный объект.
    Тогда функтор $T : A \rightarrow B$ аддитивен тогда и~только тогда, когда:
    \begin{enumerate}
        \item Для~всякой пары объектов $a, b \in \Ob(A)$ определён изоморфизм $\phi_{a, b} : T(a) + T(b) \cong T(a + b)$.
        \item $\phi_{a, b} \circ \inl_{T(a), T(b)} = T(\inl_{a, b})$.
        \item $\phi_{a, b} \circ \inr_{T(a), T(b)} = T(\inr_{a, b})$.
        \item $T(0_A) \cong 0_B$.
    \end{enumerate}
\end{theorem*}

\begin{proof}
    Немедленно следует из~единственности инициального объекта и~предыдущей теоремы.
\end{proof}

\subsection{Категории с кобордизмами}

\textit{Категорией с~кобордизмами} называем\footnote{
    В~английском языке «cobordism category» используется для~обозначения как вводимого понятия,
    так и~для~категории, объекты которой~— многообразия, а~стрелки~— кобордизмы между ними (последнюю
    ещё называют «category of~cobordisms»). В~русском языке нет устоявшихся переводов, поэтому
    во~избежание путаницы первое предлагаем переводить как «категорию с~кобордизмами», а~второе~—
    как «категорию кобордизмов».
} упорядоченную тройку $(C, \partial, i)$, где $C$~— категория, $\partial$~— функтор $C \rightarrow C$,
$i$~— естественное преобразование $\partial \Rightarrow 1_C$, такую, что:
\begin{enumerate}
    \item $\partial$~— аддитивный функтор.
    \item $\partial(\partial(x))$~— инициальный объект в~$C$ для~всех $x \in \Ob(C)$.
\end{enumerate}

От~определения, данного в~\cite{Stong68}, данное отличается тем, что не~требует существования а) малой подкатегории $C_0$
такой, что всякий объект из~$C$ изоморфен некоторому из~$C_0$ (это также эквивалентно тому, что скелет $C$~— малая категория),
и~б) всех копроизведений в~$C$.

\medskip
\noindent\textbf{Пример.} Рассмотрим категорию $\mathbf{Manifold}_p^n$ $n$-мерных $C^p$-гладких многообразий (с~границей),
где $1 \leq p \leq \infty$, стрелки в~которой~— $C^p$-отображения, переводящие границы в~границы (то~есть такие $f \in C^p(M, N)$,
что $f(\partial M) \subseteq \partial N$). Тогда функция $M \mapsto \partial M$ задаёт эндофунктор,
стрелочная функция которого~— просто ограничение на~границу: $f \mapsto f|_{\partial M}$.

Кроме того, поскольку $\partial M \subseteq M$, определены включения $C^p(\partial M, M)$,
которые (очевидно) также определяют естественное преобразование. Таким образом, задана категория с~кобордизмами.
\medskip

\noindent\textbf{Пример.} Рассмотрим категорию $\mathbf{Manifold}_p$ $C^p$-гладких многообразий произвольной размерности,
стрелки в~которой определены как в~предыдущем случае. Ясно, что эту категорию можно наделить структурой
категории с~кобордизмами точно таким~же образом.

Предыдущий пример входит в~эту категорию как подкатегория, но, в~отличие от~него, в~этой категории
существуют \textit{не~все} копроизведения, поскольку дизъюнктная сумма многообразий различной размерности
не~является снова многообразием.
\medskip

\noindent\textbf{Пример.} Зафиксируем пару категорий $J$ (называемую индексной) и~$C$. \textit{Коконусом} называется упорядоченная
тройка $(F, c, \psi)$, где $F : J \rightarrow C$~— функтор, $c \in \Ob(C)$~— объект
и~$\psi$~— семейство морфизмов $\psi_x \in \hom_C(F(x), c)$ такое, что для~любой
пары объектов $i, j \in \Ob(J)$ и~стрелки $f \in \hom_J(i, j)$ верно,
что $\psi_j \circ F(f) = \psi_i$:
\[
    \begin{tikzcd}
        &                               & \arrow{ld}{}[swap]{\psi_i} c \arrow{rd}{\psi_j} &       & \\
        & F(i) \arrow{rr}{}[swap]{F(f)} &                                                 & F(j). &
    \end{tikzcd}
\]

\textit{Копределом} называется такой коконус $(F, c, \psi)$, что для~всякого другого коконуса $(F, x, \phi)$
существует единственная стрелка $u \in \hom_C(c, x)$ такая, что для~любого $j \in \Ob(J)$
верно, что $u \circ \psi_j = \phi_j$:
\[
    \begin{tikzcd}
        F(j) \arrow{r}{\psi_j} \arrow{dr}{}[swap]{\phi_j} & \arrow{d}{u} c \\
                                                          & x.
    \end{tikzcd}
\]

Также пишем $c = \colim(F)$, $\psi = \incl(F)$ и~$u = \rec(F, x, \phi)$.

Определим структуру категории $\mathrm{Cocone}(J, C)$ на~всех коконусах между $J$ и~$C$.
\textit{Морфизмом коконусов} $(F, x, \phi)$ и~$(G, y, \psi)$ называется пара $(u, \varepsilon)$,
где $u \in \hom_C(x, y)$ и~$\varepsilon : F \Rightarrow G$, такая, что для~всякого $j \in \Ob(J)$
выполняется $u \circ \phi_j = \psi_j \circ \varepsilon_j$:
\[
    \begin{tikzcd}
        \arrow{d}{\varepsilon_j} F(j) \arrow{r}{\phi_j} & \arrow{d}{u} x \\
                                 G(j) \arrow{r}{\psi_j} &              y.
    \end{tikzcd}
\]

$(\id_C(x), 1_F)$ определяет тождественный морфизм для~коконуса $(F, x, \phi)$.

Далее, пусть заданы морфизм $(u, \varepsilon)$ между $(G, y, \psi)$ и~$(H, z, \xi)$ и~$(v, \eta)$ между $(F, x, \phi)$ и~$(G, y, \psi)$.
Композицию определим покомпонентно: $(u, \varepsilon) \circ (v, \eta) = (u \circ v, \varepsilon \circ \eta)$.
При~этом:
$$
    (u \circ v) \circ \phi_j = u \circ (v \circ \phi_j)
                             = u \circ (\psi_j \circ \eta_j)
                             = (u \circ \psi_j) \circ \eta_j
                             = (\xi_j \circ \varepsilon_j) \circ \eta_j
                             = \xi_j \circ (\varepsilon_j \circ \eta_j).
$$

Наконец, поскольку композиция в~категории $C$ и~вертикальная композиция функторов ассоциативны
и~унитальны, то~такова и~композиция морфизмов коконусов.

Пусть теперь категория $C$ содержит все копределы из~$J$ и~инициальный объект $0 \in \Ob(C)$.
Определим граничный эндофунктор следующим образом:
$$
\partial = \begin{cases}
             (F, x, \phi) \mapsto (\Delta_0, \colim(F), j \mapsto\ !_{\colim(F)}); \\
             (u, \varepsilon) \mapsto (\rec(F, \colim(G), j \mapsto \incl_j(G) \circ \varepsilon_j), 1_{\Delta_0}).
           \end{cases}
$$

$\partial(\id((F, x, \phi))) = \id(\partial((F, x, \phi)))$ верно в~силу того, что
\begin{align*}
    \rec(F, \colim(F), \incl(F)) = \id(\colim(F)).
\end{align*}
$\partial((u, \varepsilon) \circ (v, \eta)) = \partial((u, \varepsilon)) \circ \partial((v, \eta))$
верно в~силу того, что
\begin{align*}
    \phantom{=}&\ \rec(F, \colim(H), j \mapsto \incl_j(H) \circ \varepsilon_j \circ \eta_j) \\
              =&\ \rec(G, \colim(H), j \mapsto \incl_j(H) \circ \varepsilon_j) \circ \rec(F, \colim(G), j \mapsto \incl_j(G) \circ \eta_j).
\end{align*}

Также определим естественное преобразование $i : \partial \Rightarrow 1_C$:
$$
    i_{(F, x, \phi)} = (\rec(F, x, \phi), j \mapsto\ !_{F(j)}).
$$

Видно, что:
\begin{align*}
    \phantom{=}&\ i_{(G, y, \psi)} \circ \partial((u, \varepsilon)) \\
              =&\ (\rec(G, y, \psi), j \mapsto\ !_{G(j)}) \circ (\rec(F, \colim(G), j \mapsto \incl_j(G) \circ \varepsilon_j), 1_{\Delta_0}) \\
              =&\ (\rec(G, y, \psi) \circ \rec(F, \colim(G), j \mapsto \incl_j(G) \circ \varepsilon_j), j \mapsto\ !_{G(j)} \circ \id(\Delta_0(j))) \\
              =&\ (\rec(F, y, j \mapsto \psi_j \circ \varepsilon_j), j \mapsto\ !_{G(j)}) \\
              =&\ (\rec(F, y, j \mapsto u \circ \phi_j), j \mapsto\ !_{G(j)}) \\
              =&\ (u \circ \rec(F, x, \phi), j \mapsto \varepsilon_i\ \circ\ !_{F(j)}) \\
              =&\ (u, \varepsilon) \circ i_{(F, x, \phi)}.
\end{align*}

Аддитивность $\partial$ следует из~того, что копределы коммутируют с~копроизведением,
то~есть существует естественный изоморфизм $\colim(F + G) \cong \colim(F) + \colim(G)$,
и~что $\colim(\Delta_0) \cong 0$.

Наконец, $\partial(\partial((F, x, \phi)))) = (\Delta_0, \colim(\Delta_0), j \mapsto\ !_{\colim(\Delta_0)}) \cong 0$,
то~есть $(\mathrm{Cocone}(J, C), \partial, i)$~— категория с~кобордизмами.

Этот пример примечателен тем, что не~требует для~своего построения привлечения многообразий либо их обобщений.

\subsection{Кобордизмы и слабые кобордизмы}

\textit{Кобордизмом} между парой объектов $a$ и~$b \in \Ob(C)$ называется тройка $(c, i, j)$,
где $c \in \Ob(C)$, $i \in \hom_C(a, \partial(c))$ и~$j \in \hom_C(b, \partial(c))$,
такая, что $(\partial(c), i, j)$~— копроизведение. \textit{Морфизмом кобордизмов} $\epsilon_1 = (c_1, i_1, j_1)$
и~$\epsilon_2 = (c_2, i_2, j_2)$ между объектами $a$ и~$b$ называется стрелка $\phi \in \hom_C(c_1, c_2)$
такая, что $\partial(\phi) \circ i_1 = i_2$ и~$\partial(\phi) \circ j_1 = j_2$. Также говорим, что объекты \textit{кобордантны,}
если между ними существует кобордизм. Множество всех кобордизмов между объектами $a$ и~$b$ в~категории с~кобордизмами
$\Gamma = (C, \partial, i)$ будем обозначать как $\Cob_\Gamma(a, b)$.

$\Cob_\Gamma(a, b)$ вместе со~стрелками в~определённом выше смысле образует категорию,
изоморфизмы в~которой соответствует эквивалентности кобордизмов, определённой в~\cite{Mil65}.
Отметим существование из~неё забывающего функтора $|\cdot| : \Cob_\Gamma(a, b) \rightarrow C$:
$$
    | \cdot | = \begin{cases}
        (c, i, j) \mapsto c; \\
        \phi \mapsto \phi.
    \end{cases}
$$
Поэтому для~любой пары кобордизмов $\varepsilon_1, \varepsilon_2 \in \Cob_\Gamma(a, b)$
имеет место собственная категория $\Cob_\Gamma(|\varepsilon_1|, |\varepsilon_2|)$.

\begin{statement*}
    Во~всякой категории с~кобордизмами $(C, \partial, i)$:
    \begin{enumerate}
        \item Если $a$ и~$b$ кобордантны, то~$\partial(a) \cong \partial(b) \cong 0$.
        \item Для~любого $a$ объекты $\partial(a)$ и~$0$ кобордантны.
    \end{enumerate}
\end{statement*}

\begin{proof}
    Пусть $(c, i, j) \in \Cob_\Gamma(a, b)$. Тогда $\partial(c) \cong a + b$, поэтому:
    $$
        0 \cong \partial(\partial(c)) \cong \partial(a + b) \cong \partial(a) + \partial(b).
    $$
    Нетрудно показать, что если $x + y \cong 0$, то~$x \cong y \cong 0$.
    В~частности, $\partial(a) \cong \partial(b) \cong 0$.

    Далее, $\partial(a) \cong \partial(a) + 0$, поэтому само $a$ задаёт кобордизм между $a$ и~$0$.
\end{proof}

Рассмотрим две категории с~кобордизмами $\Gamma_1 = (C_1, \partial_1, i_1)$ и~$\Gamma_2 = (C_2, \partial_2, i_2)$.
Функтор $T : C_1 \rightarrow C_2$ называется \textit{граничным,} если $T \circ \partial_1 = \partial_2 \circ T$.

Говорим, что объект $a \in \Ob(C)$ \textit{замкнут,} если $\partial(a)$~— инициальный объект.
Объект $a$ \textit{точен,} если $\exists b \in \Ob(C){:}\ a = \partial(b)$.

\begin{statement*}
    Аддитивный граничный функтор $T$ переводит замкнутные объекты в~замкнутые, точные~— в~точные, а~кобордизмы~— в~кобордизмы.
\end{statement*}

\begin{proof}
    Действительно:
    \begin{enumerate}
        \item Пусть $a \in \Ob(C_1)$ замкнут. $\partial(a)$ инициален, а~потому, в~силу аддитивности,
        инициален и~$T(\partial(a)) = \partial(T(a))$.

        \item Пусть $a = \partial(b)$. Тогда $T(a) = T(\partial(b)) = \partial(T(b))$.

        \item Пусть $(c, i, j)$~— кобордизм между $a$ и~$b \in \Ob(C_1)$.
              В~силу того, что $T$~— граничный, $T(i) \in \hom_{C_2}(T(a), T(\partial(c))) = \hom_{C_2}(T(a), \partial(T(c)))$
              и~$T(j) \in \hom_{C_2}(T(b), T(\partial(c))) = \hom_{C_2}(T(b), \partial(T(c)))$.
              По~аддитивности, $(\partial(T(c)),\allowbreak T(i),\allowbreak T(j)) = (T(\partial(c)), T(i), T(j))$~— копроизведение,
              а~потому $(T(c),\allowbreak T(i),\allowbreak T(j))$~— кобордизм.
    \end{enumerate}
\end{proof}

Нетрудно видеть, что копроизведение симметрично (с~точностью до~изоморфизма).

\begin{statement*}
    Если $(c, i, j)$~— копроизведение $a$ и~$b$, то~$(c, j, i)$~— копроизведение $b$ и~$a$.
    В~частности, в~категории со~всеми копроизведениями, $a + b \cong b + a$.
\end{statement*}

Поэтому отношение кобордантности симметрично в~любой категории с~кобордизмами. Более того, отображение $(c, i, j) \mapsto (c, j, i)$
определяет изоморфизм между категориями $\Cob_\Gamma(a, b)$ и~$\Cob_\Gamma(b, a)$.
Однако не~для~всякой категории кобордантность транзитивна и~даже рефлексивна.

\medskip
\noindent\textbf{Пример.} Для~любой категории $C$ с~инициальным объектом $0$, очевидно, тройка $(C, \Delta_0, !)$
является категорией с~кобордизмами. Если категория $C$ не~содержит (бинарных) копроизведений,
то~все категории $\Cob(a, b)$ пусты; в~частности, пуста $\Cob(a, a)$.
Подойдёт, например, категория $C = \mathbf{Field}_p$ полей характеристики $p$.

\medskip

В~более простом случае, когда категория $C$ содержит все копроизведения, рассмотрим смежное понятие.
Под~\textit{слабым кобордизмом}\footnote{
    В~\cite{Stong68} для~этого понятия используется термин «кобордизм», поскольку, как мы далее
    увидим, в~(каноническом для~него) случае гладких многообразий существование слабого кобордизма
    логически эквивалентно существованию (определённого ранее) кобордизма. Однако чтобы избежать путаницы,
    мы добавляем прилагательное «слабый». Кроме того, структуры категории, которыми мы их оснастим, не~эквивалентны.
} между объектами $a$ и~$b$
понимаем тройку $(u, v, \phi)$, где $u, v \in \Ob(C)$ и~$\phi : a + \partial(u) \cong b + \partial(v)$.

Заметим, что $x \mapsto a + x$ определяет функтор. \textit{Морфизмом} слабых кобордизмов $(u_1, v_1, \phi_1)$
и~$(u_2, v_2, \phi_2)$ называем пару $(f, g)$, где $f \in \hom_C(u_1, u_2)$ и~$g \in \hom_C(v_1, v_2)$,
такую, что $\phi_2 \circ (a + \partial(f)) = (b + \partial(g)) \circ \phi_1$; то~есть такую, что следующая
диаграмма коммутативна:
\[
  \begin{tikzcd}
    \arrow{d}{\phi_1} a + \partial(u_1) \arrow{r}{a + \partial(f)} & a + \partial(u_2) \arrow{d}{\phi_2} \\
                      b + \partial(v_1) \arrow{r}{b + \partial(g)} & b + \partial(u_2).
  \end{tikzcd}
\]

С~единицей $\id((u, v, f)) = (\id(u), \id(v))$ слабые кобордизмы между $a$ и~$b$ образуют категорию, обозначаемую $\Cob^w(a, b)$.
Аналогично, объекты $a$ и~$b$ \textit{слабо кобордантны,} если $\Cob^w(a, b)$ непусто.

Тогда видим, что в~произвольной категории с~кобордизмами:
\begin{enumerate}
    \item Слабая кобордантность рефлексивна; более того, имеем включение из~$C$ в~$\Cob^w(a, a)$:
    $$
        R = \begin{cases}
            x \mapsto (x, x, \id(a + \partial(x))); \\
            f \mapsto (f, f).
        \end{cases}
    $$
    \item Слабая кобордантность симметрична; более того, определён изоморфизм между $\Cob^w(a, b)$ и~$\Cob^w(b, a)$:
    $$
        S = \begin{cases}
            (x, y, \phi) \mapsto (y, x, \phi^{-1}); \\
            (f, g) \mapsto (g, f).
        \end{cases}
    $$
    \item Слабая кобордантность транзитивна \cite{Weston}, поскольку если имеем $(u, v_1) \in \Cob^w(a, b)$ и~$(v_2, w) \in \Cob^w(b, c)$,
    то $(u + v_2, v_1 + w) \in \Cob^w(a, c)$:
    \begin{align*}
        a + \partial(u + v_2) & \cong a + \partial(u) + \partial(v_2) \\
                              & \cong b + \partial(v_1) + \partial(v_2) \\
                              & \cong \partial(v_1) + b + \partial(v_2) \\
                              & \cong \partial(v_1) + c + \partial(w) \\
                              & \cong c + \partial(v_1) + \partial(w) \\
                              & \cong c + \partial(v_1 + w).
    \end{align*}
\end{enumerate}

Наконец, в~случае гладких многообразий без~границы кобордантность эквивалентна слабой кобордантности.

\begin{statement*}[\cite{Weston}]
    Если $a$ и~$b$~— гладкие многообразия, то~$a$ кобордантно $b$ тогда и~только тогда,
    когда $a$ слабо кобордантно $b$.
\end{statement*}

\begin{proof}
    Пусть $I \in \mathbf{Manifold}_p$~— стандартный интервал $[0; 1]$.
    Как известно, если $x$~— многообразие без~границы,
    то~имеется диффеоморфизм $\partial(x \times I) \cong x + x$.

    Пусть $(c, i, j) \in \Cob(a, b)$. В~таком случае, $a + \partial c \cong a + a + b \cong b + \partial(a \times I)$,
    то~есть $(c, a \times I) \in \Cob^w(a, b)$.
    Обратно, пусть $(u, v, \phi) \in \Cob^w(a, b)$. Рассмотрим $c_1 = a \times I + u$ и~$c_2 = b \times I + v$.
    Тогда $\partial c_1 \cong a + a + \partial u $ и~$\partial c_2 \cong b + b + \partial v$.
    Поскольку $\phi : a + \partial u \cong b + \partial v$, можем склеить $c_1$ и~$c_2$ по~общей
    границе, получив $c$ такой, что $\partial c \cong a + b$.
\end{proof}

\pagebreak

\end{document}