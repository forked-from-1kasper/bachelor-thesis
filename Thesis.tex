\documentclass{article}

\usepackage[utf8]{inputenc}
\usepackage[T2A]{fontenc}
\usepackage[russian]{babel}
\usepackage[tmargin=1in,bmargin=1in,lmargin=1.25in,rmargin=1.25in]{geometry}

\usepackage[hidelinks,unicode]{hyperref}
\usepackage[threshold=0]{csquotes}
\usepackage{indentfirst}
\usepackage{enumerate}
\usepackage{totcount}
\usepackage{titlesec}
\usepackage{etoolbox}
\usepackage{amsmath}
\usepackage{amssymb}
\usepackage{amsthm}
\usepackage{hhline}
\usepackage{array}

\newtheorem*{consequence*}{Следствие}
\newtheorem*{theorem*}{Теорема}
\newtheorem*{lemma*}{Лемма}

\def\paddedtext#1#2{\leavevmode\hbox to#1{\hss#2\hss}\ignorespaces}
\patchcmd{\thebibliography}{\section*{\refname}}{}{}{}

\begin{document}

\titlelabel{\thetitle.\:}

\begin{titlepage}
  \centering

  {\LargeТема:\vspace{0.2cm}}\par
  {\largeТеория Морса и кобордизмов, применение к изучению полиэдров Вейля}
\end{titlepage}

\tableofcontents

\section{Введение}

Важнейшим инструментом в~комплексном анализе является понятие \textit{вычета} голоморфной функции.
Напомним основные сведения из~одномерной теории вычетов.

Множество функций, голоморфных в~открытом множестве $D \subseteq \mathbb{C}$ (или, более общо, в~$D \subseteq \mathbb{C}^n$), обозначим как $\mathrm{Hol}(D)$.

\begin{theorem*}
  Пусть $D \subseteq \mathbb{C}$~— область. Если $f \in \mathrm{Hol}(D)$, а~кусочно-гладкие кривые $\gamma_1$
  и~$\gamma_2 \subseteq D$ гомотопны в~$D$, то
  $$
    \int_{\gamma_1} f(z)dz = \int_{\gamma_2} f(z)dz.
  $$
\end{theorem*}

Рассмотрим область $D$ и~её подмножество $D^\prime = D \setminus \{ z_1, \ldots, z_k \} \subseteq D$, где $z_i \in D$.

Из~теоремы в~частности следует, что для~произвольной $f \in \mathrm{Hol}(D^\prime)$ и~её изолированной
особой точки $z_0 \in D \setminus D^\prime$ интегралы $\oint_{\Gamma_\epsilon} f(z)dz$ по~окружностям
$\Gamma_\epsilon = \{ z \in D \ | \ |z - z_0| = \epsilon \}$ не~зависят от~$\epsilon \in \mathbb{R}_{+}$
при~достаточно малом $\epsilon$ (конкретнее, таком, что в~круг $\{ z \in D \ | \ |z - z_0| < \epsilon \}$
не~попадают никакие другие особые точки $f$, кроме выбранной $z_0$).

В~таком случае \textit{вычетом} $f$ в~точке $z = z_0$ называют интеграл
$$
  \operatorname*{res}\limits_{z_0} f = \operatorname*{res}\limits_{z = z_0} f(z) = \lim\limits_{\epsilon \to 0} \frac{1}{2 \pi i} \oint_{\Gamma_\epsilon} f(z)dz = \frac{1}{2 \pi i} \oint_{\Gamma_{\epsilon_0}} f(z)dz,
$$
где $\epsilon_0$ достаточно мало в~описанном выше смысле.

Следующая теорема позволяет сводить вычисление интеграла от~голоморфной функции к~вычислению вычетов в~её особых точках,
что вместе со~сравнительной простотой вычисления обосновывает их практическое применение.
\begin{theorem*}[основная о вычетах]
  Пусть $D \subseteq \mathbb{C}$~— односвязная область, $\gamma \subseteq D$~— замкнутая кусочно-гладкая кривая,
  $z_i \in D$, $f \in \mathrm{Hol}(D^\prime)$, где $D^\prime = D \setminus \{ z_1, \ldots, z_k \}$.
  Тогда:
  $$
    \oint_{\gamma} f(z)dz = 2 \pi i \sum_{k = 1}^{n} \mathrm{Ind}_{\gamma}(z_k) \operatorname*{res}\limits_{z_k} f,
  $$
  где $\mathrm{Ind}_{\gamma}(z)$~— индекс точки $z$ относительно кривой $\gamma$.
\end{theorem*}

Для~построения описанных конструкций существенным фактом было то, что одномерные голоморфные функции допускают
существование \textit{изолированных} особых точек. Однако, как показывает следующая теорема \cite{ShaII}, в~многомерном случае
это допущение неверно.

\begin{theorem*}[о стирании компактных особенностей]
  Пусть $D \subseteq \mathbb{C}^n$~— область, причём $n > 1$, $\overline{K} \subseteq D$,
  $D \setminus K$ связно. Тогда всякая $f \in \mathrm{Hol}(D \setminus K)$
  голоморфно продолжается в~$D$.
\end{theorem*}

Таким образом, в~случае $n > 1$ невозможно вокруг особой точки голоморфной функции
выбрать малую сферу так, чтобы функция была голоморфна на~всей сфере (в~противном случае,
функцию можно было~бы продолжить внутрь сферы); а~потому интеграл по~такой сфере, в~общем случае, не~определён.

В~отличие от~многомерных голоморфных функций, голоморфные отображение, то~есть функции $\mathbb{C}^n \rightarrow \mathbb{C}^n$,
\textit{могут} иметь изолированные нули или особые точки. Этот факт приводит к~построению так называемого \textit{вычета Гротендика.}

\section{Вычет Гротендика}

Пусть $G$~— некоторая группа. Целочисленная степень элемента группы $g \in G$ определяется стандартным образом:
$$
  g^k =
  \begin{cases}
    g \cdot g^{k - 1}, & k > 0; \\
    1, & k = 0; \\
    g^{-1} \cdot g^{k + 1}, & k < 0.
  \end{cases}
$$

Для~векторов $z = (z_1, \ldots, z_k) \in G^k$ и~индекса $\alpha = (\alpha_1, \ldots, \alpha_k) \in \mathbb{Z}^k$
также обозначим $z^\alpha = z_1^{\alpha_1} \ldots z_k^{\alpha_k}$.

\textit{Многочленом Лорана} над~кольцом $K$ называется функция вида $z \mapsto \sum_{i \in I} a_i z^{\alpha_i}$ из~$(K^\times)^k$ в~$K$,
где $a_i \in K$ и~$I \subseteq \mathbb{Z}^k$~— конечный набор индексов.

Зафиксируем набор многочленов Лорана $g_1, \ldots, g_n$ над~$\mathbb{C}$.
Обозначим $g(z) = g_1(z) \ldots g_n(z)$ и~$\Gamma = \{ z \in (\mathbb{C}^\times)^n \ | \ g(z) = 0 \}$.

Также зафиксируем некоторое изолирование решение $z \in \Gamma$ системы $g_1(z) = \ldots = g_n(z) = 0$ и~его окрестность $U \subseteq (\mathbb{C}^\times)^n$.
Для~всякого $\epsilon \in \mathbb{R}_{+}^n$ определим $\gamma_{z, \epsilon} = \{ z \in (\mathbb{C}^\times)^n\ | \ \forall i{:}\ |g_i(z)| = \epsilon_i \}$.

Почти для~всех $\epsilon$ множество $\gamma_{z, \epsilon}$ представляет собой гладкое вещественное подмногообразие многообразия $U$.
Для~достаточно малых ненулевых $\epsilon$ многообразие $\gamma_{z, \epsilon}$ является компактным подмногообразием $U \setminus \Gamma$.
Определим на~нём ориентацию с~помощью дифференциальной формы $d(\mathrm{Arg}\ g_1) \wedge \ldots \wedge d(\mathrm{Arg}\ g_n)$.

\textit{Циклом Гротендика} $\gamma_z$ \cite{GelKho02} изолированного корня $z$ системы $g_1(z) = \ldots = g_n(z) = 0$ назовём многообразие $\gamma_{z, \epsilon}$ для~достаточно малого $\epsilon$.
Класс цикла в~$n$-мерной группе гомологий $U \setminus \Gamma$ не~зависит от~выбора $\epsilon$, но,~очевидно, зависит от~порядка уравнений $g_1(z) = 0, \ldots, g_n(z) = 0$.

Зафиксируем также ещё один многочлен Лорана $f$. \textit{Вычетом Гротендика} назовём следующий интеграл:
$$
  \operatorname*{res}\limits_{z} \frac{f}{g_1 \ldots g_n} = \frac{1}{(2 \pi i)^n} \int_{\gamma_z} \frac{f}{g_1 \ldots g_n}
$$

Существенную трудность при~вычислении такого вычета представляет устройство цикла $\gamma_z$.
В~случае $g_i(z) = z$ цикл $\gamma_z$ является просто $n$-мерный тором, и~вычисление вычета
сводится к~вычислению коэффициента в~ряде~Лорана, как и~одномерном случае; однако в~общем случае
нетривиальна даже топологическая структура $\gamma_z$.

\section{Кобордизмы и теория Морса}

\section{Алгебраические подгруппы тора}

\section{Торическое разложение цикла Гротендика}

\section{Заключение}

\pagebreak

\titleformat{\section}{\centering\normalfont\Large\bfseries}{}{0em}{}
\section{Приложение. Категории с кобордизмами}

?

\pagebreak

\section{Список использованных источников}

\begin{thebibliography}{9}

\bibitem{ShaII} B.~V.~Shabat, “Introduction to~complex analysis Part~II. Functions of~several variables.” (1992).

\bibitem{GelKho02} O.~A.~Gelfond, A.~G.~Khovanskii, “Toric geometry and Grothendieck residues”, Mosc.~Math.~J., 2:1~(2002), 99–112.

\end{thebibliography}

\end{document}