\documentclass{beamer}

\usepackage[utf8]{inputenc}
\usepackage[T2A]{fontenc}
\usepackage[russian]{babel}

\usepackage{amssymb}
\usepackage{amsmath}
\usepackage{svg}

\newcommand{\Z}{\mathbb{Z}}
\newcommand{\GL}{\mathrm{GL}}
\newcommand{\divides}{\mid}

\usetheme{Berlin}

\setbeamertemplate{caption}{\raggedright\insertcaption\par}

\title{Алгебраические подгруппы комплексного тора}
\author{Мишко Николай}
\institute{Сибирский федеральный университет\\Институт математики и фундаментальной информатики}
\date{\today}

\setbeamertemplate{headline}{}
\setbeamertemplate{footline}{}

\begin{document}

\begin{frame}
  \titlepage
\end{frame}

\begin{frame}
  \frametitle{Понятие алгебраической группы}

  Под~\textbf{алгебраическим многообразием} $H$ понимаем множество решений системы полиномиальных уравнений.

  Говорим, что $H$ является \textbf{алгебраической группой}, если на~$H$ определена структура группы.

  Известным примером являются эллиптические кривые.
  \begin{columns}[c]
    \begin{column}{0.3\textwidth}
      \begin{figure}
        \includesvg[width=1\textwidth]{Cubic.svg}
      \end{figure}
    \end{column}
  \end{columns}

  Рассмотрим подробнее алгебраические подгруппы $(\mathbb{C}^\times)^n$.
\end{frame}

\begin{frame}
  \frametitle{Теорема Артина}

  $G$~— группа, $K$~— поле, а~$K^{\times}$~— его мультипликативная группа.
  Гомоморфизм $f : G \rightarrow K^{\times}$ называют \textbf{характером.}

  Характеры $f_1, f_2, \ldots, f_n$ \textbf{линейно независимы,} если для~всех $\alpha_1, \alpha_2, \ldots, \alpha_n \in K$ выполнено:
  $$
      \alpha_1 f_1 + \alpha_2 f_2 + \ldots + \alpha_n f_n = 0 \Rightarrow \alpha_1 = \alpha_2 = \ldots = \alpha_n = 0.
  $$

  \begin{block}{Теорема (Артин)}
    Любые $n$ попарно различных характеров линейно независимы.
  \end{block}
\end{frame}

\begin{frame}
  \frametitle{Теорема Шмидта}

  Для~$\alpha = (\alpha_1, \ldots, \alpha_n) \in \Z^n$ и~$z = (z_1, \ldots, z_n) \in G^n$ пишем
  $$
    z^\alpha = z_1^{\alpha_1} \cdot z_2^{\alpha_2} \cdot \ldots \cdot z_n^{\alpha_n}
  $$

  Отображение $z \mapsto z^\alpha$ определяет характер $\chi_\alpha : (K^\times)^n \rightarrow K^\times$.

  Всякий многочлен распадается в~сумму ($I \subseteq \Z^n$ конечно):
  $$
    P = \sum_{i \in I} a_i \chi_i
  $$

  \begin{block}{Теорема (Шмидт)}
    Пусть $K$~— поле. Всякая алгебраическая подгруппа $H$ группы $(K^{\times})^n$ задаётся
    системой биномиальных уравнений.
  \end{block}
\end{frame}

\begin{frame}
  \frametitle{Мономиальные параметризации}

  Обозначим $\phi_\alpha(t) = (t^{\alpha_1}, t^{\alpha_2}, \ldots, t^{\alpha_n})$,
  где $\alpha_1, \ldots, \alpha_n \in \Z^k$.

  $$
    \phi_\alpha =
    \begin{pmatrix}
        t_1 \\
        t_2 \\
        \vdots \\
        t_k
    \end{pmatrix} \mapsto
    \begin{pmatrix}
        t_1^{\alpha_1^1} \cdot t_2^{\alpha_1^2} \cdot \ldots \cdot t_k^{\alpha_1^k} \\
        t_1^{\alpha_2^1} \cdot t_2^{\alpha_2^2} \cdot \ldots \cdot t_k^{\alpha_2^k} \\
        \vdots \\
        t_1^{\alpha_n^1} \cdot t_2^{\alpha_n^2} \cdot \ldots \cdot t_k^{\alpha_n^k}
    \end{pmatrix}
  $$

  $\alpha \mapsto \phi_\alpha$ определяет строгий функтор $\mathrm{Matr}(\Z) \rightarrow \mathrm{Grp}$
  в~полях нулевой характеристики.
\end{frame}

\begin{frame}
  \frametitle{Нормальная форма Смита}

  \begin{block}{Теорема (о существовании нормальной формы Смита)}
      Пусть $\alpha \in \Z^{m \times n}$~— матрица.
      Тогда существуют такие матрицы $\beta_1 \in \GL^m(\Z)$
      и~$\beta_2 \in \GL^n(\Z)$, что

      $$
      \exists \varepsilon_1, \ldots, \varepsilon_r \in \Z \setminus \{0\}{:} \
      \beta_1 \alpha \beta_2 =
      \begin{pmatrix}
        \varepsilon_1 & \ldots & 0             & 0      & \ldots & 0      \\
               \vdots & \ddots & \vdots        & \vdots & \ddots & \vdots \\
                    0 & \ldots & \varepsilon_r & 0      & \ldots & 0      \\
                    0 & \ldots & 0             & 0      & \ldots & 0      \\
               \vdots & \ddots & \vdots        & \vdots & \ddots & \vdots \\
                    0 & \ldots & 0             & 0      & \ldots & 0
      \end{pmatrix} = \varepsilon,
      $$
      причём $\varepsilon_1 \divides \varepsilon_2 \divides \ldots \divides \varepsilon_r$ и~$r = \mathrm{rank}(\alpha)$.
  \end{block}
\end{frame}

\begin{frame}
  \frametitle{Инъективность мономиальной параметризации}

  $\phi_\alpha$ распадается в~композицию: $\phi_{\beta_1^{-1}} \circ \phi_\varepsilon \circ \phi_{\beta_2^{-1}}$.

  $\phi_{\beta_1^{-1}}$ и~$\phi_{\beta_2^{-1}}$~— биекции, поэтому инъективность $\phi_\alpha$ определяется инъективностью $\phi_\varepsilon$.

  \begin{block}{Теорема}
    Пусть $\alpha \in \Z^{n \times k}$~— матрица.
    \begin{align*}
                     \ & \phi_\alpha : (\mathbb{C}^\times)^k \rightarrow (\mathbb{C}^\times)^n\ \text{инъективно} \\
      \Leftrightarrow\ & \varepsilon_1 = \varepsilon_r = \ldots = 1 \wedge r = k \\
      \Leftrightarrow\ & \alpha_i\ \text{порождают всю решётку}\ \Z^k
    \end{align*}
  \end{block}
\end{frame}

\begin{frame}
  \frametitle{Существование мономиальной параметризации}

  Биномиальные уравнения можно переписать в~виде $\ker(\phi_\beta)$.

  \begin{block}{Теорема}
    $\mathrm{Im}(\phi_\alpha)$~— алгебраическая подгруппа тора.
  \end{block}

  $\omega_K(n)$~— группа корней из~единицы степени $n$ над~полем $K$.

  Определим $\Pi(\beta) = |\omega_K(\varepsilon_1)| \cdot \ldots \cdot |\omega_K(\varepsilon_r)|$.

  \begin{block}{Теорема}
    Если $\ker(\phi_{\beta}) = \ker(\phi_{\beta'})$, то~$\Pi(\beta) = \Pi(\beta')$.
  \end{block}

  \begin{block}{Теорема}
    Если $\Pi(H) = 1$, то для~$H$ существует мономиальная параметризация.
  \end{block}
\end{frame}

\begin{frame}
  \frametitle{Существование мономиальной параметризации}

  Для~алгебраической подгруппы тора $H \subseteq (\mathbb{C}^\times)^n$ рассмотрим компоненту
  связности $H^\circ$, содержащую единицу.

  \begin{block}{Теорема}
    $H \subseteq (\mathbb{C}^\times)^n$ имеет $\Pi(H)$ компонент связности.
  \end{block}

  \begin{block}{Следствие}
    Для~$H \subseteq (\mathbb{C}^\times)^n$ существует мономиальная параметризация
    тогда и~только тогда, когда $H$ связно.
  \end{block}

  \begin{block}{Следствие}
    Для~$H^\circ \subseteq (\mathbb{C}^\times)^n$ существует мономиальная параметризация.
  \end{block}
\end{frame}

\begin{frame}
  \centering \Huge{Спасибо за внимание}
\end{frame}

\end{document}
