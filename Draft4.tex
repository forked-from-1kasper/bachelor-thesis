\documentclass{article}

\usepackage[utf8]{inputenc}
\usepackage[T2A]{fontenc}
\usepackage[russian]{babel}
\usepackage[tmargin=1in,bmargin=1in,lmargin=1.25in,rmargin=1.25in]{geometry}

\usepackage[hidelinks,unicode]{hyperref}
\usepackage[threshold=0]{csquotes}
\usepackage{indentfirst}
\usepackage{enumerate}
\usepackage{totcount}
\usepackage{titlesec}
\usepackage{etoolbox}
\usepackage{amsmath}
\usepackage{amssymb}
\usepackage{hhline}
\usepackage{array}

\usepackage{amsthm}

\newtheorem{statement}{Утверждение}
\newtheorem{consequence}{Следствие}
\newtheorem{theorem}{Теорема}
\newtheorem{lemma}{Лемма}

\newtheorem*{statement*}{Утверждение}
\newtheorem*{consequence*}{Следствие}
\newtheorem*{theorem*}{Теорема}
\newtheorem*{lemma*}{Лемма}

\newcommand{\divides}{\mid}

\newcommand{\N}{\mathbb{N}}
\newcommand{\Z}{\mathbb{Z}}
\newcommand{\Q}{\mathbb{Q}}
\newcommand{\R}{\mathbb{R}}

\newcommand{\primes}{\mathbb{P}}
\newcommand{\complex}{\mathbb{C}}
\newcommand{\quaternion}{\mathbb{H}}

\newcommand{\torus}{\complex^\times}

\newcommand{\image}{\mathrm{Im}}

\newcommand{\Hom}{\mathrm{Hom}}
\newcommand{\Hol}{\mathrm{Hol}}

\newcommand{\diag}{\mathrm{diag}}

\newcommand{\rank}{\mathrm{rank}}
\newcommand{\Span}{\mathrm{Span}}

\newcommand{\GL}{\mathrm{GL}}
\newcommand{\SNF}{\mathrm{SNF}}

\def\paddedtext#1#2{\leavevmode\hbox to#1{\hss#2\hss}\ignorespaces}
\patchcmd{\thebibliography}{\section*{\refname}}{}{}{}


\begin{document}

\begin{lemma*}
    Всякий голоморфный $f \in \Hom((\torus)^n, \torus)$ является мономом, то~есть:
    $$
        \exists \alpha \in \Z^n{:}\ \forall z{:}\ f(z) = z^\alpha.
    $$
\end{lemma*}

\begin{proof}

Обозначим вещественный тор как $T_\varepsilon = \{ z \in (\torus)^n\ |\ |z_1| = \varepsilon \wedge \ldots \wedge |z_n| = \varepsilon \}$.

Зафиксируем $f$ и~рассмотрим интегралы:
$$
    I_k(w) = \oint_{T_\varepsilon} \frac{f(z \cdot w)}{z^k} \frac{dz_1}{z_1} \wedge \ldots \wedge \frac{dz_n}{z_n},
$$
где $w \in (\torus)^n$ и~$k \in \Z^n$.

Сделав замену $z \mapsto z \cdot w = \xi$, получим:
$$
    I_k(w) = \oint_{T_\varepsilon} \frac{f(\xi)}{(\xi / w)^k} \frac{d(\xi_1 / w_1)}{\xi_1 / w_1} \wedge \ldots \wedge \frac{d(\xi_n / w_n)}{\xi_n / w_n}
           = w^k \oint_{T_\varepsilon} \frac{f(\xi)}{\xi} \frac{d\xi_1}{\xi_1} \wedge \ldots \wedge \frac{d\xi_n}{\xi_n}.
$$

С~другой стороны, $f$~— гомоморфизм, поэтому:
$$
    I_k(w) = f(w) \oint_{T_\varepsilon} \frac{f(z)}{z^k} \frac{dz_1}{z_1} \wedge \ldots \wedge \frac{dz_n}{z_n}.
$$

Таким образом, имеет место тождество:
$$
    (f(w) - w^k) \oint_{T_\varepsilon} \frac{f(z)}{z^k} \frac{dz_1}{z_1} \wedge \ldots \wedge \frac{dz_n}{z_n} = 0,
$$
говорящее о~том, что если для~$k \in \Z^n$ интеграл в~левой части не~равен нулю, то
$$
    \forall u \in (\torus)^n{:}\ f(u) = u^k.
$$

Но~такой интеграл выражает коэффициент Лорана функции $f$. $f$ голоморфна и~не~равна нулю
во~всём $(\torus)^n$, поэтому существует значение $k$, для~которого интеграл не~равен нулю.

\end{proof}

Ослабив условия на~функцию $f$, можно обобщить эту теорему на~случай произвольного поля $K$.
Для~этого для~каждого многочлена Лорана $f(z) = \sum_{i \in I} a_i z^i$ и~индекса $k \in \Z^n$ рассмотрим отображение
$$
    \mathrm{coeff}_k(f) =
    \begin{cases}
        a_k, & \text{если $k \in I$}; \\
        0,   & \text{иначе.}
    \end{cases}
$$

Тогда, очевидно, справедлива следующая лемма.
\begin{lemma*}
    Для~любых многочленов Лорана $f$ и~$g$ и~вектора $w \in K^n$ верно, что:
    \begin{enumerate}
        \item $\mathrm{coeff}_k(f + g) = \mathrm{coeff}_k(f) + \mathrm{coeff}_k(g)$.
        \item $\mathrm{coeff}_k(w f) = w \cdot \mathrm{coeff}_k(f)$.
        \item $\mathrm{coeff}_k(f \circ (z \mapsto wz)) = w^k \cdot \mathrm{coeff}_k(f)$.
    \end{enumerate}
\end{lemma*}

Рассмотрим теперь вместо голомфорных гомоморфизмом~— полиномиальные.
\begin{lemma*}
    Всякий многочлен Лорана $f \in \Hom((K^\times)^n, K^\times)$ является мономом.
\end{lemma*}

\begin{proof}
    Действительно, для~произвольного $w \in K^n$ в~силу того, что $f$~— гомоморфизм,
    выполняется равенство $f \circ (z \mapsto wz) = f(w) \cdot f$. Откуда:
    $$
        w^k \cdot \mathrm{coeff}_k(f) = \mathrm{coeff}_k(f \circ (z \mapsto wz)) = \mathrm{coeff}_k(f(w) \cdot f) = f(w) \cdot \mathrm{coeff}_k(f),
    $$
    а~значит $(f(w) - w^k) \cdot \mathrm{coeff}_k(f) = 0$.
    Поскольку $f$ нигде не~равен нулю, существует $k_0$ такой, что $\mathrm{coeff}_{k_0}(f) \neq 0$;
    но~тогда $f(w) - w^{k_0} = 0$ для~всех $w$.
\end{proof}

\end{document}