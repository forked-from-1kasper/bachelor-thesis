\documentclass{article}

\usepackage[utf8]{inputenc}
\usepackage[T2A]{fontenc}
\usepackage[russian]{babel}
\usepackage[tmargin=1in,bmargin=1in,lmargin=1.25in,rmargin=1.25in]{geometry}

\usepackage[hidelinks,unicode]{hyperref}
\usepackage[threshold=0]{csquotes}
\usepackage{indentfirst}
\usepackage{enumerate}
\usepackage{totcount}
\usepackage{titlesec}
\usepackage{etoolbox}
\usepackage{amsmath}
\usepackage{amssymb}
\usepackage{amsthm}
\usepackage{hhline}
\usepackage{array}

\usepackage{amsthm}

\newtheorem{statement}{Утверждение}
\newtheorem{consequence}{Следствие}
\newtheorem{theorem}{Теорема}
\newtheorem{lemma}{Лемма}

\newtheorem*{statement*}{Утверждение}
\newtheorem*{consequence*}{Следствие}
\newtheorem*{theorem*}{Теорема}
\newtheorem*{lemma*}{Лемма}

\newcommand{\divides}{\mid}

\newcommand{\N}{\mathbb{N}}
\newcommand{\Z}{\mathbb{Z}}
\newcommand{\Q}{\mathbb{Q}}
\newcommand{\R}{\mathbb{R}}

\newcommand{\primes}{\mathbb{P}}
\newcommand{\complex}{\mathbb{C}}
\newcommand{\quaternion}{\mathbb{H}}

\newcommand{\torus}{\complex^\times}

\newcommand{\image}{\mathrm{Im}}

\newcommand{\Hom}{\mathrm{Hom}}
\newcommand{\Hol}{\mathrm{Hol}}

\newcommand{\diag}{\mathrm{diag}}

\newcommand{\rank}{\mathrm{rank}}
\newcommand{\Span}{\mathrm{Span}}

\newcommand{\GL}{\mathrm{GL}}
\newcommand{\SNF}{\mathrm{SNF}}

\def\paddedtext#1#2{\leavevmode\hbox to#1{\hss#2\hss}\ignorespaces}
\patchcmd{\thebibliography}{\section*{\refname}}{}{}{}


\begin{document}

Для~поля $K$ и~$n$ векторов $\alpha_i \in \Z^k$ рассмотрим отображение $\phi_\alpha(t) = (t^{\alpha_1}, t^{\alpha_2}, \ldots, t^{\alpha_n})$,
где $t \in (K^\times)^k$, а~$\alpha$~— матрица со~строками $\alpha_i$. Поскольку $(t_1 \cdot t_2)^{\alpha_i} = t_1^{\alpha_i} \cdot t_2^{\alpha_i}$,
$\phi_\alpha$~— гомоморфизм.

Далее увидим, что в~случае $K = \complex$ отображение $\phi_\alpha$ задаёт \textit{параметризацию} некоторой алгебраической группы.

\begin{statement*}
    $\phi_{E} = 1_{(K^\times)^n}$, где $E$~— единичная матрица $n \times n$, $1_X$~— тождественная функция на~множестве $X$.
\end{statement*}

\begin{statement*}
    $
        \forall \alpha \in \Z^{k \times m}, \beta \in \Z^{m \times n}{:}\ \phi_\alpha \circ \phi_\beta = \phi_{\alpha \beta}
    $.
\end{statement*}

\begin{proof}
    Действительно, рассмотрим $t \in (K^\times)^n$. Тогда $i$-я компонента $\phi_\alpha(\phi_\beta(t))$
    равна $\phi_\beta^{\alpha_i}(t) = (t^{\beta_1})^{\alpha_i^1} \ldots (t^{\beta_m})^{\alpha_i^m}
                                    = t_1^{\beta_1^1 \alpha_i^1 + \ldots + \beta_m^1 \alpha_i^m}
                                      \ldots
                                      t_n^{\beta_1^n \alpha_i^1 + \ldots + \beta_m^n \alpha_i^m}
                                    = t^{(\alpha \beta)_i}$.
\end{proof}

Вместе эти два утверждения составляют условие на~функториальность. Более точно, рассмотрим категорию
$\mathrm{Matr}(R)$ матриц над~кольцом $R$, объектами которой являются натуральные числа, а~стрелками
между числами $m$ и~$n$~— матрицы $R^{n \times m}$, и~категорию $\mathrm{Tor}(K)$ торических групп над~полем $K$,
объектами которой являются группы $(K^\times)^k$, а~стрелками~— гомоморфизмы между ними.
Тогда определён функтор $\phi : \mathrm{Matr}(\Z) \rightarrow \mathrm{Tor}(K)$:
$$
    \phi = \begin{cases}
        k \mapsto (K^\times)^k; \\
        \alpha \mapsto \phi_\alpha.
    \end{cases}
$$

Как показывает следующая лемма, этот функтор~— строгий.

\begin{lemma*}
    $
        \forall \alpha, \beta \in \Z^{n \times k}{:} \ \phi_\alpha = \phi_\beta \Leftrightarrow \alpha = \beta
    $.
\end{lemma*}

\begin{proof}
    Импликация справа налево очевидна.

    Наоборот, пусть $\phi_\alpha = \phi_\beta$. Тогда для~всякого $1 \leq i \leq n$ верно, что
    $$
        (e^{\alpha_1^i}, \ldots, e^{\alpha_k^i}) = \phi_\alpha(e \cdot e_i) = \phi_\beta(e \cdot e_i) = (e^{\beta_1^i}, \ldots, e^{\beta_k^i}).
    $$

    Таким образом, $e^{\alpha_j^i} = e^{\beta_j^i}$. $\alpha_j^i$ и~$\beta_j^i$ целочисленные, поэтому $\alpha_j^i = \beta_j^i$.
\end{proof}

Будем пользоваться существованием для~целочисленных матриц \textit{нормальной формы Смита.}
Пусть $\diag^{m \times n}_r(x_1, \ldots, x_r)$~— диагональная матрица $\diag(x_1, \ldots, x_r)$,
дополненная (или обрезанная) справа снизу нулями до~матрицы размера $m \times n$.

\begin{theorem*}[о существовании нормальной формы Смита]
    $$
        \forall \alpha \in \Z^{m \times n}{:}\
        \exists \beta_1 \in \GL^m(\Z), \beta_2 \in \GL^n(\Z){:} \
        \exists \varepsilon_1, \ldots, \varepsilon_r \in \Z \setminus \{0\}{:} \
        \beta_1 \alpha \beta_2 = \diag^{m \times n}_r(\varepsilon_1, \ldots, \varepsilon_r),
    $$
    причём $\varepsilon_1 \divides \varepsilon_2 \divides \ldots \divides \varepsilon_r$ (так называемые инвариантные факторы).
\end{theorem*}

Поскольку числа $\varepsilon_1$, $\ldots$, $\varepsilon_r$ выбираются единственным образом с~точностью
до~обратимого элемента, то~есть в~случае $\Z$ с~точностью до~$\pm 1$, можем всегда выбрать их положительными.
Для~них матрицу $\diag^{m \times n}_r(\varepsilon_1, \ldots, \varepsilon_r)$ обозначим как $\SNF(\alpha)$.

Помимо этого, с~учётом функториальности $\phi$, ясно, что $\phi_\beta$ биективно, если $\beta \in \GL^n(\Z)$.
Действительно, $\phi_\beta \circ \phi_{\beta^{-1}} = \phi_{\beta \beta^{-1}} = \phi_E = 1_{(K^\times)^n}$
и~$\phi_{\beta^{-1}} \circ \phi_\beta = \phi_{\beta^{-1} \beta} = \phi_E = 1_{(K^\times)^n}$.
Поэтому имеет место следующая лемма.

\begin{lemma*}
    $\phi_\alpha : (\complex^\times)^k \rightarrow (\complex^\times)^n$ инъективна тогда и~только тогда, когда
    $$
        \SNF(\alpha) = \diag^{n \times k}_k(1, 1, \ldots, 1).
    $$
\end{lemma*}

\begin{proof}
    Пользуясь теоремой, возьмём унимодулярные матрицы $\beta_1 \in \GL^n(\Z)$ и~$\beta_2 \in \GL^k(\Z)$
    такие, что $\beta_1 \alpha \beta_2 = \SNF(\alpha)$.
    В~таком случае, $\alpha = \beta_1^{-1} \SNF(\alpha) \beta_2^{-1}$, и~$\phi_\alpha = \phi_{\beta_1^{-1} \SNF(\alpha) \beta_2^{-1}}
                                                                                      = \phi_{\beta_1^{-1}} \circ \phi_{\SNF(\alpha)} \circ \phi_{\beta_2^{-1}}$.

    Поскольку, по~замечанию выше, $\phi_{\beta_1^{-1}}$ и~$\phi_{\beta_2^{-1}}$ биективны, то~$\phi_\alpha$
    инъективно тогда и~только тогда, когда инъективно $\phi_{\mathrm{SNF(\alpha)}}$.

    $\phi_{\mathrm{SNF(\alpha)}}(t_1, \ldots, t_k) = (t_1^{\varepsilon_1}, \ldots, t_r^{\varepsilon_r}, 1, \ldots, 1)$,
    но~$t \mapsto t^\varepsilon$ инъективно над~$\complex^\times$ только в~случае $\varepsilon = \pm 1$
    (иначе $t^\varepsilon = t^\varepsilon \cdot 1 = t^\varepsilon \cdot u^\varepsilon = (tu)^\varepsilon$,
     где~$u \neq 1$~— нетривиальный корень из~единицы степени $\varepsilon$).
    В~силу выбора знаков, $\phi_{\mathrm{SNF(\alpha)}}$ инъективна лишь когда $\varepsilon_1 = \ldots = \varepsilon_r = 1$
    и~$r = k \leq n$, что и~требовалось.
\end{proof}

Сформулируем ещё одну вспомогательную теорему.
Обозначим \textit{$\Z$-линейную оболочку} векторов $\alpha_1, \ldots, \alpha_n$ как
$\Span_\Z(\alpha_1, \ldots, \alpha_n) = \{ k_1 \alpha_1 + \ldots + k_n \alpha_n \ | \ k_1, \ldots, k_n \in \Z \}$.

Говорим, что $\alpha_1, \ldots, \alpha_n$ \textit{порождают всю решётку} $\Z^k$, если $\Span_\Z(\alpha_1, \ldots, \alpha_n) = \Z^k$.

\begin{theorem*}
    Пусть $\alpha \in \Z^{n \times k}$. Следующие утверждения равносильны:
    \begin{enumerate}
        \item Строки матрицы $\alpha$ порождают всю решётку $\Z^k$.
        \item Миноры максимальной размерности матрицы $\alpha$ взаимно просты.
        \item $\SNF(\alpha) = \diag^{n \times k}_k(1, \ldots, 1)$.
    \end{enumerate}
\end{theorem*}

Таким образом, получаем необходимое и~достаточное условие инъективности $\phi_\alpha$.

\begin{theorem*}
    $\phi_\alpha$ инъективно над~полем $K = \complex$ тогда и~только тогда, когда~$\alpha_i$ порождают всю решётку $\Z^k$.
\end{theorem*}

Поскольку $\phi_\alpha$~— гомоморфизм, образ $\image(\phi_\alpha)$ является подгруппой в~$(K^\times)^n$.
Изучим два вопроса: является~ли $\image(\phi_\alpha)$ алгебраической подгруппой и~всякая~ли алгебраическая подгруппа задаётся некоторым $\phi_\alpha$.

Для~этого, прежде всего, заметим, что система биномиальных уравнений $z^{\beta'_i} = z^{\beta''^i}$ эквивалентна
системе $z^{\beta'_i - \beta''_i} = 1$, то~есть в~точности ядру оператора $\phi_{\beta'_i - \beta''_i}$;
поэтому первый вопрос эквивалентен тому, можно~ли данный гомоморфизм $\phi_\alpha$ достроить
до~точной последовательности $(K^\times)^k \xrightarrow[]{\phi_\alpha} (K^\times)^n \xrightarrow[]{\phi_\beta} (K^\times)^m$.

\begin{lemma*}
    Пусть $G$~— группа, $H$~— абелева группа, $f \in \Hom(G, H)$, $g \in \Hom(H, G)$ и~$g \circ f = 1_G$.
    Тогда последовательность $G \xrightarrow[]{f} H \xrightarrow[]{f \circ g - 1_H} H$ точна.
\end{lemma*}

\begin{proof}
    Действительно, $(f \circ g - 1_H) \circ f = f \circ g \circ f - f = f - f = 0$, то~есть $\image(f) \subseteq \ker(f \circ g - 1_H)$. 
    Обратно, пусть $h \in \ker(f \circ g - 1_H)$. Тогда $f(g(h)) - h = 0 \Leftrightarrow h = f(g(h))$, то~есть $h \in \image(f)$,
    и~$\ker(f \circ g - 1_H) \subseteq \image(f)$.
\end{proof}

\begin{theorem*}
    Всякая параметризация $\phi_\alpha$, где $\alpha \in \Z^{n \times k}$, задаёт алгебраическую подгруппу $(K^\times)^n$.
\end{theorem*}

\begin{proof}
    Снова представим $\alpha$ в~виде $\alpha = \beta_1^{-1} \varepsilon \beta_2^{-1}$, где
    $$
        \varepsilon = \SNF(\alpha) = \diag^{n \times k}_r(\varepsilon_1, \ldots, \varepsilon_r).
    $$

    Рассмотрим $\delta = \diag^{n \times r}_r(1, \ldots, 1)$ и~$\alpha' = \beta^{-1} \delta$.
    $\phi_{\alpha'}$ инъективна как композиция инъективных функций. С~другой стороны, очевидно, что:
    $$
        \image(\phi_\alpha) = \phi_{\beta_1^{-1}} (\phi_\varepsilon (\phi_{\beta_2^{-1}} ((\complex^\times)^k)))
                            = \phi_{\beta_1^{-1}} (\phi_\varepsilon ((\complex^\times)^k))
                            = \phi_{\beta_1^{-1}} (\phi_{\delta} ((\complex^\times)^r))
                            = \image(\phi_{\alpha'}).
    $$

    Нетрудно видеть, что $\alpha'$ имеет левой обратной матрицу $\beta' = \delta^{\top} \beta$,
    а~потому гомоморфизм $\phi_{\alpha'}$ имеет левым обратным гомоморфизм $\phi_{\beta'}$.
    Поскольку $(\phi_{\alpha'} \circ \phi_{\beta'}) \phi_{-E} = \phi_{\alpha' \beta' - E}$, остаётся лишь применить лемму.
\end{proof}

Ответ на~второй вопрос несколько сложнее. Например, уравнение $z^n = 1$ задаёт алгебраическую
группу для~любого $n \in \Z$, состоящую из~$n$ точек на~$\complex^\times$; но~если $n \neq 0, \pm 1$,
то~легко видеть, что она не~может быть параметризована никакой $\phi_\alpha$.

Действительно, $\phi_\alpha$~— голоморфное отображение, $(\complex^\times)^k$~— открытое множество,
поэтому всякая проекция полного образа $\image(\phi_\alpha) = \phi_\alpha((\complex^\times)^k)$ либо, в~силу
открытости непостоянных голоморфных отображений, открыта, либо состоит из~одной точки; но~$z^n = 1$
открыто, только если $n = 0$, и~состоит из~одной точки, только если $n = \pm 1$.

Более того, из~этих рассуждений видно, что~по~всякой системе биномиальных уравнений можно построить новую систему,
которая гарантированно не~параметризуется целиком, возведя, например, произвольное уравнение в~степень $n > 1$.

Чтобы понять, при~каких условиях существует параметризация, сначала заметим, что группа $\image(\phi_\alpha)$
изоморфна группе $(K^\times)^r$. Действительно, в~доказательстве теоремы мы видели, что для~всякого $\phi_\alpha$
можно построить инъективный $\phi_{\alpha'} : (K^\times)^r \rightarrow (K^\times)^n$ такой, что $\image(\phi_\alpha) = \image(\phi_{\alpha'})$;
но~тогда $\phi_{\alpha'}$ и~задаёт искомый изоморфизм. Таким образом, верно следующее утверждение.

\begin{statement*}
    Для~любой матрицы $\alpha \in \Z^{n \times k}$ группа $\image(\phi_\alpha)$ изоморфна алгебраическому тору размерности не~более чем $k$.
\end{statement*}

Кроме того, поскольку отображение $\phi_{\alpha'}$ полиномиально, оно, в~частности, непрерывно в~естественной топологии на~$K = \R$ и~$K = \complex$.
Множество $(\complex^\times)^r$ связно (причём использование поля $K = \complex$ тут существенно, так как $\R^\times$ содержит две компоненты связности),
а, как известно, непрерывный образ связного множества связен, в~силу чего верно ещё одно утверждение.

\begin{statement*}
    $\image(\phi_\alpha) \subseteq (\complex^\times)^n$ связно.
\end{statement*}

Теперь отметим, что группа $z^n = 1$ связна в~точности тогда и~только тогда, когда $n = 0, \pm 1$.
Это соображение указывает на~то, что критерием существования параметризации для~алгебраической
подгруппы тора является связность.

Чтобы доказать это, рассмотрим число $\Pi(\alpha) = \varepsilon_1 \cdot \ldots \cdot \varepsilon_r$, равное произведению инвариантных факторов матрицы $\alpha$.

\begin{lemma*}
    Если $\Pi(\beta) = 1$, то~существует такое $\alpha$, что $\ker(\phi_\beta) = \image(\phi_\alpha)$.
\end{lemma*}

\begin{proof}
    Пусть $H = \ker(\phi_\beta)$~— алгебраическая группа. Без~потери общности считаем,
    что строки матрицы $\beta$ линейно независимы над~$\Z$ (ясно, что строки, выражающиеся
    как линейная комбинация остальных строк, можно убрать из~матрицы $\beta$ без~изменения ядра).

    Запишем для~$\beta$ нормальную форму Смита: $\beta = \beta_1^{-1} \varepsilon \beta_2^{-1}$.
    Поскольку, по~замечанию выше, $\beta$ имеет полный ранг, матрица $\varepsilon$ не~имеет нулевых строк.
    Кроме того, $\Pi(\beta) = 1$, поэтому $\varepsilon = \diag^{k \times n}_k(1, \ldots, 1)$.

    Ядро $\varepsilon$ задаётся векторами вида $(0, \ldots, 0, t_{k + 1}, \ldots, t_{n})$.
    Рассмотрим матрицу $\delta \in \Z^{n \times (n - k)}$, соответствующую линейному оператору
    $$
        (t_1, \ldots, t_{n - k}) \mapsto (0, \ldots, 0, t_1, \ldots, t_{n - k}).
    $$

    Пусть также $\alpha = \beta_2 \delta$. Докажем, что $\image(\phi_\alpha) = \ker(\phi_\beta)$.

    Действительно, по~заданию $\delta$, $\varepsilon \delta = 0$, поэтому 
    $\phi_{\beta} \circ \phi_{\alpha} = \phi_{\beta_1^{-1} \varepsilon \beta_2^{-1} \beta_2 \delta} = \phi_{\beta_1^{-1} \varepsilon \delta} = \phi_{0} = 1$,
    то~есть $\image(\phi_{\alpha}) \subseteq \ker(\phi_{\beta})$.

    Обратно, пусть $z \in \ker(\phi_\beta)$. Тогда $\phi_{\beta_1^{-1}} (\phi_{\varepsilon \beta_2^{-1}}(z)) = \phi_\beta(z) = 1$,
    откуда $\phi_{\varepsilon}(\phi_{\beta_2^{-1}}(z)) = \phi_{\varepsilon \beta_2^{-1}}(z) = \phi_{\beta_1}(1) = 1$.
    Из~вида матрицы $\varepsilon$ получаем, что $\phi_{\beta_2^{-1}}(z) = (1, \ldots, 1, z_{k + 1}, \ldots, z_n) = \phi_\delta(z_{k + 1}, \ldots, z_n)$;
    то~есть $z = \phi_{\beta_2}(\phi_\delta(z_{k + 1}, \ldots, z_n)) = \phi_\alpha(z_{k + 1}, \ldots, z_n)$.
    Таким образом, $z \in \image(\phi_\alpha)$, и~$\ker(\phi_\beta) \subseteq \image(\phi_\alpha)$.
\end{proof}

Группу корней степени $n$ из~единицы, задаваемую уравнением $z^n = 1$, обозначим как $\omega(n) \subseteq \torus$.
Для~вектора $x \in \Z^k$ также обозначим $\omega(x) = \omega(x_1) \times \ldots \times \omega(x_k) \subseteq (\torus)^k$.
Известно, что группа $\omega(x)$ изоморфна группе $\Z / x_1 \Z \times \ldots \times \Z / x_k \Z$.

\begin{theorem*}
    Число связных компонент $\ker(\phi_\beta) \subseteq (\torus)^n$ равно $\Pi(\beta)$.
\end{theorem*}

\begin{proof}
    Снова рассмотрим нормальную форму Смита для~$\beta$: $\beta$ = $\beta_1^{-1} \varepsilon \beta_2^{-1}$.
    $\phi_{\beta_1^{-1}}$~— изоморфизм, поэтому $\ker(\phi_{\beta}) = \ker(\phi_{\varepsilon \beta_2^{-1}})$.

    Строки матрицы $\beta_2^{-1}$ обозначим как $b_i$.
    Для~вектора $u \in \omega(\varepsilon)$ рассмотрим множество $H_u = \{z \in (\torus)^n \ | \ \forall 1 \leq i \leq r{:}\ z^{b_i} = u_i\}$.
    Условие $\phi_\varepsilon(\phi_{\beta_2^{-1}}(z)) = 1$, очевидно, эквивалентно условию $\exists u \in \omega(\varepsilon){:}\ z \in H_u$.
    Ясно, что множества $H_u$ дизъюнктны, поэтому $\ker(\phi_{\beta})$ распадается в~дизъюнктное объединение:
    $$
        \ker(\phi_{\beta}) = \bigsqcup_{u \in \omega(\varepsilon)} H_u.
    $$

    Векторов $u \in \omega(\varepsilon)$ ровно $\Pi(\beta)$ штук, поэтому достаточно показать, что каждая компонента $H_u$ связна.

    Поскольку, по~определению, $H_1 = \ker(\phi_{\delta \beta_2^{-1}})$, где $\delta = \diag^{k \times n}_r(1, \ldots, 1)$,
    и~$\Pi(\delta \beta_2^{-1}) = 1$, компонента $H_1$, по~замечанию выше, связна.

    Зафиксировав $u$, рассмотрим матрицу $\tau = \diag(1 / u_1, \ldots, 1 / u_r, 1, \ldots, 1)$ и~отображение $\psi = \phi_{\beta_2} \circ \phi_\tau \circ \phi_{\beta_2^{-1}}$.
    $\psi$~— непрерывная биекция, причём $\psi^{-1} = \phi_{\beta_2^{-1}} \circ \phi_{\tau^{-1}} \circ \phi_{\beta_2}$.

    По~определению, $\phi_{\beta_2^{-1}}(\psi(z)) = \phi_{\tau \beta_2^{-1}}(z)$.
    Поэтому если $z \in H_u$, то~$\psi(z)^{b_i} = z^{b_i} / u_i = u_i / u_i = 1$. Обратно, если $z \in H_1$,
    то~$\psi^{-1}(z)^{b_i} = u_i z^{b_i} = u_i$. Таким образом, $\psi(H_u) = H_1$,
    то~есть $H_u$ гомеоморфно $H_1$, но~$H_1$ связно.
\end{proof}

Как мы видели ранее, всякая алгебраическая подгруппа $H$ тора $(K^\times)^n$ представляется как $\ker(\phi_\beta)$
для~некоторого $\beta$. Заметим, что $\Pi(\beta)$ не~зависит от~выбора $\beta$ для~группы $H$.

\begin{theorem*}
    Если $\ker(\phi_{\beta}) = \ker(\phi_{\beta'})$, то~$\Pi(\beta) = \Pi(\beta')$.
\end{theorem*}

\begin{proof}
    Запишем нормальные формы Смита: $\beta = \beta_1^{-1} \varepsilon \beta_2^{-1}$
    и~$\beta' = {\beta'}_1^{-1} \varepsilon' {\beta'}_2^{-1}$.
    Тогда $\ker(\phi_{\varepsilon \beta_2^{-1}}) = \ker(\phi_{\beta}) = \ker(\phi_{\beta'}) = \ker(\phi_{\varepsilon' {\beta'}_2^{-1}})$.

    Пусть $\varepsilon = \diag^{k \times n}_r(\varepsilon_1, \ldots, \varepsilon_r)$
    и~$\varepsilon' = \diag^{k' \times n}_{r'}(\varepsilon'_1, \ldots, \varepsilon'_r)$.
    Обозначим $\delta = \diag^{k \times n}_r(1, \ldots, 1)$ и~$\delta' = \diag^{k' \times n}_{r'}(1, \ldots, 1)$.

    Поскольку $\Pi(\delta \beta_2^{-1}) = \Pi(\delta' {\beta'}_2^{-1}) = 1$, согласно предыдущей теореме,
    обе подгруппы параметризуются некоторыми $\phi_\alpha$ и~$\phi_{\alpha'}$ соответственно.

    $\image(\phi_\alpha) = \ker(\phi_{\delta \beta_2^{-1}}) \subseteq \ker(\phi_{\varepsilon \beta_2^{-1}}) = \ker(\phi_{\varepsilon' {\beta'}_2^{-1}})$,
    поэтому $\phi_{\varepsilon' {\beta'}_2^{-1}} \circ \phi_\alpha = 1$.
    В~силу функториальности, $\varepsilon' {\beta'}_2^{-1} \alpha = 0$.

    Умножив обе части равенства на~матрицу
    $\diag(1 / \varepsilon'_1, \ldots, 1 / \varepsilon'_r, 1, \ldots, 1)$, получим, что $\delta' {\beta'}_2^{-1} \alpha = 0$;
    но~это значит, что $\ker(\phi_{\delta \beta_2^{-1}}) = \image(\phi_\alpha) \subseteq \ker(\phi_{\delta' {\beta'}_2^{-1}})$.
    Совершенно аналогично получим обратное включение. Таким образом, $\ker(\phi_{\delta \beta_2^{-1}}) = \ker(\phi_{\delta' {\beta'}_2^{-1}})$.

    Рассмотрим фактор-группу $\ker(\phi_\beta) / \ker(\phi_{\delta \beta_2^{-1}}) = \ker(\phi_{\beta'}) / \ker(\phi_{\delta' {\beta'}_2^{-1}})$,
    называемую также \textit{группой компонент.}
    Отображение $\phi_{\delta \beta_2^{-1}} : \ker(\phi_\beta) \rightarrow \omega(\varepsilon)$
    индуцирует инъективный гомоморфизм из~фактора. Помимо этого, в~теореме выше была построена биекция между компонентами $H_u$ (которая обобщается
    без~изменений на~случай произвольного поля $K$), а~потому все они непусты.
    Это означает, что $\phi_{\delta \beta_2^{-1}}$ сюръективно, а~потому индуцированное отображение тоже сюръективно.
    Аналогично проводятся рассуждения для~группы $\omega(\varepsilon')$.
    Таким образом, получаем цепочку изоморфизмов:
    \begin{align*}
        \Z / \varepsilon_1 \Z \times \ldots \times \Z / \varepsilon_r \Z & \cong \omega(\varepsilon) \\
                                                                         & \cong \ker(\phi_\beta) / \ker(\phi_{\delta \beta_2^{-1}}) \\
                                                                         &     = \ker(\phi_{\beta'}) / \ker(\phi_{\delta' {\beta'}_2^{-1}}) \\
                                                                         & \cong \omega(\varepsilon') \\
                                                                         & \cong \Z / \varepsilon'_1 \Z \times \ldots \times \Z / \varepsilon'_{r'} \Z.
    \end{align*}

    Изоморфные группы имеют одинаковые порядки, а~значит
    $$
        \Pi(\beta) = \varepsilon_1 \cdot \ldots \cdot \varepsilon_r = \varepsilon'_1 \cdot \ldots \cdot \varepsilon'_{r'} = \Pi(\beta').
    $$
\end{proof}

Таким образом, для~всякой алгебраической подгруппы $H$ тора можно определить число $\Pi(H)$
равное $\Pi(\beta)$ для~всякого $\beta$ такого, что $H = \ker(\phi_\beta)$.

Теперь для~всякой алгебраической подгруппы $H \subseteq (\complex^\times)^n$ определим \textit{компоненту единицы} $H^\circ$
как компоненту связности, содержащую нейтральный элемент группы. Из~доказанной теоремы следует, что $\Pi(H^\circ) = 1$.
Таким образом, доказаны следующие утверждения.

\begin{theorem*}
    Алгебраическая подгруппа $H$ тора $(K^\times)^n$ параметризуема тогда и~только тогда, когда $\Pi(H) = 1$.
\end{theorem*}

\begin{consequence*}
    Алгебраическая подгруппа $H$ тора $(\complex^\times)^n$ параметризуема тогда и~только тогда, когда является связной.
\end{consequence*}

\begin{consequence*}
    Всякая алгебраическая подгруппа $H$ тора $(\complex^\times)^n$ содержит параметризуемую подгруппу $H^\circ$ той~же размерности.
\end{consequence*}

\begin{proof}
    Все связные компоненты $H$ гомеоморфны, что немедленно доказывает теорему.
\end{proof}

\end{document}