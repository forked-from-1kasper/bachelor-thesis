\documentclass{article}

\usepackage[utf8]{inputenc}
\usepackage[T2A]{fontenc}
\usepackage[russian]{babel}
\usepackage[tmargin=1in,bmargin=1in,lmargin=1.25in,rmargin=1.25in]{geometry}

\usepackage[hidelinks,unicode]{hyperref}
\usepackage[threshold=0]{csquotes}
\usepackage{indentfirst}
\usepackage{enumerate}
\usepackage{totcount}
\usepackage{titlesec}
\usepackage{etoolbox}
\usepackage{amsmath}
\usepackage{amssymb}
\usepackage{amsthm}
\usepackage{hhline}
\usepackage{array}

\usepackage{amsthm}

\newtheorem{statement}{Утверждение}
\newtheorem{consequence}{Следствие}
\newtheorem{theorem}{Теорема}
\newtheorem{lemma}{Лемма}

\newtheorem*{statement*}{Утверждение}
\newtheorem*{consequence*}{Следствие}
\newtheorem*{theorem*}{Теорема}
\newtheorem*{lemma*}{Лемма}

\newcommand{\divides}{\mid}

\newcommand{\N}{\mathbb{N}}
\newcommand{\Z}{\mathbb{Z}}
\newcommand{\Q}{\mathbb{Q}}
\newcommand{\R}{\mathbb{R}}

\newcommand{\primes}{\mathbb{P}}
\newcommand{\complex}{\mathbb{C}}
\newcommand{\quaternion}{\mathbb{H}}

\newcommand{\torus}{\complex^\times}

\newcommand{\image}{\mathrm{Im}}

\newcommand{\Hom}{\mathrm{Hom}}
\newcommand{\Hol}{\mathrm{Hol}}

\newcommand{\diag}{\mathrm{diag}}

\newcommand{\rank}{\mathrm{rank}}
\newcommand{\Span}{\mathrm{Span}}

\newcommand{\GL}{\mathrm{GL}}
\newcommand{\SNF}{\mathrm{SNF}}

\def\paddedtext#1#2{\leavevmode\hbox to#1{\hss#2\hss}\ignorespaces}
\patchcmd{\thebibliography}{\section*{\refname}}{}{}{}

\newcolumntype{L}[1]{>{\raggedright\let\newline\\\arraybackslash\hspace{0pt}}m{#1}}
\newcolumntype{C}[1]{>{\centering\let\newline\\\arraybackslash\hspace{0pt}}m{#1}}
\newcolumntype{R}[1]{>{\raggedleft\let\newline\\\arraybackslash\hspace{0pt}}m{#1}}

\newcommand{\usection}[1]{\phantomsection\section*{\centering#1}
\addcontentsline{toc}{section}{\protect\numberline{}#1}}

\newcommand{\usubsection}[1]{\phantomsection\subsection*{\centering#1}
\addcontentsline{toc}{subsection}{\protect\numberline{}#1}}

\newcommand{\usubsubsection}[1]{\phantomsection\subsubsection*{\centering#1}
\addcontentsline{toc}{subsubsection}{\protect\numberline{}#1}}


\begin{document}

Зафиксируем векторное пространство $V$. Структура векторного пространства индуцирует
на~$V$ структуру метрического пространства. Пользуясь этим, \textit{решёткой} назовём
всякую дискретную подгруппу аддитивной группы $V$.
В~частности, подгруппы $\Z^n$ представляют собой решётки над~$\R^n$.

Ясно, что для~произвольной решётки $L$ и~вектора $u \in L$ верно, что $zu \in L$ для~всякого $z \in \Z$.
В~решётке $L$ векторы $u_1, \ldots, u_k \in L$ назовём линейно независимыми, если:
$$
    \forall \alpha_1, \ldots, \alpha_k \in \Z{:}\ \alpha_1 u_1 + \ldots + \alpha_k u_k = 0 \Rightarrow \alpha_1 = \ldots = \alpha_k = 0.
$$

\textit{Размерностью} решётки $L$ назовём максимальное число линейно независимых векторов $n$, обозначая $\dim_{\Z}(L) = n$,
если таковое существует, и~$\infty$ в~противном случае.

Пусть $\mathrm{Lin}_{\subseteq}(V)$~— множество всех линейных подпространств пространства $V$.
Как известно, $\mathrm{Lin}_{\subseteq}(V)$ замкнуто относительно произвольных пересечений.
Всякой решётке $L$ пространства $V$ сопоставим векторное подпространство $L[\R] = \bigcap\limits_{L \subseteq U \in \mathrm{Lin}_{\subseteq}(V)} U$.

\begin{lemma*}
    Пусть $V$~— векторное пространство, $L$~— решётка над~ним, $v_1, \ldots, v_k \in L$ линейно независимы над~$L$.
    Тогда $v_1, \ldots, v_k$ линейно независимы и~над~$V$.
\end{lemma*}

\begin{proof}
    Пусть $\alpha_1 u_1 + \ldots + \alpha_k u_k = 0$, где $\alpha_k \in \R$. Приблизим $\alpha_i$ рациональными числами:
    пусть $p^n_i \in \Z$, $q^n_i \in \mathbb{N} \setminus \{ 0 \}$ и~$\alpha_i = \lim\limits_{n \rightarrow \infty} (p^n_i / q^n_i)$.

    Пользуясь линейностью предела, получаем:
    $$
        \lim\limits_{n \rightarrow \infty} \left( \frac{p^n_1}{q^n_1} u_1 + \ldots + \frac{p^n_k}{q^n_k} u_k \right) =
        \left( \lim\limits_{n \rightarrow \infty} \frac{p^n_k}{q^n_k} \right) u_1 + \ldots + \left( \lim\limits_{n \rightarrow \infty} \frac{p^n_k}{q^n_k} \right) u_k =
        \alpha_1 u_1 + \ldots + \alpha_k u_k = 0
    $$

    ????
\end{proof}

\begin{theorem*}
    Для~всякой решётки $L$ верно, что $\dim_\Z(L) = \dim(L[\R])$.
\end{theorem*}

\begin{proof}
    ???
\end{proof}

\end{document}