\documentclass{article}

\usepackage[utf8]{inputenc}
\usepackage[T2A]{fontenc}
\usepackage[russian]{babel}
\usepackage[tmargin=1in,bmargin=1in,lmargin=1.25in,rmargin=1.25in]{geometry}

\usepackage[hidelinks,unicode]{hyperref}
\usepackage[threshold=0]{csquotes}
\usepackage{indentfirst}
\usepackage{enumerate}
\usepackage{totcount}
\usepackage{titlesec}
\usepackage{etoolbox}
\usepackage{amsmath}
\usepackage{amssymb}
\usepackage{amsthm}
\usepackage{hhline}
\usepackage{array}

\usepackage{amsthm}

\newtheorem{statement}{Утверждение}
\newtheorem{consequence}{Следствие}
\newtheorem{theorem}{Теорема}
\newtheorem{lemma}{Лемма}

\newtheorem*{statement*}{Утверждение}
\newtheorem*{consequence*}{Следствие}
\newtheorem*{theorem*}{Теорема}
\newtheorem*{lemma*}{Лемма}

\newcommand{\divides}{\mid}

\newcommand{\N}{\mathbb{N}}
\newcommand{\Z}{\mathbb{Z}}
\newcommand{\Q}{\mathbb{Q}}
\newcommand{\R}{\mathbb{R}}

\newcommand{\primes}{\mathbb{P}}
\newcommand{\complex}{\mathbb{C}}
\newcommand{\quaternion}{\mathbb{H}}

\newcommand{\torus}{\complex^\times}

\newcommand{\image}{\mathrm{Im}}

\newcommand{\Hom}{\mathrm{Hom}}
\newcommand{\Hol}{\mathrm{Hol}}

\newcommand{\diag}{\mathrm{diag}}

\newcommand{\rank}{\mathrm{rank}}
\newcommand{\Span}{\mathrm{Span}}

\newcommand{\GL}{\mathrm{GL}}
\newcommand{\SNF}{\mathrm{SNF}}

\def\paddedtext#1#2{\leavevmode\hbox to#1{\hss#2\hss}\ignorespaces}
\patchcmd{\thebibliography}{\section*{\refname}}{}{}{}

\newcolumntype{L}[1]{>{\raggedright\let\newline\\\arraybackslash\hspace{0pt}}m{#1}}
\newcolumntype{C}[1]{>{\centering\let\newline\\\arraybackslash\hspace{0pt}}m{#1}}
\newcolumntype{R}[1]{>{\raggedleft\let\newline\\\arraybackslash\hspace{0pt}}m{#1}}

\newcommand{\usection}[1]{\phantomsection\section*{\centering#1}
\addcontentsline{toc}{section}{\protect\numberline{}#1}}

\newcommand{\usubsection}[1]{\phantomsection\subsection*{\centering#1}
\addcontentsline{toc}{subsection}{\protect\numberline{}#1}}

\newcommand{\usubsubsection}[1]{\phantomsection\subsubsection*{\centering#1}
\addcontentsline{toc}{subsubsection}{\protect\numberline{}#1}}


\begin{document}

Для~поля $K$ и~$n$ векторов $\alpha_i \in \Z^k$ рассмотрим отображение $\phi_\alpha(t) = (t^{\alpha_1}, t^{\alpha_2}, \ldots, t^{\alpha_n})$,
где $t \in (K^\times)^k$, а~$\alpha$~— матрица со~строками $\alpha_i$. Поскольку $(t_1 \cdot t_2)^{\alpha_i} = t_1^{\alpha_i} \cdot t_2^{\alpha_i}$,
$\phi_\alpha$~— гомоморфизм.

Далее увидим, что в~случае $K = \complex$ отображение $\phi_\alpha$ задаёт \textit{параметризацию} некоторой алгебраической группы.

\begin{statement*}
    $\phi_{E} = 1_{(K^\times)^n}$, где $E$~— единичная матрица $n \times n$, $1_X$~— тождественная функция на~множестве $X$.
\end{statement*}

\begin{statement*}
    $
        \forall \alpha \in \Z^{k \times m}, \beta \in \Z^{m \times n}{:}\ \phi_\alpha \circ \phi_\beta = \phi_{\alpha \beta}
    $.
\end{statement*}

\begin{proof}
    Действительно, рассмотрим $t \in (K^\times)^n$. Тогда $i$-я компонента $\phi_\alpha(\phi_\beta(t))$
    равна $\phi_\beta^{\alpha_i}(t) = (t^{\beta_1})^{\alpha_i^1} \ldots (t^{\beta_m})^{\alpha_i^m}
                                    = t_1^{\beta_1^1 \alpha_i^1 + \ldots + \beta_m^1 \alpha_i^m}
                                      \ldots
                                      t_n^{\beta_1^n \alpha_i^1 + \ldots + \beta_m^n \alpha_i^m}
                                    = t^{(\alpha \beta)_i}$.
\end{proof}

Вместе эти два утверждения составляют условие на~функториальность. Более точно, рассмотрим категорию
$\mathrm{Matr}(R)$ матриц над~кольцом $R$, объектами которой являются натуральные числа, а~стрелками
между числами $m$ и~$n$~— матрицы $R^{n \times m}$, и~категорию $\mathrm{Tor}(K)$ торических групп над~полем $K$,
объектами которой являются группы $(K^\times)^k$, а~стрелками~— гомоморфизмы между ними.
Тогда определён функтор $\phi : \mathrm{Matr}(\Z) \rightarrow \mathrm{Tor}(K)$:
$$
    \phi = \begin{cases}
        k \mapsto (K^\times)^k; \\
        \alpha \mapsto \phi_\alpha.
    \end{cases}
$$

Как показывает следующая лемма, этот функтор~— строгий.

\begin{lemma*}
    $
        \forall \alpha, \beta \in \Z^{n \times k}{:} \ \phi_\alpha = \phi_\beta \Leftrightarrow \alpha = \beta
    $.
\end{lemma*}

\begin{proof}
    Импликация справа налево очевидна.

    Наоборот, пусть $\phi_\alpha = \phi_\beta$. Тогда для~всякого $1 \leq i \leq n$ верно, что
    $$
        (e^{\alpha_1^i}, \ldots, e^{\alpha_k^i}) = \phi_\alpha(e \cdot e_i) = \phi_\beta(e \cdot e_i) = (e^{\beta_1^i}, \ldots, e^{\beta_k^i}).
    $$

    Таким образом, $e^{\alpha_j^i} = e^{\beta_j^i}$. $\alpha_j^i$ и~$\beta_j^i$ целочисленные, поэтому $\alpha_j^i = \beta_j^i$.
\end{proof}

Будем пользоваться существованием для~целочисленных матриц \textit{нормальной формы Смита.}
Пусть $\mathrm{diag}^{m \times n}_r(x_1, \ldots, x_r)$~— диагональная матрица $\mathrm{diag}(x_1, \ldots, x_r)$,
дополненная (или обрезанная) справа снизу нулями до~матрицы размера $m \times n$.

\begin{theorem*}[о существовании нормальной формы Смита]
    $$
        \forall \alpha \in \Z^{m \times n}{:}\
        \exists \beta_1 \in \GL^m(\Z), \beta_2 \in \GL^n(\Z){:} \
        \exists \varepsilon_1, \ldots, \varepsilon_r \in \Z \setminus \{0\}{:} \
        \beta_1 \alpha \beta_2 = \mathrm{diag}^{m \times n}_r(\varepsilon_1, \ldots, \varepsilon_r),
    $$
    причём $\varepsilon_1 \divides \varepsilon_2 \divides \ldots \divides \varepsilon_r$ (так называемые инвариантные факторы).
\end{theorem*}

Поскольку числа $\varepsilon_1$, $\ldots$, $\varepsilon_r$ выбираются единственным образом с~точностью
до~обратимого элемента, то~есть в~случае $\Z$ с~точностью до~$\pm 1$, можем всегда выбрать их положительными.
Для~них матрицу $\mathrm{diag}^{m \times n}_r(\varepsilon_1, \ldots, \varepsilon_r)$ обозначим как $\mathrm{SNF}(\alpha)$.

Помимо этого, с~учётом функториальности $\phi$, ясно, что $\phi_\beta$ биективно, если $\beta \in \GL^n(\Z)$.
Действительно, $\phi_\beta \circ \phi_{\beta^{-1}} = \phi_{\beta \beta^{-1}} = \phi_E = 1_{(K^\times)^n}$
и~$\phi_{\beta^{-1}} \circ \phi_\beta = \phi_{\beta^{-1} \beta} = \phi_E = 1_{(K^\times)^n}$.
Поэтому имеет место следующая лемма.

\begin{lemma*}
    $\phi_\alpha : (\complex^\times)^k \rightarrow (\complex^\times)^n$ инъективна тогда и~только тогда, когда
    $$
        \mathrm{SNF}(\alpha) = \mathrm{diag}^{n \times k}_k(1, 1, \ldots, 1).
    $$
\end{lemma*}

\begin{proof}
    Пользуясь теоремой, возьмём унимодулярные матрицы $\beta_1 \in \GL^n(\Z)$ и~$\beta_2 \in \GL^k(\Z)$
    такие, что $\beta_1 \alpha \beta_2 = \mathrm{SNF}(\alpha)$.
    В~таком случае, $\alpha = \beta_1^{-1} \mathrm{SNF}(\alpha) \beta_2^{-1}$, и~$\phi_\alpha = \phi_{\beta_1^{-1} \mathrm{SNF}(\alpha) \beta_2^{-1}}
                                                                                    = \phi_{\beta_1^{-1}} \circ \phi_{\mathrm{SNF}(\alpha)} \circ \phi_{\beta_2^{-1}}$.

    Поскольку, по~замечанию выше, $\phi_{\beta_1^{-1}}$ и~$\phi_{\beta_2^{-1}}$ биективны, то~$\phi_\alpha$
    инъективно тогда и~только тогда, когда инъективно $\phi_{\mathrm{SNF(\alpha)}}$.

    $\phi_{\mathrm{SNF(\alpha)}}(t_1, \ldots, t_k) = (t_1^{\varepsilon_1}, \ldots, t_r^{\varepsilon_r}, 1, \ldots, 1)$,
    но~$t \mapsto t^\varepsilon$ инъективно над~$\complex^\times$ только в~случае $\varepsilon = \pm 1$
    (иначе $t^\varepsilon = t^\varepsilon \cdot 1 = t^\varepsilon \cdot u^\varepsilon = (tu)^\varepsilon$,
     где~$u \neq 1$~— нетривиальный корень из~единицы степени $\varepsilon$).
    В~силу выбора знаков, $\phi_{\mathrm{SNF(\alpha)}}$ инъективна лишь когда $\varepsilon_1 = \ldots = \varepsilon_r = 1$
    и~$r = k \leq n$, что и~требовалось.
\end{proof}

Сформулируем ещё одну вспомогательную теорему.
Обозначим \textit{$\Z$-линейную оболочку} векторов $\alpha_1, \ldots, \alpha_n$ как
$\Span_\Z(\alpha_1, \ldots, \alpha_n) = \{ k_1 \alpha_1 + \ldots + k_n \alpha_n \ | \ k_1, \ldots, k_n \in \Z \}$.

Говорим, что $\alpha_1, \ldots, \alpha_n$ \textit{порождают всю решётку} $\Z^k$, если $\Span_\Z(\alpha_1, \ldots, \alpha_n) = \Z^k$.

\begin{theorem*}
    Пусть $\alpha \in \Z^{n \times k}$. Следующие утверждения равносильны:
    \begin{enumerate}
        \item Строки матрицы $\alpha$ порождают всю решётку $\Z^k$.
        \item Миноры максимальной размерности матрицы $\alpha$ взаимно просты.
        \item $\mathrm{SNF}(\alpha) = \mathrm{diag}^{n \times k}_k(1, \ldots, 1)$.
    \end{enumerate}
\end{theorem*}

Таким образом, получаем необходимое и~достаточное условие инъективности $\phi_\alpha$.

\begin{theorem*}
    $\phi_\alpha$ инъективно тогда и~только тогда, когда~$\alpha_i$ порождают всю решётку $\Z^k$.
\end{theorem*}

Поскольку $\phi_\alpha$~— гомоморфизм, образ $\mathrm{Im}(\phi_\alpha)$ является подгруппой в~$(K^\times)^n$.
Изучим два вопроса: является~ли $\mathrm{Im}(\phi_\alpha)$ алгебраической подгруппой и~всякая~ли алгебраическая подгруппа задаётся некоторым $\phi_\alpha$.

Для~этого, прежде всего, заметим, что система биномиальных уравнений $z^{\beta'_i} = z^{\beta''^i}$ эквивалентна
системе $z^{\beta'_i - \beta''_i} = 1$, то~есть в~точности ядру оператора $\phi_{\beta'_i - \beta''_i}$;
поэтому первый вопрос эквивалентен тому, можно~ли данный гомоморфизм $\phi_\alpha$ достроить
до~точной последовательности $(K^\times)^k \xrightarrow[]{\phi_\alpha} (K^\times)^n \xrightarrow[]{\phi_\beta} (K^\times)^m$.

\begin{lemma*}
    Пусть $G$~— группа, $H$~— абелева группа, $f \in \Hom(G, H)$, $g \in \Hom(H, G)$ и~$g \circ f = 1_G$.
    Тогда последовательность $G \xrightarrow[]{f} H \xrightarrow[]{f \circ g - 1_H} H$ точна.
\end{lemma*}

\begin{proof}
    Действительно, $(f \circ g - 1_H) \circ f = f \circ g \circ f - f = f - f = 0$, то~есть $\mathrm{Im}(f) \subseteq \ker(f \circ g - 1_H)$. 
    Обратно, пусть $h \in \ker(f \circ g - 1_H)$. Тогда $f(g(h)) - h = 0 \Leftrightarrow h = f(g(h))$, то~есть $h \in \mathrm{Im}(f)$,
    и~$\ker(f \circ g - 1_H) \subseteq \mathrm{Im}(f)$.
\end{proof}

\begin{theorem*}
    Всякая параметризация $\phi_\alpha$, где $\alpha \in \Z^{n \times k}$, задаёт алгебраическую подгруппу $(K^\times)^n$.
\end{theorem*}

\begin{proof}
    Снова представим $\alpha$ в~виде $\alpha = \beta_1^{-1} \varepsilon \beta_2^{-1}$, где
    $$
        \varepsilon = \mathrm{SNF}(\alpha) = \mathrm{diag}^{n \times k}_r(\varepsilon_1, \ldots, \varepsilon_r).
    $$

    Рассмотрим $\delta = \mathrm{diag}^{n \times r}_r(1, \ldots, 1)$ и~$\alpha' = \beta^{-1} \delta$.
    $\phi_{\alpha'}$ инъективна как композиция инъективных функций. С~другой стороны, очевидно, что:
    $$
        \mathrm{Im}(\phi_\alpha) = \phi_{\beta_1^{-1}} (\phi_\varepsilon (\phi_{\beta_2^{-1}} ((\complex^\times)^k)))
                                 = \phi_{\beta_1^{-1}} (\phi_\varepsilon ((\complex^\times)^k))
                                 = \phi_{\beta_1^{-1}} (\phi_{\delta} ((\complex^\times)^r))
                                 = \mathrm{Im}(\phi_{\alpha'}).
    $$

    Нетрудно видеть, что $\alpha'$ имеет левой обратной матрицу $\beta' = \delta^{\top} \beta$,
    а~потому гомоморфизм $\phi_{\alpha'}$ имеет левым обратным гомоморфизм $\phi_{\beta'}$.
    Поскольку $(\phi_{\alpha'} \circ \phi_{\beta'}) \phi_{-E} = \phi_{\alpha' \beta' - E}$, остаётся лишь применить лемму.
\end{proof}

Ответ на~второй вопрос несколько сложнее. Например, уравнение $z^n = 1$ задаёт алгебраическую
группу для~любого $n \in \Z$, состоящую из~$n$ точек на~$\complex^\times$; но~если $n \neq 0, \pm 1$,
то~легко видеть, что она не~может быть параметризована никакой $\phi_\alpha$.

Действительно, $\phi_\alpha$~— голоморфное отображение, $(\complex^\times)^k$~— открытое множество,
поэтому всякая проекция полного образа $\mathrm{Im}(\phi_\alpha) = \phi_\alpha((\complex^\times)^k)$ либо, в~силу
открытости непостоянных голоморфных отображений, открыта, либо состоит из~одной точки; но~$z^n = 1$
открыто, только если $n = 0$, и~состоит из~одной точки, только если $n = \pm 1$.

Более того, из~этих рассуждений видно, что~по~всякой системе биномиальных уравнений можно построить новую систему,
которая гарантированно не~параметризуется целиком, возведя, например, произвольное уравнение в~степень $n > 1$.

\begin{theorem*}
    Всякая алгебраическая подгруппа тора $(\complex^\times)^n$ содержит параметризуемую подгруппу той~же размерности.
\end{theorem*}

\begin{proof}
    Пусть $H = \ker(\phi_\beta)$~— алгебраическая группа. Без~потери общности считаем,
    что строки матрицы $\beta$ линейно независимы над~$\Z$ (ясно, что строки, выражающиеся
    как линейная комбинация остальных строк, можно убрать из~матрицы $\beta$ без~изменения ядра).

    Запишем для~$\beta$ нормальную форму Смита: $\beta = \beta_1^{-1} \varepsilon \beta_2^{-1}$.
    Поскольку, по~замечанию выше, $\beta$ имеет полный ранг, матрица $\varepsilon$ не~имеет нулевых строк.

    Это значит, что ядро $\varepsilon$ задаётся векторами вида $(0, \ldots, 0, t_{k + 1}, \ldots, t_{n})$.
    Рассмотрим матрицу $\delta \in \Z^{n \times (n - k)}$, соответствующую линейному оператору
    $$
        (t_1, \ldots, t_{n - k}) \mapsto (0, \ldots, 0, t_1, \ldots, t_{n - k}).
    $$

    Пусть также $\alpha = \beta_2 \delta$. Тогда $\phi_\alpha$ инъективно по~построению,
    а~$\mathrm{Im}(\phi_{\alpha})$~— искомая алгебраическая подгруппа.
    Действительно, по~заданию $\delta$, $\varepsilon \delta = 0$, поэтому 
    $\phi_{\beta} \circ \phi_{\alpha} = \phi_{\beta_1^{-1} \varepsilon \beta_2^{-1} \beta_2 \delta} = \phi_{\beta_1^{-1} \varepsilon \delta} = \phi_{0} = 1$,
    то~есть $\mathrm{Im}(\phi_{\alpha}) \subseteq \ker(\phi_{\beta})$.

    Кроме того, $H = \ker(\phi_{\beta})$ задаётся $k$ уравнениями, а~потому (локально) не~более
    чем $n - k$ независимыми параметрами, как и~$\mathrm{Im}(\phi_{\alpha})$.
\end{proof}

\end{document}