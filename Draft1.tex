\documentclass{article}

\usepackage[utf8]{inputenc}
\usepackage[T2A]{fontenc}
\usepackage[russian]{babel}
\usepackage[tmargin=1in,bmargin=1in,lmargin=1.25in,rmargin=1.25in]{geometry}

\usepackage[hidelinks,unicode]{hyperref}
\usepackage[threshold=0]{csquotes}
\usepackage{indentfirst}
\usepackage{enumerate}
\usepackage{totcount}
\usepackage{titlesec}
\usepackage{etoolbox}
\usepackage{amsmath}
\usepackage{amssymb}
\usepackage{amsthm}
\usepackage{hhline}
\usepackage{array}

\usepackage{amsthm}

\newtheorem{statement}{Утверждение}
\newtheorem{consequence}{Следствие}
\newtheorem{theorem}{Теорема}
\newtheorem{lemma}{Лемма}

\newtheorem*{statement*}{Утверждение}
\newtheorem*{consequence*}{Следствие}
\newtheorem*{theorem*}{Теорема}
\newtheorem*{lemma*}{Лемма}

\newcommand{\divides}{\mid}

\newcommand{\N}{\mathbb{N}}
\newcommand{\Z}{\mathbb{Z}}
\newcommand{\Q}{\mathbb{Q}}
\newcommand{\R}{\mathbb{R}}

\newcommand{\primes}{\mathbb{P}}
\newcommand{\complex}{\mathbb{C}}
\newcommand{\quaternion}{\mathbb{H}}

\newcommand{\torus}{\complex^\times}

\newcommand{\image}{\mathrm{Im}}

\newcommand{\Hom}{\mathrm{Hom}}
\newcommand{\Hol}{\mathrm{Hol}}

\newcommand{\diag}{\mathrm{diag}}

\newcommand{\rank}{\mathrm{rank}}
\newcommand{\Span}{\mathrm{Span}}

\newcommand{\GL}{\mathrm{GL}}
\newcommand{\SNF}{\mathrm{SNF}}

\def\paddedtext#1#2{\leavevmode\hbox to#1{\hss#2\hss}\ignorespaces}
\patchcmd{\thebibliography}{\section*{\refname}}{}{}{}


\begin{document}

\section{Алгебраические подгруппы комплексного тора}

\textbf{Алгебраической подгруппой} группы $G$ называется подгруппа, также являющася
алгебраическим многообразием, то~есть задаваемая системой полиномиальных уравнений.

Для~векторов $\alpha = (\alpha_1, \ldots, \alpha_n) \in \mathbb{Z}^n$ и~$z = (z_1, \ldots, z_n) \in G^n$ (где $G$~— группа)
обозначим $z^\alpha = z_1^{\alpha_1} \cdot z_2^{\alpha_2} \cdot \ldots \cdot z_n^{\alpha_n}$. Тогда справедлива следующая теорема.

\begin{theorem*}[Шмидт]
    Пусть $K$~— поле. Всякая алгебраическая подгруппа $H$ группы $(K^{\times})^n$ задаётся
    системой некоторого числа $N$ биномиальных уравнений, то~есть существуют такие показатели $\alpha_i, \beta_i \in \mathbb{Z}^n$, что
    $$
        H = \{ z \in (K^{\times})^n\ |\ \forall 1 \leq i \leq N{:}\ z^{\alpha_i} = z^{\beta_i} \}.
    $$
\end{theorem*}

Для~её доказательства нам понадобится вспомогательное утверждение.

\subsection{Теорема Артина}

Множество гомоморфизмов между группами $G$ и~$H$ обозначим как~$\Hom(G, H)$.

Пусть $G$~— группа, $K$~— поле, а~$K^{\times}$~— его мультипликативная группа.
В~таком случае произвольный гомоморфизм $f \in \Hom(G, K^{\times})$ называют \textbf{характером.}
Говорят, что характеры $f_1, f_2, \ldots, f_n$ \textbf{линейно независимы,}
если
$$
    \forall \alpha_1, \alpha_2, \ldots, \alpha_n \in K{:}\ \alpha_1 f_1 + \alpha_2 f_2 + \ldots + \alpha_n f_n = 0 \Rightarrow \alpha_1 = \alpha_2 = \ldots = \alpha_n = 0.
$$

\begin{theorem*}[Артин]
    Любые $n$ попарно различных характеров линейно независимы.
\end{theorem*}

\begin{proof}

Докажем индукцией по~числу характеров $n$.

Возьмём произвольный характер $f$. Поскольку он является гомоморфизмом,
то~$f(1_G) = 1$, где $1_G$~— единица в~группе $G$. Но~тогда если $\alpha f = 0$,
то~и~$\alpha = \alpha \cdot 1 = \alpha \cdot f(1_G) = 0$, что доказывает базу индукции.

Пусть теперь утверждение теоремы верно для~любых $n$ различных характеров. Докажем его для~$(n + 1)$ характера.
Пусть $\alpha_1 f_1 + \alpha_2 f_2 + \ldots + \alpha_n f_n + \alpha_{n + 1} f_{n + 1} = 0$.

Зафиксируем произвольный $y \in G$. Тогда для~любого $x \in G$ имеем:
\begin{equation}\label{Artin:first}
    \alpha_1 f_1(yx) + \alpha_2 f_2(yx) + \ldots + \alpha_n f_n(yx) + \alpha_{n + 1} f_{n + 1}(yx) = 0
\end{equation}

Поскольку все $f_i$~— гомоморфизмы, $f_i(yx) = f_i(y) f_i(x)$, а~потому:
$$
    \alpha_1 f_1(y) f_1(x) + \alpha_2 f_2(y) f_2(x) + \ldots + \alpha_n f_n(y) f_n(x) + \alpha_{n + 1} f_{n + 1}(y) f_{n + 1}(x) = 0
$$

С~другой стороны, для~$x \in G$ также верно, что:
$$
    \alpha_1 f_1(x) + \alpha_2 f_2(x) + \ldots + \alpha_n f_n(x) + \alpha_{n + 1} f_{n + 1}(x) = 0
$$

Умножим это равенство на~$f_{n + 1}(y)$:
\begin{equation}\label{Artin:second}
    \alpha_1 f_{n + 1}(y) f_1(x) + \alpha_2 f_{n + 1}(y) f_2(x) + \ldots + \alpha_n f_{n + 1}(y) f_n(x) + \alpha_{n + 1} f_{n + 1}(y) f_{n + 1}(x) = 0
\end{equation}

Вычитая \eqref{Artin:first} из~\eqref{Artin:second}, получаем:
$$
    \alpha_1 (f_{n + 1}(y) - f_1(y)) f_1(x) + \ldots + \alpha_n (f_{n + 1}(y) - f_n(y)) f_n(x) = 0
$$

Тогда, в~силу произвольности $x$, получаем линейную комбинацию $n$ характеров:
$$
    \alpha_1 (f_{n + 1}(y) - f_1(y)) f_1 + \ldots + \alpha_n (f_{n + 1}(y) - f_n(y)) f_n = 0
$$

Но~в~таком случае, по~индуктивной гипотезе, $\alpha_i (f_{n + 1}(y) - f_i(y)) = 0$.
Теперь, выбрав для~каждого $i = 1, \ldots, n$ такое $y$, что $f_{n + 1}(y) \neq f_i(y)$
(это возможно, потому~что, по~условию теоремы, все $f_i$ попарно различны), получим,
что $\alpha_1 = \alpha_2 = \ldots = \alpha_n = 0$.

Таким образом, с~учётом вышесказанного, $\alpha_{n + 1} f_{n + 1} = \alpha_1 f_1 + \alpha_2 f_2 + \ldots + \alpha_n f_n + \alpha_{n + 1} f_{n + 1} = 0$.
Согласно базе индукции, $\alpha_{n + 1} = 0$.

\end{proof}

\subsection{Доказательство теоремы Шмидта}

\begin{proof}
    Пусть $I \subseteq \mathbb{Z}^n$~— конечное множество индексов,
    а~$P_j(z) = \sum_{i \in I} a_{ji} z^i$~— многочлены ($k$ штук), задающие подгруппу:
    $$
        H = \{ z \in (K^{\times})^n\ |\ \forall 1 \leq j \leq k{:}\ P_j(z) = 0 \}.
    $$

    Поскольку $z_1^{i} z_2^{i} = (z_1 z_2)^i$, отображение $z \mapsto z^i$ определяет
    характер $\chi_i \in \Hom(H, K^{\times})$.

    Рассмотрим на~множестве $I$ отношение эквивалентности $i \sim j \Leftrightarrow \chi_i = \chi_j$.
    Оно разбивает $I$ на~$m$ классов $I_1$, $I_2$, …, $I_m$. В~таком случае, для~всех $i_1, i_2 \in I_j$ верно, что $\chi_{i_1} = \chi_{i_2}$.
    Обозначим этот характер, соответствующий каждому $I_k$, как $\chi_k = \chi_{i_1} = \chi_{i_2}$.

    Собрав подобные слагаемые при~каждом $\chi_k$ в~многочленах $P_j$, получим:
    $$
        P_j = \sum_{i \in I} a_{ji} \chi_i = \sum_{k = 1}^{m} \left( \sum_{i \in I_k} a_{ji} \right) \chi_k
    $$

    Но~$P_j$ равны нулю на~всём $H$, поэтому:
    $$
    \sum_{k = 1}^{m} \left( \sum_{i \in I_k} a_{ji} \right) \chi_k = 0
    $$

    Все $\chi_k$ попарно различны по~их заданию, поэтому к~ним применима теорема Артина.
    Из~неё заключаем, что:
    $$
    \sum_{i \in I_k} a_{ji} = 0
    $$

    Наконец, пусть $N_k$~— мощность $I_k$, $I_k = \{i_{k, 1}, i_{k, 2}, \ldots, i_{k, N_k}\}$ и~$N = \sum_{k = 1}^m (N_k - 1)$.
    В~качестве $\alpha_i$ и~$\beta_i$ положим подряд идущие индексы $i_{k, j}$ и~$i_{k, j + 1}$.
    Тогда подгруппа $G = \{ z \in (K^{\times})^n\ |\ \forall 1 \leq i \leq N{:}\ z^{\alpha_i} = z^{\beta_i} \}$ задаётся следующими уравнениями:
    $$
        z^{i_{1, 1}} = z^{i_{1, 2}}, z^{i_{1, 2}} = z^{i_{1, 3}}, \ldots, z^{i_{1, N_1 - 1}} = z^{i_{1, N_1}}, z^{i_{2, 1}} = z^{i_{2, 2}}, \ldots, z^{i_{m, N_m - 1}} = z^{i_{m, N_m}}.
    $$

    Покажем, что $G$ совпадает с~$H$. Действительно, пусть $z \in H$. Тогда, по построению $I_k$,
    $z^{i_{k, j}} = \chi_k(z) = z^{i_{k, j + 1}}$; то~есть $z \in G$, и~$H \subseteq G$.

    Наоборот, пусть $z \in G$. Тогда, по~заданию $G$, $z^{i_k} = z^{i_{k, j_1}} = z^{i_{k, j_2}}$
    для~любых $j_1$ и~$j_2$. Снова собирая подобные слагаемые в~$P_j$, окончательно получим:
    $$
        P_j(z) = \sum_{k = 1}^{m} \left( \sum_{i \in I_k} a_{ji} \right) z^{i_k} = \sum_{k = 1}^{m} 0 \cdot z^{i_k} = 0
    $$

    Таким образом, $G \subseteq H$, что и~требовалось.
\end{proof}

\end{document}