\documentclass{article}

\usepackage[utf8]{inputenc}
\usepackage[T2A]{fontenc}
\usepackage[russian]{babel}
\usepackage[tmargin=1in,bmargin=1in,lmargin=1.25in,rmargin=1.25in]{geometry}

\usepackage[hidelinks,unicode]{hyperref}
\usepackage[threshold=0]{csquotes}
\usepackage{indentfirst}
\usepackage{enumerate}
\usepackage{totcount}
\usepackage{titlesec}
\usepackage{etoolbox}
\usepackage{amsmath}
\usepackage{amssymb}
\usepackage{hhline}
\usepackage{array}

\usepackage{amsthm}

\newtheorem{statement}{Утверждение}
\newtheorem{consequence}{Следствие}
\newtheorem{theorem}{Теорема}
\newtheorem{lemma}{Лемма}

\newtheorem*{statement*}{Утверждение}
\newtheorem*{consequence*}{Следствие}
\newtheorem*{theorem*}{Теорема}
\newtheorem*{lemma*}{Лемма}

\newcommand{\divides}{\mid}

\newcommand{\N}{\mathbb{N}}
\newcommand{\Z}{\mathbb{Z}}
\newcommand{\Q}{\mathbb{Q}}
\newcommand{\R}{\mathbb{R}}

\newcommand{\primes}{\mathbb{P}}
\newcommand{\complex}{\mathbb{C}}
\newcommand{\quaternion}{\mathbb{H}}

\newcommand{\torus}{\complex^\times}

\newcommand{\image}{\mathrm{Im}}

\newcommand{\Hom}{\mathrm{Hom}}
\newcommand{\Hol}{\mathrm{Hol}}

\newcommand{\diag}{\mathrm{diag}}

\newcommand{\rank}{\mathrm{rank}}
\newcommand{\Span}{\mathrm{Span}}

\newcommand{\GL}{\mathrm{GL}}
\newcommand{\SNF}{\mathrm{SNF}}

\def\paddedtext#1#2{\leavevmode\hbox to#1{\hss#2\hss}\ignorespaces}
\patchcmd{\thebibliography}{\section*{\refname}}{}{}{}


\begin{document}

Для~$n$ векторов $\alpha_i \in \Z^k$ рассмотрим отображение $\phi_\alpha(t) = (t^{\alpha_1}, t^{\alpha_2}, \ldots, t^{\alpha_n})$,
где $t \in (\torus)^k$, а~$\alpha$~— матрица со~строками $\alpha_i$.
Отображение $\phi_\alpha$ задаёт \textit{параметризацию} некоторой алгебраической группы.

$j$-ю компоненту вектора $\alpha_i$ обозначим как $\alpha_i^j$, а~вектор, состоящий
из~$j$-х компонент, как $\alpha^j = (\alpha_1^j, \alpha_2^j, \ldots, \alpha_n^j)$.

Обозначим \textit{$\Z$-линейную оболочку} векторов $\alpha_1, \ldots, \alpha_n$ как
$\Span_\Z(\alpha_1, \ldots, \alpha_n) = \{ k_1 \alpha_1 + \ldots + k_n \alpha_n \ | \ k_1, \ldots, k_n \in \Z \}$.

Говорим, что $\alpha_1, \ldots, \alpha_n$ порождают всю решётку, если $\Span_\Z(\alpha_1, \ldots, \alpha_n) = \Z^k$.

Как $e_i$ обозначим \textit{естественный базис} над~$\Z^k$, то~есть такой вектор, у~которого на~$i$-м месте единица, а~на~остальных~— нули.
В~таком случае очевидно следующее утверждение.

\begin{statement*}
    $\Span_\Z(\alpha_1, \ldots, \alpha_n) = \Z^k \Leftrightarrow \forall i{:}\ e_i \in \Span_\Z(\alpha_1, \ldots, \alpha_n)$.
\end{statement*}

\begin{lemma*}
    Если $\alpha_i \in \Z^k$ порождают всю решётку, то~$\phi_\alpha$ инъективно.
\end{lemma*}

\begin{proof}
    $\phi_\alpha$~— гомоморфизм, потому достаточно показать, что $\ker(\phi_\alpha) = \{1\}$.

    Возьмём $t \in (\torus)^k$ такой, что $\phi_\alpha(t) = 1$, и~докажем, что $t = 1$.

    Действительно, поскольку $\alpha_i$ порождают всю решётку, каждый $e_j$ выражается как линейная комбинация векторов $\alpha_i$:
    $$
        e_j = b^j_1 \alpha_1 + \ldots + b^j_n \alpha_n.
    $$

    $\phi_\alpha(t) = 1$ означает, что $t^{\alpha_i} = 1$ для~всех $1 \leq i \leq n$.
    Зафиксируем $1 \leq j \leq k$ и~возведём это равенство в~степень $b^j_i$: $t^{b^j_i \alpha_i} = (t^{\alpha_i})^{b^j_i} = 1^{b^j_i} = 1$.
    Наконец, перемножим полученные равенства:
    $$
        t_j = t^{e_j} = t^{b^j_1 \alpha_1 + \ldots + b^j_n \alpha_n} = t^{b^j_1 \alpha_1} \ldots t^{b^j_n \alpha_n} = 1 \cdot \ldots \cdot 1 = 1,
    $$
    что и~требовалось.
\end{proof}

Теперь покажем, что полученное в~лемме небходимое условие является также и~достаточным.
Для~этого сначала сведём задачу к~более простому условию.

Во~всякой абелевой группе $G$ естественным образом определено умножение на~целые числа.
Для~вектора $\alpha = (\alpha_1, \ldots, \alpha_k) \in \Z^k$ и~элемента $g \in G$
под~$g \alpha$ понимаем вектор $(\alpha_1 g, \ldots, \alpha_k g) \in G^k$.

Говорим, что набор векторов $\alpha_1, \ldots, \alpha_n \in \Z^k$ \textit{линейно независим над~абелевой группой} $G$, если
$$
    \forall g_1, \ldots, g_n \in G{:}\ g_1 \alpha_1 + \ldots + g_n \alpha_n = 0 \Rightarrow g_1 = g_n = \ldots = 0.
$$

Видно, что линейная независимость векторов $\Z^k$ над~аддитивной группой $\R$ эквивалентна линейной независимости
в~векторном пространстве $\R^k$. Однако, например, $\R / 2 \pi \Z$, как известно, не~обладает структурой кольца,
поэтому и~не~имеет смысла говорить о~модуле $(\R / 2 \pi \Z)^k$ над~$\R / 2 \pi \Z$, как и~о~линейной независимости
в~нём.

\begin{lemma*}
    Если $\alpha^j$ линейно независимы над~группами $\R$ и~$\R / 2 \pi \Z$, то $\phi_\alpha$ инъективно.
\end{lemma*}

\begin{proof}
    Снова рассмотрим $t \in (\torus)^k$ такой, что $\phi_\alpha(t) = 1$, и~докажем, что $t = 1$.
    Для~этого представим компоненты $t_i$ в~показательной форме: $t_i = r_i e^{i {\theta_i}}$, где $r_i \in \R_{>0}^\times$ и~$\theta_i \in \R / 2 \pi \Z$.
    Но~тогда, поскольку $\phi_\alpha(t) = 1$, имеем:
    $$
        r_1^{\alpha_i^1} \ldots r_k^{\alpha_i^k} e^{i(\alpha_i^1 \theta_1 + \ldots + \alpha_i^k \theta_k)} =
        r_1^{\alpha_i^1} e^{i(\alpha_i^1 \theta_1)} \ldots r_k^{\alpha_i^k} e^{i(\alpha_i^k \theta_k)} =
        t_1^{\alpha_i^1} \ldots t_k^{\alpha_i^k} =
        t^{\alpha_i} = 1.
    $$

    В~силу единственности показательного представления заключаем, что
    $r_1^{\alpha_i^1} \ldots r_k^{\alpha_i^k} = 1$ и~$\alpha_i^1 \theta_1 + \ldots + \alpha_i^k \theta_k = 0$.

    Взяв логарифм от~первого уравнения, получим:
    $ \alpha_i^1 \log(r_1) + \ldots + \alpha_i^k \log(r_k) = 0$.

    Наконец, в~силу линейной независимости, $\log(r_1) = \ldots = \log(r_k) = 0 \Leftrightarrow r_1 = \ldots = r_k = 0$
    и~$\theta_1 = \ldots = \theta_k = 0$; то~есть $t_i = r_i e^{i \theta_i} = 1 \cdot e^0 = 1$.
\end{proof}

\begin{lemma*}
    Если $\phi_\alpha$ инъективно, то $\alpha^j$ линейно независимы над~$\R$.
\end{lemma*}

\begin{proof}
    Действительно, рассмотрим произвольную линейную комбинацию $ \beta_1 \alpha^1 + \beta_2 \alpha^2 + \ldots + \beta_k \alpha^k = 0$,
    где $\beta_j \in \R$.

    Тогда для~всякого $i$ верно, что $ \beta_1 \alpha_i^1 + \beta_2 \alpha_i^2 + \ldots + \beta_k \alpha_i^k = 0$.
    Откуда получаем:
    $$
        (e^{\beta_1})^{\alpha_i^1} (e^{\beta_2})^{\alpha_i^2} \ldots (e^{\beta_k})^{\alpha_i^k} =
        e^{\beta_1 \alpha_i^1} e^{\beta_2 \alpha_i^2} \ldots e^{\beta_k \alpha_i^k} =
        e^{\beta_1 \alpha_i^1 + \beta_2 \alpha_i^2 + \ldots + \beta_k \alpha_i^k} =
        e^0 = 1.
    $$

    Рассмотрим вектора $x = (e^{\beta_1}, e^{\beta_2}, \ldots, e^{\beta_k})$ и~$y = (1, 1, \ldots, 1)$.
    Выражение выше в~точности означает, что $\phi_\alpha(x) = \phi_\alpha(y)$. В~силу инъективности, $x = y$;
    то~есть $e^{\beta_j} = 1 \Leftrightarrow \beta_j = 0$.
\end{proof}

\begin{consequence*}
    Если $\phi_\alpha$ инъективно, то~$\rank(\alpha) = k$.
\end{consequence*}

В~частности, это означает, что параметризация с~числом переменных $k > n$ заведомо неинъективна, чего и~следовало ожидать.

\begin{lemma*}
    Если $\phi_\alpha$ инъективно, то $\alpha^j$ линейно независимы над~$\R / 2 \pi \Z$.
\end{lemma*}

\begin{proof}
    Аналогично, рассмотрим произвольную линейную комбинацию $\beta_1 \alpha^1 + \beta_2 \alpha^2 + \ldots + \beta_k \alpha^k = 0$,
    где $\beta_j \in \R / 2 \pi \Z$. Откуда получаем:
    $$
        (e^{i\beta_1})^{\alpha_i^1} (e^{i\beta_2})^{\alpha_i^2} \ldots (e^{i\beta_k})^{\alpha_i^k} =
        e^{i\beta_1 \alpha_i^1} e^{i\beta_2 \alpha_i^2} \ldots e^{i\beta_k \alpha_i^k} =
        e^{i(\beta_1 \alpha_i^1 + \beta_2 \alpha_i^2 + \ldots + \beta_k \alpha_i^k)} =
        e^0 = 1.
    $$

    Рассмотрим вектора $x = (e^{i\beta_1}, e^{i\beta_2}, \ldots, e^{i\beta_k})$ и~$y = (1, 1, \ldots, 1)$.
    Из~равенства выше получаем, что $\phi_\alpha(x) = \phi_\alpha(y)$, а~потому $x = y$, и~$e^{i\beta_j} = 1 \Leftrightarrow \beta_j = 0$
    (то~есть существуют такие $k_j \in \Z$, что $\beta_j = 2 \pi k_j$).
\end{proof}

Таким образом, верна следующая теорема.

\begin{theorem*}
    $\phi_\alpha$ инъективно тогда и~только тогда, когда $\alpha^j$ линейно независимы над~группами $\R$ и~$\R / 2 \pi \Z$.
\end{theorem*}

Нетрудно видеть, что подобное условие проистекает из~того факта, что группа $\torus$ изоморфна прямому произведению
$\R_{>0}^\times \times \R / 2 \pi \Z \cong \R_{>0}^\times \times S^1$. Заметив, что $\R_{>0}^\times \cong \R_{>0}^\times \times \Z / 1\Z$,
$\R^\times \cong \R_{>0}^\times \times \Z / 2 \Z$ и~$\quaternion^\times \cong \R_{>0}^\times \times S^3$ (кватернионы), несложно получить
подобные условия для~инъективности $\phi_\alpha$ над~группами $\R_{>0}^\times$, $\R^\times$ и~$\quaternion^\times$.

Изучим подробнее следствия из~линейной независимости над~$\R / 2 \pi \Z$.

\begin{statement*}
    $\mu_1, \ldots, \mu_n$ линейно независимы над~$\R / 2 \pi \Z$
    тогда и~только тогда, когда они~же линейно независимы над~$\R / \Z$.
\end{statement*}

\begin{proof}
    Непосредственно следует из~цепочки эквивалентностей:
    \begin{align*}
                         & \mu_1, \ldots, \mu_n\ \text{линейно независимы над}\ \R / 2 \pi \Z \\
        \Leftrightarrow\ & \forall \theta_1, \ldots, \theta_n \in \R / 2 \pi \Z{:}\ \theta_1 \mu_1 + \ldots + \theta_n \mu_n = 0 \Rightarrow \theta_1 = \ldots = \theta_n = 0 \\
        \Leftrightarrow\ & \forall \theta_1, \ldots, \theta_n \in \R{:}\ \forall k \in \Z {:}\ \theta_1 \mu_1 + \ldots + \theta_n \mu_n = 2 \pi k \Rightarrow
                           \exists k_1, \ldots, k_n{:}\ \theta_1 = 2 \pi k_1 \wedge \ldots \wedge \theta_n = 2 \pi k_n \\
        \Leftrightarrow\ & \forall \theta_1, \ldots, \theta_n \in \R{:}\ \forall k \in \Z {:}\ \frac{\theta_1}{2\pi} \mu_1 + \ldots + \frac{\theta_n}{2 \pi} \mu_n = k \Rightarrow
                           \exists k_1, \ldots, k_n{:}\ \frac{\theta_1}{2 \pi} = k_1 \wedge \ldots \wedge \frac{\theta_n}{2 \pi} = k_n \\
        \Leftrightarrow\ & \forall \rho_1, \ldots, \rho_n \in \R{:}\ \forall k \in \Z {:}\ \rho_1 \mu_1 + \ldots + \rho_n \mu_n = k \Rightarrow
                           \exists k_1, \ldots, k_n{:}\ \rho_1 = k_1 \wedge \ldots \wedge \rho_n = k_n \\
        \Leftrightarrow\ & \forall \rho_1, \ldots, \rho_n \in \R / \Z {:}\ \rho_1 \mu_1 + \ldots + \rho_n \mu_n = 0 \Rightarrow \rho_1 = \ldots = \rho_n = 0 \\
        \Leftrightarrow\ & \mu_1, \ldots, \mu_n\ \text{линейно независимы над}\ \R / \Z.
    \end{align*}
\end{proof}

\begin{lemma*}
    $\mu_1, \ldots, \mu_n \in \Z^k$ линейно независимы над~$\R$ тогда и~только тогда, когда линейно независимы над~$\Q$.
\end{lemma*}

\begin{proof}
    Импликация слева направа очевидна. Обратно, пусть $\mu_1, \ldots, \mu_n$ линейно независимы над~$\Q$.
    Рассмотрим их линейную комбинацию: $r_1 \mu_1 + \ldots + r_n \mu_n = 0$.

    Как известно, $\R$ является (бесконечномерным) векторным пространством над~$\Q$, поэтому, пользуясь аксиомой
    выбора, зафиксируем базис Гамеля $B$ для~$\R$ над~$\Q$. Разложим $r_i$ по~этому базису:
    $$
        r_i = q_i^1 b_1 + \ldots + q_i^s b_s,
    $$
    где $q_i^j \in \Q$ и~$b_j \in B$. Векторов $r_i$ конечное число, поэтому наборы базисных векторов $b_j$ для~них можно выбрать одинаковыми.

    Подставим разложение в~линейную комбинацию:
    \begin{align*}
        r_1 \mu_1 + \ldots + r_n \mu_n & = (q_1^1 b_1 + \ldots + q_1^s b_s) \mu_1 + \ldots + (q_n^1 b_1 + \ldots + q_n^s b_s) \mu_n \\
                                       & = (q_1^1 \mu_1 + \ldots + q_n^1 \mu_n) b_1 + \ldots + (q_1^s \mu_1 + \ldots + q_n^s \mu_n) b_s \\
                                       & = 0
    \end{align*}

    В~каждой проекции получаем нулевую рациональную линейную комбинацию чисел $b_j$. В~силу линейной независимости,
    все эти проекции равны нулю, поэтому равны нулю и~составленные из~них векторы: $q_1^j \mu_1 + \ldots + q_n^j \mu_n = 0$.

    Однако все $q_i^j$ рациональны, а~$\mu_i$ линейно независимы над~$\Q$, поэтому $q_i^j = 0$.
    Таким образом, $r_i = q_i^1 b_1 + \ldots + q_i^s b_s = 0 \cdot b_1 + \ldots + 0 \cdot b_s = 0$.
\end{proof}

\begin{lemma*}
    $\mu_1, \ldots, \mu_n \in \Z^k$ линейно независимы над~$\Q$ тогда и~только тогда, когда линейно независимы над~$\Z$.
\end{lemma*}

\begin{proof}
    Импликация слева направа снова очевидна. Обратно, пусть $q_1 \mu_1 + \ldots + q_n \mu_n = 0$, где $q_i \in \Q$.
    Выберем для~дробей $q_i$ общий знаменатель $q \in \N \setminus \{ 0 \}$. Числители обозначим как $p_i \in \Z$, то~есть $q_i = p_i / q$.

    В~таком случае $(p_1 \mu_1 + \ldots + p_n \mu_n) / q = 0$, откуда и~$p_1 \mu_1 + \ldots + p_n \mu_n = 0$.
    Все $p_i$~— целые числа, поэтому по~линейной независимости над~$\Z$ получаем, что $p_1 = \ldots = p_n = 0$.
    Таким образом, $q_i = p_i / q = 0 / q = 0$.
\end{proof}

Далее для~числа $d \in \Z$ и~вектора $x \in \Z^k$ под~$d \divides x$ понимаем, что $d \divides x_i$ для~всех $1 \leq i \leq k$
или, что эквивалентно, $d \divides \gcd(x_1, \ldots, x_n)$.

\begin{lemma*}
    Если $\mu_1, \ldots, \mu_n$ линейно независимы над~$\R / \Z$, то:
    $$
        \forall d \in \Z{:}\ \forall x \in \Z^n{:}\ d \divides x_1 \mu_1 + \ldots + x_n \mu_n \Rightarrow d \divides x.
    $$
\end{lemma*}

\begin{proof}
    Действительно, рассмотрим дроби $q_i = x_i / d \in \R / \Z$. Делимость $x_1 \mu_1 + \ldots + x_n \mu_n$ на~$d$ означает,
    что в~$\R / \Z$ выполнено $(x_1 \mu_1 + \ldots + x_n \mu_n) / d = 0$, то~есть $q_1 \mu_1 + \ldots + q_n \mu_n = 0$.
    Однако в~силу линейной независимости верно, что $q_1 = \ldots = q_n = 0$, а~это означает $d \divides x_i$ для~всех $1 \leq i \leq n$.
\end{proof}

Отметим простой критерий для~проверки неинъективности $\phi_\alpha$.

\begin{lemma*}
    Если $\phi_\alpha$ инъективно, то~$\forall j{:}\ \gcd(\alpha^j) = 1$.
\end{lemma*}

\begin{consequence*}
    Если $\exists j{:}\ \gcd(\alpha^j) \neq 1$, то~$\phi_\alpha$ не~является инъективным отображением.
\end{consequence*}

\begin{proof}
    Выбрав $x_j = 1$ и~$x_1 = \ldots = x_{j - 1} = x_{j + 1} = \ldots = x_k = 0$, получим, что:
    \begin{align*}
                         & \forall d \in \Z{:}\ \forall x \in \Z^k{:}\ d \divides x_1 \alpha^1 + \ldots + x_k \alpha^k \Rightarrow d \divides x \\
            \Rightarrow\ & \forall j{:}\ \forall d \in \Z{:}\ d \divides \alpha^j \Rightarrow d \divides 1 \\
        \Leftrightarrow\ & \forall j{:}\ \gcd(\alpha^j) = 1.
    \end{align*}

    Нетрудно видеть, что использование поля $\complex$ в~этой лемме \textit{существенно;} поскольку, например,
    отображение $t \mapsto (t^3, t^2)$ инъективно над~$\R$, но~не~над~$\complex$.
\end{proof}

Множество простых чисел обозначим как $\primes \subseteq \Z$.

\begin{lemma*}
    Для~произвольных целочисленных векторов $\mu_1, \ldots, \mu_n$ верно, что:
    \begin{align*}
                         & \forall d \in \Z{:}\ \forall x \in \Z^n{:}\ d \divides x_1 \mu_1 + \ldots + x_n \mu_n \Rightarrow d \divides x \\
        \Leftrightarrow\ & \forall p \in \primes{:}\ \forall \beta \in \N{:}\ \forall x \in \Z^n{:}\ p^\beta \divides x_1 \mu_1 + \ldots + x_n \mu_n \Rightarrow p^\beta \divides x \\
        \Leftrightarrow\ & \forall p \in \primes{:}\ \forall x \in \Z^n{:}\ p \divides x_1 \mu_1 + \ldots + x_n \mu_n \Rightarrow p \divides x
    \end{align*}
\end{lemma*}

\begin{proof}
    Импликации слева направо очевидны. Докажем импликации справа налево.

    Зафиксируем произвольное $d \in \Z$ и~некоторый вектор $x \in \Z^k$.
    Разложим $d$ на~простые множители: $d = p_1^{\beta_1} \ldots p_m^{\beta_m}$.

    Поскольку $x_1 \mu_1 + \ldots + x_n \mu_n$ делится на~$d$, то~оно делится и~на~каждое $p_i^{\beta_i}$.
    Пользуясь предпосылкой, получаем, что $p_i^{\beta_i} \divides x$ для~всех $1 \leq i \leq m$;
    но~тогда и~$d = p_1^{\beta_1} \ldots p_m^{\beta_m} \divides x$, что и~требовалось.

    Далее, зафиксируем степень $\beta \in \N$ и~простое число $p$.
    Поскольку $x_1 \mu_1 + \ldots + x_n \mu_n$ делится на~$p^\beta$,
    то~оно делится и~на~$p$, поэтому из~предпосылки следует, что $p \divides x$.

    Но~это означает, что $x_i / p$~— целые числа, а~$(x_1 / p) \mu_1 + \ldots + (x_n / p) \mu_n$ делится на~$p^{\beta - 1}$.
    Снова применив предпосылку, получим, что $p \divides x / p$.
    Повторив эту процедуру $\beta$ раз, окончательно заключим, что $p \divides x / p^{\beta - 1}$;
    но~это эквивалентно $p^k \divides x$.
\end{proof}

Последнее условие можно переписать как линейную независимость векторов $\mu_i$ над~полями $\Z / p\Z$ для~всех простых $p$,
что эквивалентно условию на~максимальность ранга матрицы $\mu$.

\begin{consequence*}
    Для~произвольных целочисленных векторов $\mu_1, \ldots, \mu_n$ верно, что:
    \begin{align*}
                         & \forall d \in \Z{:}\ \forall x \in \Z^n{:}\ d \divides x_1 \mu_1 + \ldots + x_n \mu_n \Rightarrow d \divides x \\
        \Leftrightarrow\ & \forall p \in \primes{:}\ \forall x \in (\Z / p\Z)^n{:}\ x_1 \mu_1 + \ldots + x_n \mu_n = 0 \Rightarrow x = 0 \\
        \Leftrightarrow\ & \forall p \in \primes{:}\ \mu_1, \ldots, \mu_n\ \text{линейно независимы над~полем}\ \Z / p\Z \\
        \Leftrightarrow\ & \forall p \in \primes{:}\ \rank_{\Z / p \Z}(\mu) = \max(n, k).
    \end{align*}
\end{consequence*}

Перед тем, как доказать достаточность, сформулируем необходимую для~доказательства теорему.

\begin{theorem*}
    $\alpha_i \in \Z^k$ порождают всю решётку $\Z^k$ тогда и~только тогда,
    когда миноры максимальной размерности матрицы $\alpha$ взаимно просты.
\end{theorem*}

Наконец, докажем упомянутое достаточное условие.

\begin{lemma*}
    Если $\phi_\alpha$ инъективно, то~$\alpha_i$ порождают всю решётку.
\end{lemma*}

\begin{proof}
    От~противного. Пусть $\alpha_i$ не~порождают решётку. Тогда, по~отмеченной теореме,
    миноры максимальной размерности матрицы $\alpha$ имеют общий делитель $d > 1$.

    Возьмём некоторый простой делитель $p$ числа $d$. Поскольку максимальные миноры
    делятся на~$d$, то~они делятся и~на~$p$; поэтому в~поле $\Z / p \Z$
    все максимальные миноры $\alpha$ равны нулю, что означает $\rank_{\Z / p \Z}(\alpha) < \max(n, k)$,
    но~это противоречит заключению из~следствия.
\end{proof}

Таким образом, следующая теорема также верна.

\begin{theorem*}
    $\phi_\alpha$ инъективно тогда и~только тогда, когда~$\alpha_i$ порождают всю решётку.
\end{theorem*}

\end{document}