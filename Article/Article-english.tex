\documentclass[twoside]{article}

\usepackage{sfuEng}


%%%%%%%%%%%%%%%%%%%%%  ЗДЕСЬ МОЖНО ПОДГРУЗИТЬ ДОПОЛНИТЕЛЬНЫЕ СТИЛЕВЫЕ ФАЙЛЫ, ЕСЛИ МАЛО УЖЕ ЗАГРУЖЕННЫХ %%%%%%%%%%%%%%%
\usepackage{} %%% и т.д.

%%%%%%%%%%%%%%%%%%%%%%%%%%%%%%%%%%%%%%%%%%%%%%%%%%%%%%%%%%%%%%%%%%%%%%%%%%%%%%

%%%%%%%%%%%%%%%%%%%%%%%%%% ЭТА ЧАСТЬ ЗАПОЛНЯЕТСЯ РЕДАКЦИЕЙ  %%%%%%%%%%%%%%
\setcounter{page}{258}

\god{2009}

\nomer{2(3)}


\pp{258--270}                 %%%% Диапазон страниц

\firstvar{10.05.2009}         %%%% Дата получения статьи

\lastvar{10.06.2009}          %%%% Дата получения последнего варианта

\toprint{20.06.2009}          %%%% Дата принятия к печати

 %%%%%%%%%%%%%%%%%%%%%%%%%  ЭТА ЧАСТЬ ЗАПОЛНЯЕТСЯ АВТОРОМ %%%%%%%%%%%%%%%%%

\udk{512.74}

 %%%%%%%%%% Название статьи

\tit{Algebraic subgroups of the complex torus}


%%%%%%%%%%%%%%%%%% Краткая аннотация

\abstract{We study monomial parameterizations of~algebraic subgroups of~the~torus
over an~arbitrary field and separately over the~field of~complex numbers. It is proved that every
monomial parameterization defines an~algebraic group. The~necessary and~sufficient conditions for
the~injectivity and existence of~such parameterizations are obtained.}

%%%%%%%%%%%%%%%%%% Ключевые слова
\keywords{algebraic subgroups,
monomial parameterization, complex algebraic torus.}



%%%%%%%%%% Название статьи на русском

\Rustit{Алгебраические подгруппы комплексного тора}

\authorRus{Николай А. Мишко}



%%%%%%%%%%%%%%%%%% Краткая аннотация на английском

\russianabstract{В~работе изучаются мономиальные параметризации алгебраических подгрупп тора
над произвольным полем и отдельно над полем комплексных чисел. Доказывается, что всякая
мономиальная параметризация определяет алгебраическую группу. Получены необходимые
и~достаточные условия инъективности и~существования такого рода параметризаций.}

%%%%%%%%%%%%%%%%%% Ключевые слова на английском

\russiankeywords{алгебраические подгруппы,
мономиальная параметризация, комплексный алгебраический тор.}


%%%%%%%%%%%%%%%%%%%%%%%%%%%%%%%%%%%%%%%%%%%%%%%%%%%%%%%%%%%%%%%%%%%%%%%

\begin{document}

\maketitBegin %% DON'T CHANGE!!!
%%%%%%%%%%%%%%%%%%%%%%%%%%%%%%%%%%  Авторы, адреса рабочие, электронные адреса%%%%%%%%%%%%
%%%%%%%%%%%%%%%%%%% Заполнять в скобках:\author{Имя}{Адрес}{e-mail} %%%%%%%%%%%%%%%%%%%%%%%%%%%%%
%%%%%%%%%%%%%%%%%%% адрес разбивать на строки c помощью \\ (см. образец)

\author{Nikolay A. Mishko}
{Siberian Federal University,\\
Krasnoyarsk, Russia} {siegmentationfault@yandex.ru}





 \maketitEnd  %% DON'T CHANGE!!


%%%%%%%%%%%%%%%%%%%%%%%%%%%%%%%%%%%%%%  ДАЛЕЕ ТЕКСТ СТАТЬИ АВТОРА %%%%%%%%%%%%%%%%%%%%%%%%%%%%%%%%%%
\newcommand{\p}{\partial}
\renewcommand{\geq}{\geqslant}
\renewcommand{\leq}{\leqslant}
\renewcommand{\baselinestretch}{1}
\newtheorem {theorem}{Theorem}[section]
\newtheorem {rem}{Remark}[section]

\renewcommand{\thetheorem}{\arabic{section}.\arabic{theorem}}


%%%%%%%%%%%%%%%%%%%%%%%%%%%%%%%%%%%%%%%%%%%%%%%%%%%%%%%%%%%%%%%%%%%%%%%%%%%%%%%%%%%%%%%%%%%%%%%%%%%%%%%%%

\textit{An~algebraic subgroup} of~a~group $G$ is a~subgroup endowed with an~algebraic variety structure,
i.e.~defined by~means of~a~system of~polynomial equations. A~natural example of~the~algebraic group is a~set
of~solutions of~a~system of~binomial equations:
$$
    z_1^{\alpha_1} z_2^{\alpha_2} \ldots z_n^{\alpha_n} = z_1^{\beta_1} z_2^{\beta_2} \ldots z_n^{\beta_n}.
$$

The~following theorem shows that such groups in~fact exhaust all algebraic varieties globally inheriting
the~group structure of~the~torus $(K^\times)^n$.

For~vectors $\alpha = (\alpha_1, \ldots, \alpha_n) \in \mathbb{Z}^n$ and~$z = (z_1, \ldots, z_n) \in G^n$ (where $G$ is~a~group)
we write $z^\alpha = z_1^{\alpha_1} \cdot z_2^{\alpha_2} \cdot \ldots \cdot z_n^{\alpha_n}$.

\medskip\noindent\textbf{Теорема} (Шмидт) \cite{Schm94}. \emph{Let $K$ be a~field. Every algebraic subgroup $H$
    of~the~group $(K^{\times})^n$ is~defined by~a~system of~some number $N$ of~binomial equations, namely,
    there are $N$ indices $\alpha_i, \beta_i \in \mathbb{Z}^n$ such that
    $$
        H = \{ z \in (K^{\times})^n\ |\ \forall 1 \leq i \leq N{:}\ z^{\alpha_i} = z^{\beta_i} \}.
    $$
}

Next we'll give a~self-contained (independent of~other major theorems) proof of~this theorem.
For~this purpose we need an~auxiliary statement.

\section*{1. Artin's theorem}
We denote the~set of~homomorphisms between groups $G$ and~$H$ by~$\mathrm{Hom}(G, H)$.

Let $G$ be a~group, $K$ be a~field, and $K^{\times}$ be its multiplicative group. Then an~arbitrary
homomorphism $f \in \mathrm{Hom}(G, K^{\times})$ is called \textit{a~character.} The~characters $f_1, f_2, \ldots, f_n$
are \textit{linearly independent} if
$$
    \forall \alpha_1, \alpha_2, \ldots, \alpha_n \in K{:}\ \alpha_1 f_1 + \alpha_2 f_2 + \ldots + \alpha_n f_n = 0 \Rightarrow \alpha_1 = \alpha_2 = \ldots = \alpha_n = 0.
$$

\noindent\textbf{Теорема} (Артин) \cite{Art48}.\emph{
    Any $n$ pairwise distinct characters are linearly independent.
}\medskip

\noindent{\it Proof.}
    Let us prove by induction on~the~number of~characters $n$.

    Take an~arbitrary character $f$. Since it is a~homomorphism, $f(1_G) = 1$
    where $1_G$ is an~identity element in~the~group $G$. But then if $\alpha f = 0$, then
    $\alpha = \alpha \cdot 1 = \alpha \cdot f(1_G) = 0$ what proves the~base case.

    Now let the~statement of~theorem be true for any $n$ distinct characters. Let us prove
    it for $n + 1$ characters. Let $\alpha_1 f_1 + \alpha_2 f_2 + \ldots + \alpha_n f_n + \alpha_{n + 1} f_{n + 1} = 0$.

    Fix an~arbitrary $y \in G$. The for any $x \in G$ we have:
    \begin{equation}\label{Artin:first}
        \alpha_1 f_1(yx) + \alpha_2 f_2(yx) + \ldots + \alpha_n f_n(yx) + \alpha_{n + 1} f_{n + 1}(yx) = 0
    \end{equation}

    Since all $f_i$ are homomorphisms, $f_i(yx) = f_i(y) f_i(x)$, and therefore:
    $$
        \alpha_1 f_1(y) f_1(x) + \alpha_2 f_2(y) f_2(x) + \ldots + \alpha_n f_n(y) f_n(x) + \alpha_{n + 1} f_{n + 1}(y) f_{n + 1}(x) = 0
    $$

    On the~other hand, for $x \in G$ it is also true that:
    $$
        \alpha_1 f_1(x) + \alpha_2 f_2(x) + \ldots + \alpha_n f_n(x) + \alpha_{n + 1} f_{n + 1}(x) = 0
    $$

    Multiplying this equation by~$f_{n + 1}(y)$, we'll get:
    \begin{equation}\label{Artin:second}
        \alpha_1 f_{n + 1}(y) f_1(x) + \alpha_2 f_{n + 1}(y) f_2(x) + \ldots + \alpha_n f_{n + 1}(y) f_n(x) + \alpha_{n + 1} f_{n + 1}(y) f_{n + 1}(x) = 0
    \end{equation}

    Subtracting \eqref{Artin:first} from \eqref{Artin:second}, we obtain:
    $$
        \alpha_1 (f_{n + 1}(y) - f_1(y)) f_1(x) + \ldots + \alpha_n (f_{n + 1}(y) - f_n(y)) f_n(x) = 0
    $$

    $x$ was chosen arbitrarily, so we get a~linear combination of~$n$ characters:
    $$
        \alpha_1 (f_{n + 1}(y) - f_1(y)) f_1 + \ldots + \alpha_n (f_{n + 1}(y) - f_n(y)) f_n = 0
    $$

    But then, using the~inductive hypothesis, $\alpha_i (f_{n + 1}(y) - f_i(y)) = 0$. Now, choosing $y$
    for each $i = 1, \ldots, n$ such that $f_{n + 1}(y) \neq f_i(y)$ (it's always possible because all $f_i$
    are pairwise distinct), we obtain that $\alpha_1 = \alpha_2 = \ldots = \alpha_n = 0$.

    Thus, given the~above, $\alpha_{n + 1} f_{n + 1} = \alpha_1 f_1 + \alpha_2 f_2 + \ldots + \alpha_n f_n + \alpha_{n + 1} f_{n + 1} = 0$.
    According to~the~base case again, $\alpha_{n + 1} = 0$.
\hfill$\Box$

\section*{2. Proof of Schmidt's theorem}

\noindent{\it Proof.}
    Let $I \subseteq \mathbb{Z}^n$ be a~finite set of~indices, and let $P_j(z) = \sum_{i \in I} a_{ji} z^i$ be the~$k$ polynomials
    defining the~subgroup:
    $$
        H = \{ z \in (K^{\times})^n\ |\ \forall 1 \leq j \leq k{:}\ P_j(z) = 0 \}.
    $$

    Since $z_1^{i} z_2^{i} = (z_1 z_2)^i$, the~mapping $z \mapsto z^i$ defines character $\chi_i \in \mathrm{Hom}(H, K^{\times})$.

    Consider an~equivalence relation $i \sim j \Leftrightarrow \chi_i = \chi_j$ on~the~set $I$. It partitions $I$
    into $m$ classes $I_1$, $I_2$, \textellipsis, $I_m$. Then for~all $i_1, i_2 \in I_j$ it is true that $\chi_{i_1} = \chi_{i_2}$.
    Let us denote this character corresponding to~each $I_k$ by~$\chi_k = \chi_{i_1} = \chi_{i_2}$.

    Combining like terms at~each~$\chi_k$ in~the~polynomials $P_j$, we obtain:
    $$
        P_j = \sum_{i \in I} a_{ji} \chi_i = \sum_{k = 1}^{m} \left( \sum_{i \in I_k} a_{ji} \right) \chi_k.
    $$

    But $P_j$ vanishes on~the~whole $H$, so:
    $$
        \sum_{k = 1}^{m} \left( \sum_{i \in I_k} a_{ji} \right) \chi_k = 0.
    $$

    All $\chi_k$ are pairwise distinct by~their definition, so Artin's theorem applies. We conclude that:
    $$
    \sum_{i \in I_k} a_{ji} = 0.
    $$

    Finally, let $N_k$ be~the~cardinality of~$I_k$, $I_k = \{i_{k, 1}, i_{k, 2}, \ldots, i_{k, N_k}\}$, and~$N = \sum_{k = 1}^m (N_k - 1)$.
    $I_k$ were taken such that the~characters $\chi_{i_{k, i}}$ and~$\chi_{i_{k, j}}$ coincide on~$H$ for fixed $k$ and any $i$ and $j$.
    In~other words, this means that any element $z \in H$ satisfies the~equations $z^{i_{k, i}} = z^{i_{k, j}}$ for all $1 \leq i \leq N_k$
    and~$1 \leq j \leq N_k$.

    Since the~equality is reflexive, symmetric and transitive, the~system of~all equations $z^{i_{k, i}} = z^{i_{k, j}}$
    is equivalent to~the~system of~$N$ equations composed of~consecutive indices:
    $$
        z^{i_{1, 1}} = z^{i_{1, 2}}, z^{i_{1, 2}} = z^{i_{1, 3}}, \ldots, z^{i_{1, N_1 - 1}} = z^{i_{1, N_1}}, z^{i_{2, 1}} = z^{i_{2, 2}}, \ldots, z^{i_{m, N_m - 1}} = z^{i_{m, N_m}}.
    $$
    We denote the~set of~solutions of~this system by~$A$.

    Let us show that $A$ coincides with $H$. Indeed, let $z \in H$. Then, as noted before,
    by~definition of~$I_k$ is is true that $z^{i_{k, j}} = \chi_k(z) = z^{i_{k, j + 1}}$;
    that is, $z \in A$ and~$H \subseteq A$.

    Conversely, let $z \in A$. Then by~definition of~$A$ we have $z^{i_{k, j_1}} = z^{i_{k, j_1 + 1}} = \ldots = z^{i_{k, j_2 - 1}} = z^{i_{k, j_2}}$
    for all $1 \leq j_1 \leq j_2 \leq N_k$. Choose a~representative $i_k \in I_k$ in~each~$I_k$
    and combine like terms:
    $$
        P_j(z) = \sum_{k = 1}^{m} \left( \sum_{i \in I_k} a_{ji} \right) z^{i_k} = \sum_{k = 1}^{m} 0 \cdot z^{i_k} = 0,
    $$
    what means $z \in H$, that is, $A \subseteq H$.

    Thus, $H = A$, so it suffices to take all the~consecutive indices from $I_k$ as $\alpha_i$ and $\beta_i$.
\hfill$\Box$\medskip

\section*{3. Injectivity of monomial parameterizations}

For~an abelian group $G$ and~vectors $\alpha_1, \ldots, \alpha_n \in \mathbb{Z}^k$ consider the~mapping
$\phi_\alpha(t) = (t^{\alpha_1}, t^{\alpha_2}, \ldots, t^{\alpha_n})$ from~$G^k$ to~$G^n$ where $\alpha$ is~a~matrix
with rows $\alpha_i$. Since $(t_1 \cdot t_2)^{\alpha_i} = t_1^{\alpha_i} \cdot t_2^{\alpha_i}$,
$\phi_\alpha$ is a~homomorphism. Further we will that if $K$ is a~field, then the~mapping $\phi_\alpha$
for the~group $G = K^\times$ defines \textit{a~parameterization} of~some algebraic group.

\medskip\noindent\textbf{Proposition.}\emph{
    $\phi_{E} = 1_{G^n}$ where $E$ is an~$n \times n$ identity matrix, $1_X$ is~an~identity function on~$X$.
}

\medskip\noindent\textbf{Proposition.}\emph{
    $
        \forall \alpha \in \mathbb{Z}^{k \times m}, \beta \in \mathbb{Z}^{m \times n}{:}\ \phi_\alpha \circ \phi_\beta = \phi_{\alpha \beta}
    $.
}\medskip

\noindent{\it Proof.}
    Indeed, consider $t \in G^n$. Then the~$i$-th component of~$\phi_\alpha(\phi_\beta(t))$
    is equal to $\phi_\beta^{\alpha_i}(t) = (t^{\beta_1})^{\alpha_i^1} \ldots (t^{\beta_m})^{\alpha_i^m}
                                          = t_1^{\beta_1^1 \alpha_i^1 + \ldots + \beta_m^1 \alpha_i^m}
                                            \ldots
                                            t_n^{\beta_1^n \alpha_i^1 + \ldots + \beta_m^n \alpha_i^m}
                                          = t^{(\alpha \beta)_i}$.
\hfill$\Box$\medskip

Together these two propositions constitute a~condition for functoriality. More precisely, consider the~category $\mathrm{Matr}(R)$ of~matrices over the~ring $R$
whose objects are the~natural numbers and the~arrows between the~numbers $m$ and $n$ are the~matrices $R^{n \times m}$,
and the~category $\mathrm{Grp}$ of~small groups with small groups as objects and homomorphisms between them as arrows.
Then the~functor $\phi : \mathrm{Matr}(\mathbb{Z}) \rightarrow \mathrm{Grp}$ is defined:
$$
    \phi = \begin{cases}
        k \mapsto G^k; \\
        \alpha \mapsto \phi_\alpha.
    \end{cases}
$$

As lemma~2 shows, in~fields of characteristic zero this functor is faithful.

\medskip\noindent\textbf{Lemma~1.}\emph{
    Let $K$ be a~field such that $\mathrm{char}(K) = 0$.
    Then $K^\times$ contains an~element of~infinite order.
}\medskip

\noindent{\it Proof.}
    Indeed, consider $2 = 1 + 1 \in K^\times$. Then for any $n > 0$:
    $$
        2^n - 1 = (1 + 1)^n - 1 = \sum_{k = 0}^n C^n_k 1^k 1^{n - k} - 1
                                = \sum_{k = 0}^n C^n_k \cdot 1 - 1
                                = \left(\sum_{k = 1}^n C^n_k \right) \cdot 1.
    $$

    On~the~right side we have the~sum of $\sum_{k = 1}^n C^n_k > 0$ units. Since $\mathrm{char}(K) = 0$,
    it is not equal to zero, that is, $2^n - 1 \neq 0$.
\hfill$\Box$\medskip

We denote \textit{a~standard basis} over~$\mathbb{Z}^k$ by~$e_i$, that is, a~vector with one on~the~$i$-th place
and zeroes on~the~others. We denote the~$j$-th component of~the~vector $\alpha_i$ by~$\alpha_i^j$
and the~vector consisting of~the~$j$-th components by~$\alpha^j = (\alpha_1^j, \alpha_2^j, \ldots, \alpha_n^j)$.

\medskip\noindent\textbf{Lemma~2.}\emph{
    $
        \mathrm{char}(K) = 0 \Rightarrow \forall \alpha, \beta \in \mathbb{Z}^{n \times k}{:} \ \phi_\alpha = \phi_\beta \Leftrightarrow \alpha = \beta
    $.
}\medskip

\noindent{\it Proof.}
    The~left-to-right implication is obvious.

    Conversely, let $\phi_\alpha = \phi_\beta$. Using lemma~1 we fix an~element $z \in K^\times$
    of~infinite multiplicative order. Then for all $1 \leq i \leq n$ it is true that
    $$
        (z^{\alpha_1^i}, \ldots, z^{\alpha_k^i}) = \phi_\alpha(z \cdot e_i) = \phi_\beta(z \cdot e_i) = (z^{\beta_1^i}, \ldots, z^{\beta_k^i}).
    $$

    Thus, $z^{\alpha_j^i} = z^{\beta_j^i}$, that is, $z^{\alpha_j^i - \beta_j^i} = 1$.
    $z$ has infinite multiplicative order, so $\alpha_j^i - \beta_j^i = 0$.
\hfill$\Box$\medskip

We denote a~\textit{$\mathbb{Z}$-linear span} of~$\alpha_1$, $\ldots$, $\alpha_n$
by~$\mathrm{Span}_\mathbb{Z}(\alpha_1, \ldots, \alpha_n) = \{ k_1 \alpha_1 + \ldots + k_n \alpha_n \ | \ k_1,\allowbreak \ldots,\allowbreak k_n \in \mathbb{Z} \}$.
We say that $\alpha_1$, $\ldots$, $\alpha_n$ span the~whole lattice $\mathbb{Z}^k$ if
$\mathrm{Span}_\mathbb{Z}(\alpha_1, \ldots, \alpha_n) = \mathbb{Z}^k$.

It is easy to~obtain a~necessary condition for the~injectivity of the mapping $\phi_\alpha$. For this purpose,
we note that the~following proposition is obvious.

\medskip\noindent\textbf{Proposition.}\emph{
    $\mathrm{Span}_\mathbb{Z}(\alpha_1, \ldots, \alpha_n) = \mathbb{Z}^k \Leftrightarrow \forall i{:}\ e_i \in \mathrm{Span}_\mathbb{Z}(\alpha_1, \ldots, \alpha_n)$.
}

\medskip\noindent\textbf{Lemma~3.}\emph{
    If vectors $\alpha_i \in \mathbb{Z}^k$ span the~whole lattice $\mathbb{Z}^k$, then $\phi_\alpha$ is injective.
}\medskip

\noindent{\it Proof.}
    $\phi_\alpha$ is a~homomorphism, so it suffices to~show that $\ker(\phi_\alpha) = \{1\}$.
    We'll take $t \in G^k$ such that $\phi_\alpha(t) = 1$ and prove that $t = 1$.

    Indeed, since $\alpha_i$ span the~whole lattice, each $e_j$ can be expressed as their linear combination:
    $
        e_j = b^j_1 \alpha_1 + \ldots + b^j_n \alpha_n.
    $

    $\phi_\alpha(t) = 1$ means that $t^{\alpha_i} = 1$ for all $1 \leq i \leq n$.
    Fix $1 \leq j \leq k$ and raise to~the power of~$b^j_i$ both sides of~this equation:
    $t^{b^j_i \alpha_i} = (t^{\alpha_i})^{b^j_i} = 1^{b^j_i} = 1$.
    Finally, multiply the~obtained equations:
    $
        t_j = t^{e_j} = t^{b^j_1 \alpha_1 + \ldots + b^j_n \alpha_n} = t^{b^j_1 \alpha_1} \ldots t^{b^j_n \alpha_n} = 1 \cdot \ldots \cdot 1 = 1.
    $
\hfill$\Box$\medskip

Now we will show that the~obtained necessary condition is also sufficient for~the~group $G = {\mathbb{C}^\times}$.

We will use the~existence of~\textit{Smith normal form} for~integer matrices \cite{Smth60}.
Let $\mathrm{diag}^{m \times n}_r(x_1,\allowbreak \ldots,\allowbreak x_r)$ be a~diagonal matrix $\mathrm{diag}(x_1,\allowbreak \ldots,\allowbreak x_r)$
augmented (or cut off) from the~bottom right by~zeroes to a~matrix of size $m \times n$.

\medskip\noindent\textbf{Theorem} (on~the~existence of~Smith normal form).\emph{
    $$
        \forall \alpha \in \mathbb{Z}^{m \times n}{:}\
        \exists \beta_1 \in \mathrm{GL}^m(\mathbb{Z}), \beta_2 \in \mathrm{GL}^n(\mathbb{Z}){:} \
        \exists \varepsilon_1, \ldots, \varepsilon_r \in \mathbb{Z} \setminus \{0\}{:} \
        \beta_1 \alpha \beta_2 = \mathrm{diag}^{m \times n}_r(\varepsilon_1, \ldots, \varepsilon_r),
    $$
    where $\varepsilon_1 \mid \varepsilon_2 \mid \ldots \mid \varepsilon_r$ (the so~called invariant factors) and~$r = \mathrm{rank}(\alpha)$.
}\medskip

Since the numbers $\varepsilon_1$, $\ldots$, $\varepsilon_r$ are chosen in~a~unique way up to~the~invertible element,
that is, up to $\pm 1$ in~case of~$\mathbb{Z}$, we can always choose them positive. For them we denote
the~matrix $\mathrm{diag}^{m \times n}_r(\varepsilon_1, \ldots, \varepsilon_r)$ by~$\mathrm{SNF}(\alpha)$.

In~addition, since $\phi$ defines the~functor, it is clear that $\phi_\beta$ is bijective if $\beta \in \mathrm{GL}^n(\mathbb{Z})$.
Indeed, $\phi_\beta \circ \phi_{\beta^{-1}} = \phi_{\beta \beta^{-1}} = \phi_E = 1_{(K^\times)^n}$
and~$\phi_{\beta^{-1}} \circ \phi_\beta = \phi_{\beta^{-1} \beta} = \phi_E = 1_{(K^\times)^n}$. Therefore,
the~following lemma holds.

\medskip\noindent\textbf{Lemma~4.}\emph{
    $\phi_\alpha : (\mathbb{C}^\times)^k \rightarrow (\mathbb{C}^\times)^n$ is injective if and only if
    $$
        \mathrm{SNF}(\alpha) = \mathrm{diag}^{n \times k}_k(1, 1, \ldots, 1).
    $$
}

\noindent{\it Proof.}
    Using the~theorem, let us take unimodular matrices $\beta_1 \in \mathrm{GL}^n(\mathbb{Z})$ and~$\beta_2 \in \mathrm{GL}^k(\mathbb{Z})$
    such that $\beta_1 \alpha \beta_2 = \mathrm{SNF}(\alpha)$.
    Then, $\alpha = \beta_1^{-1} \mathrm{SNF}(\alpha) \beta_2^{-1}$ and~$\phi_\alpha = \phi_{\beta_1^{-1} \mathrm{SNF}(\alpha) \beta_2^{-1}}
                                                                                    = \phi_{\beta_1^{-1}} \circ \phi_{\mathrm{SNF}(\alpha)} \circ \phi_{\beta_2^{-1}}$.

    Since, by~the~remark above, $\phi_{\beta_1^{-1}}$ and~$\phi_{\beta_2^{-1}}$ are bijective, $\phi_\alpha$ is injective if and only if
    $\phi_{\mathrm{SNF(\alpha)}}$ is injective.

    $\phi_{\mathrm{SNF(\alpha)}}(t_1, \ldots, t_k) = (t_1^{\varepsilon_1}, \ldots, t_r^{\varepsilon_r}, 1, \ldots, 1)$, but $t \mapsto t^\varepsilon$
    is injective over $\mathbb{C}^\times$ only if $\varepsilon = \pm 1$ (otherwise $t^\varepsilon = t^\varepsilon \cdot 1 = t^\varepsilon \cdot u^\varepsilon = (tu)^\varepsilon$
    where $u \neq 1$ is~a~nontrivial $\varepsilon$-th root of~unity).
    By~choice of~signs, $\phi_{\mathrm{SNF(\alpha)}}$ is injective only if $\varepsilon_1 = \ldots = \varepsilon_r = 1$
    and~$r = k \leq n$.
\hfill$\Box$\medskip

Let us formulate one more auxiliary theorem \cite{TsikhSad14}.

\medskip\noindent\textbf{Theorem~1.}\emph{
    Let $\alpha \in \mathbb{Z}^{n \times k}$. The~following statements are equivalent:
    \begin{enumerate}
        \item The~rows of~the~matrix $\alpha$ span the~whole lattice $\mathbb{Z}^k$.
        \item The~minors of~the~maximum order of~the~matrix $\alpha$ are coprime.
        \item $\mathrm{SNF}(\alpha) = \mathrm{diag}^{n \times k}_k(1, \ldots, 1)$.
    \end{enumerate}
}

Thus, from lemma~4 and~theorem~1 we obtain a~necessary and~sufficient condition for~the~injectivity of~$\phi_\alpha$.

\medskip\noindent\textbf{Theorem.}\emph{
    $\phi_\alpha$ is injective over the~field $K = \mathbb{C}$ if and only if $\alpha_i$ span the~whole lattice $\mathbb{Z}^k$.
}

\section*{4. Monomial parameterizability of~torus subgroups}

As in~the~case of~the~theory of~curves and surfaces, we say that a~subgroup of~an~algebraic torus
is \textit{parameterizable} if it can be expressed as an~image of~some mapping. Similarly, a~subgroup
is \textit{monomially parameterizable} if it can be expressed as an~image of~$\phi_\alpha$ for some matrix $\alpha$.

Since $\phi_\alpha$ is the~homomorphism, its image $\mathrm{Im}(\phi_\alpha)$ is a~subgroup in~$(K^\times)^n$.
Next we'll study two questions: whether $\mathrm{Im}(\phi_\alpha)$ is an~algebraic subgroup and whether
any algebraic subgroup is expressed by~some $\phi_\alpha$.

For this, first of all, we note that the~system of binomial equations $z^{\beta'_i} = z^{\beta''^i}$ is equivalent
to~the system $z^{\beta'_i - \beta''_i} = 1$, that is, exactly the~kernel of the~operator $\phi_{\beta'_i - \beta''_i}$;
therefore, the~first question is equivalent to whether a~given homomorphism $\phi_\alpha$ can be extended
to~the~exact sequence $(K^\times)^k \xrightarrow[]{\phi_\alpha} (K^\times)^n \xrightarrow[]{\phi_\beta} (K^\times)^m$.

\medskip\noindent\textbf{Lemma~5.}\emph{
    Let $G$ be a~group, $H$ be an~abelian group, $f \in \mathrm{Hom}(G, H)$, $g \in \mathrm{Hom}(H, G)$, and~$g \circ f = 1_G$.
    Then the~sequence $G \xrightarrow[]{f} H \xrightarrow[]{f \circ g - 1_H} H$ is exact.
}\medskip

\noindent{\it Proof.}
    Indeed, $(f \circ g - 1_H) \circ f = f \circ g \circ f - f = f - f = 0$, that is, $\mathrm{Im}(f) \subseteq \ker(f \circ g - 1_H)$.
    Conversely, let $h \in \ker(f \circ g - 1_H)$. Then $f(g(h)) - h = 0 \Leftrightarrow h = f(g(h))$, that is, $h \in \mathrm{Im}(f)$
    and $\ker(f \circ g - 1_H) \subseteq \mathrm{Im}(f)$.
\hfill$\Box$\medskip

\medskip\noindent\textbf{Lemma~6.}\emph{
    $\phi_\alpha$ for $\alpha \in \mathbb{Z}^{n \times k}$ can be expressed as some image $\mathrm{Im}(\phi_\beta)$
    if the~mappings $g \mapsto g^{\varepsilon_i}$ are surjective in~$G$ for all $\varepsilon_i$ из~$\mathrm{SNF}(\alpha)$.
}\medskip

\noindent{\it Proof.}
    Let us again write $\alpha$ as $\alpha = \beta_1^{-1} \varepsilon \beta_2^{-1}$ where
    $
        \varepsilon = \mathrm{diag}^{n \times k}_r(\varepsilon_1, \ldots, \varepsilon_r).
    $

    Consider $\delta = \mathrm{diag}^{n \times r}_r(1, \ldots, 1)$ и~$\alpha' = \beta^{-1} \delta$.
    $\phi_{\alpha'}$ is injective since it is a~composition of~injective functions. On~the~other hand, it is obvious that:
    $$
        \mathrm{Im}(\phi_\alpha) = \phi_{\beta_1^{-1}} (\phi_\varepsilon (\phi_{\beta_2^{-1}} (G^k)))
                                 = \phi_{\beta_1^{-1}} (\phi_\varepsilon (G^k))
                                 = \phi_{\beta_1^{-1}} (\phi_{\delta} (G^r))
                                 = \mathrm{Im}(\phi_{\alpha'}).
    $$

    It is easy to see that $\beta' = \delta^{\top} \beta$ is a~left inverse of~the~matrix $\alpha'$,
    and hence $\phi_{\beta'}$ is a~left inverse of~$\phi_{\alpha'}$. Since $(\phi_{\alpha'} \circ \phi_{\beta'}) \phi_{-E} = \phi_{\alpha' \beta' - E}$,
    we only need to~apply lemma~5.
\hfill$\Box$\medskip

\medskip\noindent\textbf{Theorem~2.}\emph{
    Any parameterization $\phi_\alpha$ for $\alpha \in \mathbb{Z}^{n \times k}$ defines an~algebraic subgroup
    of~$(K^\times)^n$ if the~field $K$ is algebraically closed.
}\medskip

The~answer to~the~second question is somewhat more complicated. For example, the~equation $z^n = 1$ defines an~algebraic
group for any $n \in \mathbb{Z}$ consisting of~$n$~points on~$\mathbb{C}^\times$; but if $n \neq 0, \pm 1$
it is easy to see that it cannot be parameterized by~any $\phi_\alpha$.

Indeed, notice that the~group $\mathrm{Im}(\phi_\alpha)$ is isomorphic to~the~group $(\mathbb{C}^\times)^r$.
In~the~proof of~lemma~6 we saw that for any $\phi_\alpha$ we can construct an~injective $\phi_{\alpha'} : (\mathbb{C}^\times)^r \rightarrow (\mathbb{C}^\times)^n$
such that $\mathrm{Im}(\phi_\alpha) = \mathrm{Im}(\phi_{\alpha'})$; but then $\phi_{\alpha'}$ defines the~desired isomorphism.
Thus, the~following proposition holds.

\medskip\noindent\textbf{Proposition.}\emph{
    For any matrix $\alpha \in \mathbb{Z}^{n \times k}$ the~group $\mathrm{Im}(\phi_\alpha)$ is isomorphic to an~algebraic torus
    of~dimension at mosk $k$.
}\medskip

Moreover, since the~mapping $\phi_{\alpha'}$ is polynomial, it is, in particular, continuous in~the~standard topology
on~$K = \mathbb{C}$. The~set $(\mathbb{C}^\times)^r$ is connected, and, as we know, the~continuous image
of~a~connected set is connected, so next proposition is also true.

\medskip\noindent\textbf{Proposition.}\emph{
    $\mathrm{Im}(\phi_\alpha) \subseteq (\mathbb{C}^\times)^n$ is connected.
}\medskip

Now notice that the~group $z^n = 1$ is connected exactly when $n = 0, \pm 1$.
This indicates that the~criterion for the~existence of a~monomial parameterization
for~an~algebraic torus subgroup is connectedness.

We denote the~group of~$n$-th root of unity in~the~field $K$ (given by~the~equation $z^n = 1$) by~$\omega_K(n) \subseteq K^\times$.
For~a~vector $x \in \mathbb{Z}^k$ we denote $\omega_K(x) = \omega_K(x_1) \times \ldots \times \omega_K(x_k) \subseteq (K^\times)^k$.
For brevity, we will write $\omega(x) = \omega_\mathbb{C}(x)$. It is known that the~group $\omega(x)$ is isomorphic
to~the~group $\mathbb{Z} / x_1 \mathbb{Z} \times \ldots \times \mathbb{Z} / x_k \mathbb{Z}$.

To~prove the~criterion described before, consider the~number $\Pi(\alpha) = |\omega_K(\varepsilon)|$
where $|G|$ is the~order of~the~group $G$. $|G \times H| = |G| |H|$, so $\Pi(\alpha) = |\omega_K(\varepsilon_1)| \cdot \ldots \cdot |\omega_K(\varepsilon_r)|$.
Since $\omega(n)$ is isomorphic to~$\mathbb{Z} / n \mathbb{Z}$, in~case of~the~field $K = \mathbb{C}$ it is also true that
$\Pi(\alpha) = \varepsilon_1 \cdot \ldots \cdot \varepsilon_r$.

\medskip\noindent\textbf{Lemma~7.}\emph{
    If $\Pi(\beta) = 1$, then there is $\alpha$ such that $\ker(\phi_\beta) = \mathrm{Im}(\phi_\alpha)$.
}\medskip

\noindent{\it Proof.}
    Let $H = \ker(\phi_\beta)$. Without loss of generality, we assume that the~rows of~the~matrix $\beta$ are linearly independent
    over $Z$ (it is clear that the~rows expressed as a~linear combination of~the~other rows can be removed from the~matrix $\beta$
    without changing the~kernel).

    Let us write the~Smith normal form for $\beta$: $\beta = \beta_1^{-1} \varepsilon \beta_2^{-1}$.
    Since, according to~the~remark above, $\beta$ has full rank, the~matrix $\varepsilon$ has no zero rows.
    Furthermore, $\Pi(\beta) = 1$, so $|\omega_K(\varepsilon_i)| = 1$, that is, the~equations $z^{\varepsilon_i} = 1$
    have only $z = 1$ as a~solution.

    The~kernel of~$\varepsilon$ is given by~vectors of~the~form $(0,\allowbreak \ldots,\allowbreak 0,\allowbreak t_{k + 1},\allowbreak \ldots, t_{n})$.
    Consider a~matrix $\delta \in \mathbb{Z}^{n \times (n - k)}$ corresponding to~the~linear operator
    $$
        (t_1, \ldots, t_{n - k}) \mapsto (0, \ldots, 0, t_1, \ldots, t_{n - k}).
    $$

    Let also $\alpha = \beta_2 \delta$. We'll prove that $\mathrm{Im}(\phi_\alpha) = \ker(\phi_\beta)$.

    Indeed, by~definition of~$\delta$ it holds that $\delta$, $\varepsilon \delta = 0$, so $\phi_{\beta} \circ \phi_{\alpha} = \phi_{\beta_1^{-1} \varepsilon \beta_2^{-1} \beta_2 \delta} = \phi_{\beta_1^{-1} \varepsilon \delta} = \phi_{0} = 1$,
    that is, $\mathrm{Im}(\phi_{\alpha}) \subseteq \ker(\phi_{\beta})$.

    Conversely, let $z \in \ker(\phi_\beta)$. Then $\phi_{\beta_1^{-1}} (\phi_{\varepsilon \beta_2^{-1}}(z)) = \phi_\beta(z) = 1$,
    so $\phi_{\varepsilon}(\phi_{\beta_2^{-1}}(z)) = \phi_{\varepsilon \beta_2^{-1}}(z) = \phi_{\beta_1}(1) = 1$.
    From the~form of~the~matrix $\varepsilon$ we obtain that
    $
        \phi_{\beta_2^{-1}}(z) = (1,\allowbreak \ldots,\allowbreak 1,\allowbreak t_{k + 1},\allowbreak \ldots, t_n) = \phi_\delta(t_{k + 1}, \ldots, t_n)
    $; that is, $z = \phi_{\beta_2}(\phi_\delta(t_{k + 1}, \ldots, t_n)) = \phi_\alpha(t_{k + 1}, \ldots, t_n)$ for~some $t_j$.
    Thus, $z \in \mathrm{Im}(\phi_\alpha)$ and $\ker(\phi_\beta) \subseteq \mathrm{Im}(\phi_\alpha)$.
\hfill$\Box$\medskip

\medskip\noindent\textbf{Теорема 3.}\emph{
    The~number of~connected components of~$\ker(\phi_\beta) \subseteq ({\mathbb{C}^\times})^n$ is equal to~$\Pi(\beta)$.
}\medskip

\noindent{\it Proof.}
    Consider again the~Smith normal form of~$\beta$: $\beta$ = $\beta_1^{-1} \varepsilon \beta_2^{-1}$.
    $\phi_{\beta_1^{-1}}$ is isomorphism, so $\ker(\phi_{\beta}) = \ker(\phi_{\varepsilon \beta_2^{-1}})$.

    We denote the~rows of~$\beta_2^{-1}$ by~$b_i$. For a~vector $u \in \omega(\varepsilon)$ consider
    the~set $H_u = \{z \in ({\mathbb{C}^\times})^n \ | \ \forall 1 \leq i \leq r{:}\ z^{b_i} = u_i\}$.
    The~condition $\phi_\varepsilon(\phi_{\beta_2^{-1}}(z)) = 1$ is obviously equivalent to~the~condition
    $\exists u \in \omega(\varepsilon){:}\ z \in H_u$. Clearly, the~sets $H_u$ are disjunctive,
    so $\ker(\phi_{\beta})$ decomposes into a~disjunctive union:
    $
        \ker(\phi_{\beta}) = \bigsqcup_{u \in \omega(\varepsilon)} H_u.
    $
    There are exactly $\Pi(\beta)$ vectors $u \in \omega(\varepsilon)$, so it suffices to~show that
    each component $H_u$ is connected.

    Since, by~definition, $H_1 = \ker(\phi_{\delta \beta_2^{-1}})$ where $\delta = \mathrm{diag}^{k \times n}_r(1, \ldots, 1)$
    and $\Pi(\delta \beta_2^{-1}) = 1$, the~component $H_1$ is connected by~the~remark.

    Fix $u$ and consider the~matrix $\tau = \mathrm{diag}(1 / u_1, \ldots, 1 / u_r, 1, \ldots, 1)$
    and~the~mapping $\psi = \phi_{\beta_2} \circ \phi_\tau \circ \phi_{\beta_2^{-1}}$.
    $\psi$ is a~continuous bijection, moreover, $\psi^{-1} = \phi_{\beta_2} \circ \phi_{\tau^{-1}} \circ \phi_{\beta_2^{-1}}$.

    By~definition, $\phi_{\beta_2^{-1}}(\psi(z)) = \phi_{\tau \beta_2^{-1}}(z)$. So if $z \in H_u$,
    then $\psi(z)^{b_i} = z^{b_i} / u_i = u_i / u_i = 1$. Conversely, if $z \in H_1$, then $\psi^{-1}(z)^{b_i} = u_i z^{b_i} = u_i$.
    Thus, $\psi(H_u) = H_1$, that is, $H_u$ is homeomorphic to~$H_1$, but $H_1$ is connected.
\hfill$\Box$\medskip

As we saw before, any algebraic subgroup $H$ of the~torus $(K^\times)^n$ can be expressed as $\ker(\phi_\beta)$
for some $\beta$. Notice that $\Pi(\beta)$ does not depend on~the~choice of~$\beta$ for~the~group $H$.

\medskip\noindent\textbf{Theorem~4.}\emph{
    If $\mathrm{char}(K) = 0$ and $\ker(\phi_{\beta}) = \ker(\phi_{\beta'})$, then $\Pi(\beta) = \Pi(\beta')$.
}\medskip

\noindent{\it Proof.}
    Let us write down the~Smith normal forms: $\beta = \beta_1^{-1} \varepsilon \beta_2^{-1}$
    and~$\beta' = {\beta'}_1^{-1} \varepsilon' {\beta'}_2^{-1}$. So we have: $\ker(\phi_{\varepsilon \beta_2^{-1}}) = \ker(\phi_{\beta}) = \ker(\phi_{\beta'}) = \ker(\phi_{\varepsilon' {\beta'}_2^{-1}})$.

    Let $\varepsilon = \mathrm{diag}^{k \times n}_r(\varepsilon_1, \ldots, \varepsilon_r)$
    and~$\varepsilon' = \mathrm{diag}^{k' \times n}_{r'}(\varepsilon'_1, \ldots, \varepsilon'_r)$.
    We write $\delta = \mathrm{diag}^{k \times n}_r(1, \ldots, 1)$ and~$\delta' = \mathrm{diag}^{k' \times n}_{r'}(1, \ldots, 1)$.

    Since $\Pi(\delta \beta_2^{-1}) = \Pi(\delta' {\beta'}_2^{-1}) = 1$, both groups are parameterized
    according to~lemma~7 by~some $\phi_\alpha$ и~$\phi_{\alpha'}$ respectively.

    $\mathrm{Im}(\phi_\alpha) = \ker(\phi_{\delta \beta_2^{-1}}) \subseteq \ker(\phi_{\varepsilon \beta_2^{-1}}) = \ker(\phi_{\varepsilon' {\beta'}_2^{-1}})$,
    so $\phi_{\varepsilon' {\beta'}_2^{-1}} \circ \phi_\alpha = 1$. Using lemma~2, $\varepsilon' {\beta'}_2^{-1} \alpha = 0$.
    Multiplying both sides of~the~equality by~the~matrix $\mathrm{diag}(1 / \varepsilon'_1, \ldots, 1 / \varepsilon'_r, 1, \ldots, 1)$
    one obtains that $\delta' {\beta'}_2^{-1} \alpha = 0$; but this means that $\ker(\phi_{\delta \beta_2^{-1}}) = \mathrm{Im}(\phi_\alpha) \subseteq \ker(\phi_{\delta' {\beta'}_2^{-1}})$.
    Similarly we get the~converse inclusion. Thus, $\ker(\phi_{\delta \beta_2^{-1}}) = \ker(\phi_{\delta' {\beta'}_2^{-1}})$.

    Consider the~quotient group $\ker(\phi_\beta) / \ker(\phi_{\delta \beta_2^{-1}}) = \ker(\phi_{\beta'}) / \ker(\phi_{\delta' {\beta'}_2^{-1}})$,
    also called \textit{the~component group.} The~mapping $\phi_{\delta \beta_2^{-1}} : \ker(\phi_\beta) \rightarrow \omega_K(\varepsilon) \times \{ 1 \} \times \ldots \times \{ 1 \}$
    induces an~injective homomorphism from the~quotient. In addtion, in~the~theorem~3 the~bijection between the~components $H_u$
    was constructed (it generalises unchanged to~the~case of~an~arbitrary field $K$), and hence they are all nonempty.
    This means that $\phi_{\delta \beta_2^{-1}}$ is surjective, so the~induced mapping is also surjective.
    The~reasoning is similar for~the~group $\omega_K(\varepsilon')$. Thus, we have a~chain of~isomorphisms:
    $$
        \omega(\varepsilon) \cong \ker(\phi_\beta) / \ker(\phi_{\delta \beta_2^{-1}}) = \ker(\phi_{\beta'}) / \ker(\phi_{\delta' {\beta'}_2^{-1}}) \cong \omega(\varepsilon').
    $$

    Isomorphic groups have the~same order, and hence $\Pi(\beta) = |\omega_K(\varepsilon)| = |\omega_K(\varepsilon')| = \Pi(\beta')$.
\hfill$\Box$\medskip

Finally, for any algebraic subgroup $H$ of~the~torus we can define a~number $\Pi(H)$ equal to~$\Pi(\beta)$
for any $\beta$ such that $H = \ker(\phi_\beta)$.

Now for any algebraic subgroup $H \subseteq (\mathbb{C}^\times)^n$ we defined \textit{the~identity component} $H^\circ$
as a~connected component containing an~identity element of~the~group $H$. It follows from theorem~3
that $\Pi(H^\circ) = 1$. Thus, the~following statements are proved.

\medskip\noindent\textbf{Theorem~5.}\emph{
    An~algebraic subgroup $H$ of~the~torus $(K^\times)^n$ is parameterizable if and only if $\Pi(H) = 1$.
}

\medskip\noindent\textbf{Consequence.}\emph{
    An~algebraic subgroup $H$ of~the~torus $(\mathbb{C}^\times)^n$ is parameterizable if and only if it is connected.
}

\medskip\noindent\textbf{Consequence.}\emph{
    Every algebraic subgroup $H$ of~the~torus $(\mathbb{C}^\times)^n$ contains a~parameterizable subgroup $H^\circ$ of~the~same dimension.
}\medskip

\noindent{\it Proof.}
    All components of~$H$ are homeomorphic, what immediately proves the~theorem.
\hfill$\Box$\medskip

\section*{5. Linear independence over an~abelian group}

In~any abelian group $G$ there is naturally defined multiplication by~integers.
For a~vector $\alpha = (\alpha_1, \ldots, \alpha_k) \in \mathbb{Z}^k$ and~element $g \in G$
by~$g \alpha$ we mean the~vector $(\alpha_1 g, \ldots, \alpha_k g) \in G^k$. We say that
the~collection of~vectors $\alpha_1, \ldots, \alpha_n \in \mathbb{Z}^k$ is \textit{linearly independent
over an~abelian group} if
$$
    \forall g_1, \ldots, g_n \in G{:}\ g_1 \alpha_1 + \ldots + g_n \alpha_n = 0 \Rightarrow g_1 = \ldots = g_n = 0.
$$

We see that the~linear independence of~vectors from $\mathbb{Z}^k$ over the~additive group of~the~field $\mathbb{R}$
is equivalent to~the~ordinary linear independence in~the~vector space $\mathbb{R}^k$. However, for example,
$\mathbb{R} / 2 \pi \mathbb{Z}$, as it is known, has no ring structure, so it makes no sense
to~talk about the~$\mathbb{R} / 2 \pi \mathbb{Z}$-module $(\mathbb{R} / 2 \pi \mathbb{Z})^k$,
as well as about linear independence in it.

Let us immediately show how this definition is related to~the~studied parameterizations $\phi_\alpha$.

\medskip\noindent\textbf{Theorem~6.}\emph{
    $\phi_\alpha : G^k \rightarrow G^n$ is injective if and only if $\alpha^j$ are linearly independent over the~group $G$.
}\medskip

\noindent{\it Proof.} It follows directly from the~fact that injectivity of~a~homomorphism is equivalent to the~triviality of~its kernel.
\hfill$\Box$\medskip

Next, we need two general lemmas.

\medskip\noindent\textbf{Lemma~8.}\emph{
    Let $G$ and~$H$ be abelian groups. $\mu_1,\allowbreak \ldots,\allowbreak \mu_n \in \mathbb{Z}^k$ are linearly independent
    over~$G \times H$ if and only if the are linearly independent over~$G$ and~over~$H$.
}\medskip

\noindent{\it Proof.}
    The~elements of~$G \times H$ can be expressed as pairs $(g, h)$ where $g \in G$ and~$h \in H$,
    so linear independence over~$G \times H$ can be written as:
    $$
        \forall (g_1, h_1) \ldots, (g_n, h_n) \in G \times H{:}\ (g_1, h_1) \mu_1 + \ldots + (g_n, h_n) \mu_n = 0 \Rightarrow (g_1, h_1) = \ldots = (g_n, h_n) = 0.
    $$
    Furthermore, it is clear that
    $$
        (g_1, h_1) \mu_1 + \ldots + (g_n, h_n) \mu_n = 0 \Leftrightarrow g_1 \mu_1 + \ldots + g_n \mu_n = 0 \wedge h_1 \mu_1 + \ldots + h_n \mu_n = 0,
    $$
    as well as
    $$
        (g_1, h_1) = \ldots = (g_n, h_n) = 0 \Leftrightarrow g_1 = \ldots = g_n = 0 \wedge h_1 = \ldots = h_n = 0.
    $$

    By~fixing $g_1 = \ldots = g_n = 0$ and~then $h_1 = \ldots = h_n = 0$, we'll obtain the~left-to-right
    implication. The~right-to-left implication is obvious.
\hfill$\Box$\medskip

\medskip\noindent\textbf{Lemma~9.}\emph{
    Let $G$ and~$H$ be abelian groups, $f : G \rightarrow H$ be an~isomorphism. Then $\mu_1, \ldots, \mu_n \in \mathbb{Z}^k$
    are linearly independent over~$G$ if and only if they are linearly independent over~$H$.
}\medskip

\noindent{\it Proof.}
    Since $f$ is an~isomorphic, we obtain a~chain of~equivalences:
    \begin{align*}
                       &\ \forall g_1, \ldots, g_n \in G{:}\ g_1 \alpha_1 + \ldots + g_n \alpha_n = 0 \Rightarrow g_1 = \ldots = g_n = 0 \\
        \Leftrightarrow&\ \forall g_1, \ldots, g_n \in G{:}\ f(g_1 \alpha_1 + \ldots + g_n \alpha_n) = 0 \Rightarrow g_1 = \ldots = g_n = 0 \\
        \Leftrightarrow&\ \forall g_1, \ldots, g_n \in G{:}\ f(g_1) \alpha_1 + \ldots + f(g_n) \alpha_n = 0 \Rightarrow f(g_1) = \ldots = f(g_n) = 0 \\
        \Leftrightarrow&\ \forall h_1, \ldots, h_n \in H{:}\ h_1 \alpha_1 + \ldots + h_n \alpha_n = 0 \Rightarrow h_1 = \ldots = h_n = 0.
        \tag*{$\Box$}
    \end{align*}
{}

Applying the~lemmas, we obtain the~injectivity criterion for~the~case $K = \mathbb{C}$.

\medskip\noindent\textbf{Consequence.}\emph{
    $\phi_\alpha$ is injective if and only if $\alpha^j$ are linearly independent over the~groups $\mathbb{R}$ and~$\mathbb{R} / 2 \pi \mathbb{Z}$.
}\medskip

\noindent{\it Proof.}
    Let $S^1 = \{ z \in \mathbb{C}\ | \ |z| = 1 \}$ be a~circle group, a~subgroup in~${\mathbb{C}^\times}$.
    From the~trigonometric form $z = re^{i \theta}$ we see that the~group ${\mathbb{C}^\times}$ is isomorphic
    to~the~product $\mathbb{R}_{> 0}^\times \times S^1$. The~mapping $t \mapsto e^t$ defines the~isomorphism
    of~the~groups $\mathbb{R}$ and~$\mathbb{R}_{> 0}^\times$, and the~mapping $\theta \mapsto e^{i\theta}$
    defines the~isomorphism between $\mathbb{R} / 2 \pi \mathbb{Z}$ and~$S^1$; this proves a~corollary
    by~virtue of~the~previous two lemmas.
\hfill$\Box$\medskip

\medskip\noindent\textbf{Consequence.}\emph{
    If $\phi_\alpha$ is injective over~$K = \mathbb{C}$, then~$\mathrm{rank}(\alpha) = k$.
}\medskip

\noindent{\it Proof.}
    As noted before, linear independence over~$\mathbb{R}$ is equivalent to linear independence
    in~the~vector space $\mathbb{R}^k$ over~$\mathbb{R}$, and this in turn is equivalent to having full rank.
\hfill$\Box$\medskip

stub stub stub stub stub stub stub stub stub stub stub stub stub stub stub stub stub stub stub stub stub stub stub stub stub stub stub stub
stub stub stub stub stub stub stub stub stub stub stub stub stub stub stub stub stub stub stub stub stub stub stub stub stub stub stub stub
stub stub stub stub stub stub stub stub stub stub stub stub stub stub stub stub stub stub stub stub stub stub stub stub stub stub stub stub
stub stub stub stub stub stub stub stub stub stub stub stub stub stub stub stub stub stub stub stub stub stub stub stub stub stub stub stub
stub stub stub stub stub stub stub stub stub stub stub stub stub stub stub stub stub stub stub stub stub stub stub stub stub stub stub stub
stub stub stub stub stub stub stub stub stub stub stub stub stub stub stub stub stub stub stub stub stub stub stub stub stub stub stub stub
stub stub stub stub stub stub stub stub stub stub stub stub stub stub stub stub stub stub stub stub stub stub stub stub stub stub stub stub
stub stub stub stub stub stub stub stub stub stub stub stub stub stub stub stub stub stub stub stub stub stub stub stub stub stub stub stub
stub stub stub stub stub stub stub stub stub stub stub stub stub stub stub stub stub stub stub stub stub stub stub stub stub stub stub stub
stub stub stub stub stub stub stub stub stub stub stub stub stub stub stub stub stub stub stub stub stub stub stub stub stub stub stub stub

\medskip

\emph{This work is supported by~the~Krasnoyarsk Mathematical Center and financed
by~the~Ministry of~Science and~Higher Education of~the~Russian Federation (Agreement No. 075-02-2024-1429).}

\bigskip

\begin{thebibliography}{9}

\bibitem{Schm94} Schmidt, W.~M. Heights of points on subvarieties of $\mathbb{G}^n_m$~/ W.~M.~{Sch\-midt}~//
London Mathematical Society Lecture Note Series. Issue~235: Number Theory.
S\'eminaire de~th\'eorie des~nombres de~Paris 1993–94.~-- 1996.~-- P.~157--\allowbreak187.

\bibitem{Art48} Artin, Е. Galois Theory~/ E.~Artin; London~: University of~Notre Dame, 1942, 1944.~-- 96~p.~-- ISBN: 978-0-486-62342-9.

\bibitem{Smth60} Smith, H.J.S. On~systems of~linear indeterminate equations and~{con\-gru\-ences}. Philos.~/ H.J.S.~Smith~//
Philosophical Transactions of~the~Royal Society.~-- 1861.~-- Vol.~151.~-- P.~293--326.

\bibitem{TsikhSad14} Sadykov, T.~M. Multivariate Hypergeometric and~Algebraic Functions~/ T.~M.~Sadykov, A.~K.~Tsikh;
Moscow~: Nauka, 2014.~-- 408~p.~-- ISBN: 978-5-02-039082-9.

\bibitem{Brbk70} Bourbaki, N. Alg\`ebre: Chapitres 1 \`a~3. \'El\'ements de~math\'ematique~/ N.~Bourbaki; Berlin~: Springer, 2006.~-- 652~p.~-- ISBN: 978-3-540-33849-9.

\end{thebibliography}

\makeRusTit   %% DON'T CHANGE!!!
\end{document}

