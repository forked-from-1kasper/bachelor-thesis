\documentclass{article}

\usepackage{indentfirst}

\usepackage[T2A]{fontenc}
\usepackage[utf8]{inputenc}
\usepackage[russian]{babel}

\usepackage[tmargin=1in,bmargin=1in,lmargin=1.25in,rmargin=1.25in]{geometry}

\begin{document}
  \textbf{Слайд~1.} Здравствуйте. Меня зовут Мишко Николай, мой научный руководитель~— Август Карлович Цих,
  тема моей бакалаврской работы~— «Алгебраические подгруппы комплексного тора».

  \textbf{Слайд~2.} Важнейшим инструментом одномерного комплексного анализа является вычет.
  Его многомерное обобщение, вычет Гротендика, представляет не~меньший интерес.
  Однако в~отличие от~одномерного вычета, область интегрирования — цикл Гротендика — в~многомерном случае зависит от~выбора знаменателя,
  то~есть функций $g_1$ — $g_n$. Более того, устройство, например, топологическое, цикла Гротендика неочевидно, что представляет
  известную трудность.

  Задачу можно значительно упростить, если найти набор циклов, устроенных проще, линейная комбинация которых гомологична исходному,
  например, набор торов. Торы среди прочих циклов выделяются тем, что имеют структуру алгебраической группы.

  Под~алгебраическим многообразием понимаем множество решений системы алгебраических уравнений.
  Когда на~алгебраическом многообразии задана дополнительная алгебраическая структура, например, группы,
  говорим, соответственно, об~алгебраической группе.
  Аналогично, под~алгебраической подгруппой понимаем алгебраическое многообразие, являющееся подгруппой в~заданной группе.

  Рассмотрим подгруппы комплексного тора, то~есть степени мультипликативной группы, подробнее.

  \textbf{Слайд~3.} Теорема, доказанная Шмидтом, даёт полную характеризацию алгебраических подгрупп тора
  над~произвольным полем. А~именно, если дана алгебраическая подгруппа $H$ тора, то~существует конечный набор
  показателей $\alpha_i$ и~$\beta_i$ такой, что $H$ задаётся системой биномиальных уравнений $z^{\alpha_i} = z^{\beta_i}$,
  где под~вектором в~степени вектора понимаем произведение степеней компонент.

  \textbf{Слайд~4.} Несмотря на~то, что изучение биномиальных уравнений представляет
  меньшую трудность, чем изучение полиномиальных систем общего вида, поскольку, например,
  легко видеть, что амёба поверхности, задаваемой биномиальной системой, является плоскостью,
  проходящей через начало системы координат, такое представление всё ещё не~слишком удобно.
  При~интегрировании значительно удобнее иметь параметризацию поверхности, чем систему уравнений.

  Пользуясь теоремой Шмидта, изучим существование параметризаций для~алгебраических подгрупп тора.
  Естественным кандидатом на~роль такой параметризации, в~свете упомянутой теоремы, выступает функция,
  задаваемая набором мономов, которую мы обозначаем $\phi_\alpha$, где $\alpha$~— целочисленная матрица,
  в~строках которой стоят степени определяющих её мономов. Такую параметризацию будем называть мономиальной.

  \textbf{Слайд~5.} В~качестве основного инструмента изучения мономиальных параметризаций будем использовать
  так называемую нормальную форму Смита.

  Теорема утверждает, что для~произвольной целочисленной матрицы существуют унимодулярные
  матрицы $\beta_1$ и~$\beta_2$ такие, что произведение $\beta_1$ на~исходную матрицу и~на~$\beta_2$ равно
  матрице, у~которой на~главной диагонали стоят $r$ чисел $\varepsilon_i$, называемых инвариантными факторами,
  а~остальные элементы равны нулю.

  \textbf{Слайд~6.} Прежде всего, для~применений в~интегрировании важно изучить инъективность описанной параметризации.
  Нетрудно видеть, что инъективность $\phi_\alpha$ определяется инъективностью $\phi_\varepsilon$,
  то~есть $\phi_\alpha$ инъективно тогда и~только тогда, когда инъективно $\phi_\varepsilon$.

  Далее, если $r$ строго меньше $k$, то~$\phi_\varepsilon$ не~зависит от~последних $k - r$ переменных, и~потому
  не~является инъективным. Кроме того, в~поле комплексных чисел существуют нетривиальные корни из~единицы любой
  степени, кроме $0$ и~$\pm 1$, поэтому отображения $z \mapsto z^{\varepsilon_i}$ инъективны лишь когда $\varepsilon_i = 1$.

  В~книге «Гипергеометрические и~алгебраические функции многих переменных» доказывается, что это условие выполнено тогда
  и~только тогда, когда $\alpha_i$ порождают всю решётку, то~есть когда любой целочисленный $k$-мерный вектор выражается
  как линейная комбинация $\alpha_i$.
  Таким образом, получаем критерий инъективности: $\phi_\alpha$ инъективно тогда и~только тогда, когда $\alpha_i$ порождают всю решётку.

  \textbf{Слайд~7.} Рассмотрим теперь вопросы существования такого рода параметризаций для~алгебраических подгрупп тора.
  Во-первых, заметим, что если в~системе биномиальных уравнений разделить каждое уравнений, например, на~его правую часть,
  получим уравнения, в~точности определяющие ядро некоторого отображения $\phi_\beta$. Удобно рассматривать задание алгебраических
  групп именно в~таком виде.

  Во-вторых, перед тем, как искать мономиальную параметризацию, следует проверить, что образ такого отображения является алгебраической
  группой. Если он алгебраической группой не~является, то~никакую группу он, соответственно, параметризовать и~не~может.

  Так вот, оказывается, что имеет место теорема: над~алгебраически замкнутым полем, в~частности, над~полем комплексных чисел, образ любой
  мономиальной параметризации является алгебраической подгруппой тора.

  Поэтому имеет смысл рассмотреть обратную задачу: представить ядро некоторого $\phi_\beta$ как образ некоторого $\phi_\alpha$,
  другими словами, найти параметризацию по~зафиксированной подгруппе тора. Однако, например, кривая $z^2 = 1$ над~комплексными числами
  не~является связным множеством, а~образ мономиальной параметризации~— всегда связное множество, поскольку является непрерывным
  образом связного множества.

  Рассмотрим условия, при~которых мономиальная параметризация существует. Для~этого обозначим как $\omega_K(n)$ группу корней из~единицы
  степени $n$ над~полем $K$ и~как $\Pi(\beta)$ произведение порядков таких групп для~степеней $\varepsilon_i$, инвариантных факторов матрицы $\beta$.

  Во-первых, это число не~зависит от~конкретного вида матрицы $\beta$. Более точно, равенство ядер отображений $\phi_\beta$ и~$\phi_\beta'$
  влечёт равенство $\Pi(\beta)$ и~$\Pi(\beta')$. Это означает, что $\Pi$ можно приписать алгебраической подгруппе тора вне~зависимости
  от~выбора конкретного вида матрицы $\beta$.

  Наконец, из~существования нормальной формы Смита вытекает следующая теорема: если $\Pi(H)$ равняется единице для~алгебраической подгруппы
  тора, то для~$H$ существует мономиальная параметризация. Более того, матрица, задающая параметризацию, строится явным образом из~уравнений,
  задающих группу $H$.

  \textbf{Слайд~8.} Как мы видели, инвариант $\Pi$ определяется из~чисто алгебраических соображений в~произвольном поле $K$.
  Произвольное поле не~обладает естественной топологической структурой, однако стандартная топология имеется в~поле комплексных чисел.

  Справедлива теорема: число связных компонент в~$H$, алгебраической подгруппе комплексного алгебраического тора, равно $\Pi(H)$.

  Как было отмечено ранее, образ мономиальной параметризации связен, откуда немедленно вытекает следствие:
  алгебраическая подгруппа комплексного тора имеет мономиальную параметризацию тогда и~только тогда,
  когда она связна.

  Также можно заметить, что все компоненты связности подгруппы тора гомеоморфны, а~компонента связности,
  содержащая единицу, является подгруппой в~исходной группе.

  Поэтому, несмотря на~то, что в~общем случае для~подгруппы тора не~существует мономиальной параметризации,
  имеет место следствие из~теоремы: для~компоненты связности алгебраической подгруппы комплексного тора, содержащей
  единицу, всегда существует мономиальная параметризация.

  \textbf{Слайд~9.} Таким образом, в~ходе работы были: <и~далее по слайду>.

  \textbf{Слайд~10.} Спасибо за~внимание, теперь я готов ответить на~ваши вопросы.

\end{document}