\documentclass{article}

\usepackage[utf8]{inputenc}
\usepackage[T2A]{fontenc}
\usepackage[russian]{babel}
\usepackage[tmargin=1in,bmargin=1in,lmargin=1.25in,rmargin=1.25in]{geometry}

\usepackage{tempora}
\usepackage[fontsize=14pt]{fontsize}

\usepackage[normalem]{ulem}

\usepackage[threshold=0]{csquotes}
\usepackage{indentfirst}
\usepackage{enumerate}
\usepackage{titlesec}
\usepackage{etoolbox}
\usepackage{amsmath}
\usepackage{amssymb}
\usepackage{array}
\usepackage{mathtools}

\usepackage{stackengine}

\setlength{\parindent}{0in}

\def\paddedtext#1#2{\leavevmode\hbox to#1{\hss#2\hss}\ignorespaces}
\def\rightpaddedtext#1#2{\leavevmode\hbox to#1{#2\hss}\ignorespaces}

\newcolumntype{L}[1]{>{\raggedright\let\newline\\\arraybackslash\hspace{0pt}}m{#1}}
\newcolumntype{C}[1]{>{\centering\let\newline\\\arraybackslash\hspace{0pt}}m{#1}}
\newcolumntype{R}[1]{>{\raggedleft\let\newline\\\arraybackslash\hspace{0pt}}m{#1}}

\pagenumbering{gobble}

\newcommand{\subscript}[1]{\fontsize{10pt}{0}\selectfont#1}
\newcommand{\markedfield}[3]{\uline{\hfill\stackunder[#3]{#1}{\subscript{#2}}\hfill}}

\begin{document}

\begin{titlepage}
  \centering

  Министерство науки и высшего образования РФ\par
  \vspace{1pt}

  {\fontsize{12pt}{0}\selectfont
  Федеральное государственное автономное\par
  образовательное учреждение высшего образования\par
  \textbf{«СИБИРСКИЙ ФЕДЕРАЛЬНЫЙ УНИВЕРСИТЕТ»}}

  \vspace{0.3cm}
  Институт математики и фундаментальной информатики\par
  Кафедра теории функций

  \vspace{0.5cm}

  \begin{flushright}
    \begin{tabular}{l@{}}
      УТВЕРЖДАЮ\\
      Заведующий кафедрой\\
      \uline{\stackunder[6pt]{\hspace{2.5cm}}{\subscript{подпись}}}
      \markedfield{А. К. Цих}{фамилия, инициалы}{2pt} \\
      «\underline{\hspace{0.7cm}}» \underline{\hspace{2.5cm}} 2024~г.
    \end{tabular}
  \end{flushright}

  \vspace{3cm}

  \textbf{ЗАДАНИЕ}\par
  \textbf{НА ВЫПУСКНУЮ КВАЛИФИКАЦИОННУЮ РАБОТУ}\par
  \textbf{в форме бакалаврской работы}

  \vfill
  Красноярск, 2024
\end{titlepage}

Студенту \markedfield{Мишко Николаю Алексеевичу}{фамилия, имя, отчество}{1pt}

Группа \markedfield{ИМ20-04Б}{номер}{6pt} Направление (специальность) \markedfield{01.03.01}{код}{6pt}

\markedfield{Математика}{полное наименование}{6pt}

\vspace{2pt}

Тема выпускной квалификационной работы \uline{Алгебраические подгруппы \hfill}

\uline{комплексного тора \hfill}

\uline{\hfill}

Утверждена приказом по~университету № \uline{\paddedtext{1cm}{}} от \uline{\hfill}

Руководитель ВКР: \uline{А. К. Цих, доктор физико-математических наук, \hfill}

\uline{\hfill профессор, кафедра теории функций \hfill}

\markedfield{Сибирского федерального университета}{инициалы, фамилия, должность, ученое звание и место работы}{1pt}

\vspace{2pt}

Исходные данные для~ВКР: \uline{1.~W.~M.~Schmidt, Heights of points on \hfill}

\uline{subvarieties of $\mathbb{G}^n_m$. In~Number Theory 93–94, ed. by~S.~David, London Math. \hfill}

\uline{Soc. Lecture Note Ser.~235, Cambridge University Press, Cambridge 1996, 157–187. \hfill}

\uline{2.~Садыков Т. М., Цих А. К. Гипергеометрические и алгебраические \hfill}

\uline{функции многих переменных. М.: Наука, 2014. — 408 с. — ISBN: \hfill}

\uline{978-5-02-039082-9. \hfill}

Перечень разделов ВКР: \uline{1.~Введение (о~алгебраических группах). \hfill}

\uline{2.~О~задании подгрупп тора биномиальными уравненями. \hfill}

\uline{3.~Мономиальные параметризации и~их инъективность. \hfill}

\uline{4.~Теорема о~существовании параметризации. \hfill}

\uline{5.~Список использованных источников. 6.~Приложение. \hfill}

Перечень графического материала: \uline{Отсутствует. \hfill}

\uline{\hfill}

\uline{\hfill}

\uline{\hfill}

\uline{\hfill}

\vfill

\makebox[0.42\textwidth][l]{Руководитель ВКР} \markedfield{}{подпись}{8pt} \markedfield{А. К. Цих}{инициалы и фамилия}{4pt}

\vspace{2pt}

\makebox[0.42\textwidth][l]{Задание принял к исполнению} \markedfield{\hspace{1.3cm} Н. А. Мишко}{подпись, инициалы и фамилия студента}{4pt}

\vspace{1cm}

\hfill «\underline{\paddedtext{0.7cm}{}}» \underline{\paddedtext{2.5cm}{}} 2024~г.

\end{document}