\documentclass[a4paper,oneside]{article}

\usepackage[utf8]{inputenc}
\usepackage[T2A]{fontenc}
\usepackage[russian]{babel}
\usepackage[tmargin=20mm,bmargin=20mm,lmargin=30mm,rmargin=10mm]{geometry}

\usepackage{tempora}
\usepackage[fontsize=14pt]{fontsize}

\usepackage[normalem]{ulem}

\usepackage[threshold=0]{csquotes}
\usepackage{indentfirst}
\usepackage{enumerate}
\usepackage{titlesec}
\usepackage{etoolbox}
\usepackage{amsmath}
\usepackage{amssymb}
\usepackage{array}
\usepackage{mathtools}

\usepackage{stackengine}

\setlength{\parindent}{0pt}
\setlength{\parskip}{0pt}

\def\paddedtext#1#2{\leavevmode\hbox to#1{\hss#2\hss}\ignorespaces}
\def\rightpaddedtext#1#2{\leavevmode\hbox to#1{#2\hss}\ignorespaces}

\newcolumntype{L}[1]{>{\raggedright\let\newline\\\arraybackslash\hspace{0pt}}m{#1}}
\newcolumntype{C}[1]{>{\centering\let\newline\\\arraybackslash\hspace{0pt}}m{#1}}
\newcolumntype{R}[1]{>{\raggedleft\let\newline\\\arraybackslash\hspace{0pt}}m{#1}}

\pagenumbering{gobble}

\newcommand{\subscript}[1]{\fontsize{10pt}{0}\selectfont#1}
\newcommand{\markedfield}[3]{\uline{\hfill\stackunder[#3]{#1}{\subscript{#2}}\hfill}}

\begin{document}

\begin{titlepage}
  \centering

  Министерство науки и высшего образования РФ\par
  \vspace{1pt}

  {\fontsize{12pt}{0}\selectfont
  Федеральное государственное автономное\par
  образовательное учреждение высшего образования\par
  \textbf{«СИБИРСКИЙ ФЕДЕРАЛЬНЫЙ УНИВЕРСИТЕТ»}}

  \vspace{0.3cm}
  Институт математики и фундаментальной информатики\par
  Кафедра теории функций

  \vspace{0.5cm}

  \begin{flushright}
    \begin{tabular}{l@{}}
      УТВЕРЖДАЮ\\
      Заведующий кафедрой\\
      \uline{\stackunder[6pt]{\hspace{2.5cm}}{\subscript{подпись}}}
      \markedfield{А. К. Цих}{фамилия, инициалы}{2pt} \\
      «\underline{\hspace{0.7cm}}» \underline{\hspace{2.5cm}} 2024~г.
    \end{tabular}
  \end{flushright}

  \vspace{3cm}

  \textbf{ЗАДАНИЕ}\par
  \textbf{НА ВЫПУСКНУЮ КВАЛИФИКАЦИОННУЮ РАБОТУ}\par
  \textbf{в форме бакалаврской работы}

  \vfill
  Красноярск, 2024
\end{titlepage}

Студенту \markedfield{Мишко Николаю Алексеевичу}{фамилия, имя, отчество}{1pt}

Группа \markedfield{ИМ20-04Б}{номер}{6pt} Направление (специальность) \markedfield{01.03.01}{код}{6pt}

\uline{\hfill Математика \hfill}\par{\vspace{2pt}\centering\subscript{полное наименование}\par}

Тема выпускной квалификационной работы \uline{Алгебраические подгруппы ком\-пле\-ксно\-го тора \hfill}

Утверждена приказом по~университету № \uline{\paddedtext{1cm}{}} от \uline{\hfill}

Руководитель ВКР: \uline{А. К. Цих, доктор физико-математических наук, профессор, кафедра теории функций Сибирского федерального университета \hfill}\par
{\centering\subscript{инициалы, фамилия, должность, ученое звание и место работы}\par}

Исходные данные для~ВКР:
\uline{1.~Schmidt, W.~M. Heights of points on subvarieties of $\mathbb{G}^n_m$~/ W.~M.~Schmidt~//
       London Mathematical Society Lecture Note Series. Issue~235: Number Theory.
       Séminaire de~théorie des~nombres de~Paris 1993–94.~— 1996.~— P.~157—187.
       2.~Садыков, Т.~М. Гипергеометрические и~алгебраические функции многих переменных~/
       Т.~М.~Садыков, А.~К.~Цих; Москва : Наука, 2014.~— 408~с.~— ISBN: 978-5-02-039082-9. \hfill}

Перечень разделов ВКР:
\uline{1.~Введение. 2.~Задание алгебраических подгрупп тора.
       3.~Мономиальные параметризации и~их свойства. 4.~Линейная независимость над
       абелевой группой. 5.~Голоморфные характеры. 6.~Дальнейшие перспективы.
       7.~Список использованных источников. 8.~Приложение. \hfill}

Перечень графического материала: \uline{Отсутствует. \hfill}

\uline{\hfill}

\uline{\hfill}

\uline{\hfill}

\uline{\hfill}

\vfill

\makebox[0.42\textwidth][l]{Руководитель ВКР} \markedfield{}{подпись}{8pt} \markedfield{А. К. Цих}{инициалы и фамилия}{4pt}

\vspace{2pt}

\makebox[0.42\textwidth][l]{Задание принял к исполнению} \markedfield{\hspace{1.3cm} Н. А. Мишко}{подпись, инициалы и фамилия студента}{4pt}

\vspace{1cm}

\hfill «\underline{\paddedtext{0.7cm}{}}» \underline{\paddedtext{2.5cm}{}} 2024~г.

\end{document}